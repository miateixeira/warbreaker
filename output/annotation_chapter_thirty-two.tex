\section{Annotation Chapter Thirty-Two}

\subsection*{Siri Lies in Bed and Decides to Take Charge}

Reading through this scene again, I feel like it needs a bit of a trim. Ah well. There are always going to be sections like that that make it through.

I felt that there needed to be a scene where Siri finally stopped looking toward the past and berating herself for not being more like Vivenna. For her to step forward and become the woman she must be, she needed to do it of her own choice, with her own motivations. She needed this chance.

Sometimes in writing classes or in books on telling stories, they’ll mention a moment somewhere in act two where the character decides to take charge. I always dislike explanations like that, since I think it’s too easy for newer writers to look at such explanations as an item on a checklist that you have to do. I never use things like that. I don’t think, “This is act two, so the characters need to do X.” The tendency to follow a formula like that is part of what bothers me about the screenwriting profession. It seems like if you always follow the rules, there’s never any spontaneity in a book.

Still, those guidelines and suggestions are used by a lot of people who tell good stories, so I guess you use what works for you.

\subsection*{Hoid the Storyteller Tells Us the History of Hallandren}

This whole scene came about because I wanted an interesting way to delve into the history. Siri needed to hear it, and I felt that many readers would want to know it. However, that threatened to put me into the realm of the dreaded infodump.

And so, I brought in the big guns. This cameo is so obvious (or, at least, someday it will be) that I almost didn’t use the name Hoid for the character, as I felt it would be \textit{too} obvious. The first draft had him using another of his favorite pseudonyms. However, in the end, I decided that too many people would be confused (or at least even more confused) if I didn’t use the same name. So here it is. And if you have no idea what I’m talking about . . . well, let’s just say that there’s a lot more to this random appearance than you might think.

Anyway, I love this storytelling method, and I worry that Hoid here steals the show. However, he’s very good at what he does, and I think it makes for a very engaging scene that gets us the information we need without boring us out of our skulls.

Is everything he says here true? No. There are some approximations and some guesses. However, all things considered, it’s pretty accurate. All of the large bits are true.

I wasn’t sure if I wanted a map of the world in the front of this book or not. The problem is, if I give a world map, I risk doing it wrong. It takes a very specific set of geographic requirements for a rain forest to work, and what I wanted here was a kind of rain forest valley, irregular and out of place in the world. In the abstract, that can work—but the more details I pin down in the map, the less likely it is to be believable.



