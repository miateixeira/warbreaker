\section{Annotation Prologue Part 2}

There are a couple of other relics from the original short story version of this chapter that made it into the final book. I toyed with cutting these, but figured that there were good reasons for them.

\subsection*{The Guard Approaches, and His Clothing Becomes Brighter}

This is an essential part of the magic system. When you get close to someone’s aura, their clothing—and everything else about them—brightens in color slightly. It’s important to show it in this prologue.

Unfortunately, it also shouldn’t be there. You see, Vasher should be smart enough to hide his Breath in his clothing, as the book later shows is quite easy to do. He shouldn’t have left himself holding any Breath. It’s suspicious. If those guards \textit{had} noticed his aura—or if someone working in the prison had been of the First Heightening—Vasher would have been spotted. It’s such an easy fix that he should have thought of it.

The problem is, I felt I needed to establish the way the magic works from the beginning. Having to explain why Vasher didn’t make the clothing glow would have been awkward and confusing at this point in the book. So I left this as it is.

However, being who I am, I developed a background for why Vasher did it this way. He left his Breath in, and thought that maybe it \textit{would} be noticed—but if it was, he knew that the guards would lock him in a cell much closer to Vahr. That would be convenient, as it would ensure that he was much closer to his quarry. Of course, in such a cell, he wouldn’t be able to Awaken anything and escape. However, he’d planned for that too. He set a little straw figure outside the prison the night before, with specific Commands instructing it to search through the cells and find him, delivering a set of lock picks.

It was risky—but either way he did it would be risky. He couldn’t know for certain that the guards would take him to the area he needed to be in, and even if he \textit{had} hidden his Breath in his clothing, some prisons have rules in place requiring each prisoner to be stripped, just in case they’ve done just that. Fortunately, these guards were particularly lazy. Anyway, Vasher’s contingency plan wasn’t needed, as the guards didn’t end up noticing his Breath.

\subsection*{Vasher Awakens the Straw Figure}

I love how intricate and delicate Vasher is in creating the straw figure. The little eyebrow is a nice touch, and forming the creature into the shape of a person has a nice resonance with our own world’s superstitions.

Voodoo dolls, for instance. This is very common in tribal magics and shamanistic rituals—something in the figure of a person, or the figure of the thing it’s supposed to affect, is often seen as being more powerful or more desirable. The same is said for having a drop of blood or a tiny piece of skin, even a piece of hair.

Those two things—making the doll in the shape of a man and using a bit of his own body as a focus—are supposed to create instant resonance in the magic for those reading it. I think it works, too. Unfortunately, there’s a problem with this, much like with the colors above. In later chapters, the characters are generally powerful enough with the magic that they don’t \textit{have} to make things in human shape or use pieces of their own body as a focus.

If I were to write a sequel to the book (and I just might—more on this later) I’d want to get back to these two aspects of the magic. Talk about them more, maybe have characters who have smaller quantities of Breath, and so need to use these tricks to make their Awakening more powerful.

Anyway, this little scene threw all kinds of problems into the book. Later on, I had to decide if I wanted to force the characters to always make things into the shape of a person before Awakening them. That proved impossible, it was \textit{too }limiting on the magic and interfered with action sequences. The same was true for using bits of their own flesh as focuses. It just didn’t work.

I toyed with cutting these things from the prologue. (Again, they are artifacts from the short story I wrote, back when Awakening wasn’t fully developed yet.) However, I like the resonance they give, and think they add a lot of depth to the magic system.

So I made them optional. They’re things that you \textit{can} do to make your Awakenings require fewer Breaths. That lets me have them for resonance, but not talk about them when I don’t need them. I still worry that they set up false expectations for the magic, however.



