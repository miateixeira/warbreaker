\section{Chapter One}

There were great advantages to being unimportant.

True, by many people’s standards, Siri wasn’t “unimportant.” She was, after all, the daughter of a king. Fortunately, her father had four living children, and Siri—at seventeen years of age—was the youngest. Fafen, the daughter just older than Siri, had done the family duty and become a monk. Above Fafen was Ridger, the eldest son. He would inherit the throne.

And then there was Vivenna. Siri sighed as she walked down the path back to the city. Vivenna, the firstborn, was~.~.~. well~.~.~. Vivenna. Beautiful, poised, perfect in most every way. It was a good thing, too, considering the fact that she was betrothed to a god. Either way, Siri—as fourth child—was redundant. Vivenna and Ridger had to focus on their studies; Fafen had to do her work in the pastures and homes. Siri, however, could get away with being unimportant. That meant she could disappear into the wilderness for hours at a time.

People would notice, of course, and she~\textit{would}~get into trouble. Yet even her father would have to admit that her disappearance hadn’t caused much inconvenience. The city got along just fine without Siri—in fact, it tended to do a little better when she wasn’t around.

Unimportance. To another, it might have been offensive. To Siri it was a blessing. She smiled, walking into the city proper. She drew the inevitable stares. While Bevalis was technically the capital of Idris, it wasn’t that big, and everyone knew her by sight. Judging by the stories Siri had heard from passing ramblemen, her home was hardly even a village compared with the massive metropolises in other nations.

She liked it the way it was, even with the muddy streets, the thatched cottages, and the boring—yet sturdy—stone walls. Women chasing runaway geese, men pulling donkeys laden with spring seed, and children leading sheep on their way to pasture. A grand city in Xaka, Hudres, or even terrible Hallandren might have exotic sights, but it would be crowded with faceless, shouting, jostling crowds, and haughty noblemen. Not Siri’s preference; she generally found even Bevalis to be a bit busy for her.

\textit{Still,}~she thought, looking down at her utilitarian grey dress,~\textit{I’ll bet those cities have more colors. That’s something I might like to see.}

Her hair wouldn’t stand out so much there. As usual, the long locks had gone blond with joy while she’d been out in the fields. She concentrated, trying to rein them in, but she was only able to bring the color to a dull brown. As soon as she stopped focusing, her hair just went back to the way it had been. She’d never been very good at controlling it. Not like Vivenna.

As she continued through the town, a group of small figures began trailing her. She smiled, pretending to ignore the children until one of them was brave enough to run forward and tug on her dress. Then she turned, smiling. They regarded her with solemn faces. Idrian children were trained even at this age to avoid shameful outbursts of emotion. Austrin teachings said there was nothing wrong with feelings, but drawing attention to yourself with them was wrong.

Siri had never been very devout. It wasn’t her fault, she reasoned, if Austre had made her with a distinct inability to obey. The children waited patiently until Siri reached into her apron and pulled out a couple of brightly colored flowers. The children’s eyes opened wide, gazing at the vibrant colors. Three of the flowers were blue, one yellow.

The flowers stood out starkly against the town’s determined drabness. Other than what one could find in the skin and eyes of the people, there wasn’t a drop of color in sight. Stones had been whitewashed, clothing bleached grey or tan. All to keep the color away.

For without color, there could be no Awakeners.

The girl who had tugged Siri’s skirt finally took the flowers in one hand and dashed away with them, the other children following behind. Siri caught a look of disproval in the eyes of several passing villagers. None of them confronted her, though. Being a princess—even an unimportant one—did have its perks.

She continued on toward the palace. It was a low, single-story building with a large, packed-earth courtyard. Siri avoided the crowds of haggling people at the front, rounding to the back and going in the kitchen entrance. Mab, the kitchen mistress, stopped singing as the door opened, then eyed Siri.

“Your father’s been looking for you, child,” Mab said, turning away and humming as she attacked a pile of onions.

“I suspect that he has.” Siri walked over and sniffed at a pot, which bore the calm scent of boiling potatoes.

“Went to the hills again, didn’t you? Skipped your tutorial sessions, I’ll bet.”

Siri smiled, then pulled out another of the bright yellow flowers, spinning it between two fingers.

Mab rolled her eyes. “And been corrupting the city youth again, I suspect. Honestly, girl, you should be beyond these things at your age. Your father will have words with you about shirking your responsibilities.”

“I like words,” Siri said. “And I always learn a few new ones when Father gets angry. I shouldn’t neglect my education, now should I?”

Mab snorted, dicing some pickled cucumbers into the onions.

“Honestly, Mab,” Siri said, twirling the flower, feeling her hair shade a little bit red. “I don’t see what the problem is. Austre made the flowers, right? He put the colors on them, so they can’t be evil. I mean, we call him God of Colors, for heaven’s sake.”

“Flowers ain’t evil,” Mab said, adding something that looked like grass to her concoction, “assuming they’re left where Austre put them. We shouldn’t use Austre’s beauty to make ourselves more important.”

“A flower doesn’t make me look more important.”

“Oh?” Mab asked, adding the grass, cucumber, and onions to one of her boiling pots. She banged the side of the pot with the flat of her knife, listening, then nodded to herself and began fishing under the counter for more vegetables. “You tell me,” she continued, voice muffled. “You really think walking through the city with a flower like that didn’t draw attention to yourself?”

“That’s only because the city is so drab. If there were a bit of color around, nobody would notice a flower.”

Mab reappeared, hefting a box filled with various tubers. “You’d have us decorate the place like Hallandren? Maybe we should start inviting Awakeners into the city? How’d you like that? Some dev il sucking the souls out of children, strangling people with their own clothing? Bringing men back from the grave, then using their dead bodies for cheap labor? Sacrificing women on their unholy altars?”

Siri felt her hair whiten slightly with anxiety.~\textit{Stop that!}~she thought. The hair seemed to have a mind of its own, responding to gut feelings.

“That sacrificing-maidens part is only a story,” Siri said. “They don’t really do that.”

“Stories come from somewhere.”

“Yes, they come from old women sitting by the hearth in the winter. I don’t think we need to be so frightened. The Hallandren will do what they want, which is fine by me, as long as they leave us alone.”

Mab chopped tubers, not looking up.

“We’ve got the treaty, Mab,” Siri said. “Father and Vivenna will make sure we’re safe, and that will make the Hallandren leave us alone.”

“And if they don’t?”

“They will. You don’t need to worry.”

“They have better armies,” Mab said, chopping, not looking up, “better steel, more food, and those~.~.~. those~\textit{things.}~It makes people worry. Maybe not~\textit{you,}~but sensible folk.”

The cook’s words were hard to dismiss out of hand. Mab had a sense, a wisdom beyond her instinct for spices and broths. However, she~\textit{also}~tended to fret. “You’re worrying about nothing, Mab. You’ll see.”

“I’m just saying that this is a bad time for a royal princess to be running around with flowers, standin’ out and inviting Austre’s dislike.”

Siri sighed. “Fine, then,” she said, tossing her last flower into the stewpot. “Now we can all stand out together.”

Mab froze, then rolled her eyes, chopping a root. “I assume that was a vanavel flower?”

“Of course,” Siri said, sniffing at the steaming pot. “I know better than to ruin a good stew. And I still say you’re overreacting.”

Mab sniffed. “Here,” she said, pulling out another knife. “Make yourself useful. There’s roots that need choppin’.”

“Shouldn’t I report to my father?” Siri said, grabbing a gnarled vanavel root and beginning to chop.

“He’ll just send you back here and make you work in the kitchens as a punishment,” Mab said, banging the pot with her knife again. She firmly believed that she could judge when a dish was done by the way the pot rang.

“Austre help me if Father ever discovers I like it down here.”

“You just like being close to the food,” Mab said, fishing Siri’s flower out of the stew, then tossing it aside. “Either way, you can’t report to him. He’s in conference with Yarda.”

Siri gave no reaction—she simply continued to chop. Her hair, however, grew blond with excitement.~\textit{Father’s conferences with Yarda usually last hours,}~she thought.~\textit{Not much point in simply sitting around, waiting for him to get done.~.~.~.}

Mab turned to get something off the table, and before she looked back, Siri bolted out the door on her way toward the royal stables. Bare minutes later, she galloped away from the palace, wearing her favorite brown cloak, feeling an exhilarated thrill that sent her hair into a deep blond. A nice quick ride would be a good way to round out the day.

After all, her punishment was likely to be the same either way.

\bigskip \hrule \bigskip

Dedelin, king of Idris, set the letter down on his desk. He had stared at it long enough. It was time to decide whether or not to send his eldest daughter to her death.

Despite the advent of spring, his chamber was cold. Warmth was a rare thing in the Idris highlands; it was coveted and enjoyed, for it lingered only briefly each summer. The chambers were also stark. There was a beauty in simplicity. Even a king had no right to display arrogance by ostentation.

Dedelin stood up, looking out his window and into the courtyard. The palace was small by the world’s standards—only a single story high, with a peaked wooden roof and squat stone walls. But it was large by Idrian standards, and it bordered on flamboyant. This could be forgiven, for the palace was also a meeting hall and center of operations for his entire kingdom.

The king could see General Yarda out of the corner of his eye. The burly man stood waiting, his hands clasped behind his back, his thick beard tied in three places. He was the only other person in the room.

Dedelin glanced back at the letter. The paper was a bright pink, and the garish color stood out on his desk like a drop of blood in the snow. Pink was a color one would never see in Idris. In Hallandren, however—center of the world’s dye industry—such tasteless hues were commonplace.

“Well, old friend?” Dedelin asked. “Do you have any advice for me?”

General Yarda shook his head. “War is coming, Your Majesty. I feel it in the winds and read it in the reports of our spies. Hallandren still considers us rebels, and our passes to the north are too tempting. They will attack.”

“Then I shouldn’t send her,” Dedelin said, looking back out his window. The courtyard bustled with people in furs and cloaks coming to market.

“We can’t stop the war, Your Majesty,” Yarda said. “But~.~.~. we can slow it.”

Dedelin turned back.

Yarda stepped forward, speaking softly. “This is not a good time. Our troops still haven’t recovered from those Vendis raids last fall, and with the fires in the granary this winter~.~.~.” Yarda shook his head. “We~\textit{cannot}~afford to get into a defensive war in the summer. Our best ally against the Hallandren is the snow. We can’t let this conflict occur on their terms. If we do, we are dead.”

The words all made sense.

“Your Majesty,” Yarda said, “they are~\textit{waiting}~for us to break the treaty as an excuse to attack. If we move first, they will strike.”

“If we keep the treaty, they will~\textit{still}~strike,” Dedelin said.

“But later. Perhaps months later. You know how slow Hallandren politics are. If we keep the treaty, there will be debates and arguments. If those last until the snows, then we will have gained the time we need so badly.”

It all made sense. Brutal, honest sense. All these years, Dedelin had stalled and watched as the Hallandren court grew more and more aggressive, more and more agitated. Every year, voices called for an assault on the “rebel Idrians” living up in the highlands. Every year, those voices grew louder and more plentiful. Every year, Dedelin’s placating and politics kept the armies away. He had hoped, perhaps, that the rebel leader Vahr and his Pahn Kahl dissidents would draw attention away from Idris, but Vahr had been captured, his so-called army dispersed. His actions had only served to make Hallandren more focused on its enemies.

The peace would not last. Not with Idris ripe, not with the trade routes worth so much. Not with the current crop of Hallandren gods, who seemed so much more erratic than their predecessors. He~\textit{knew}~all of that. But he also knew that breaking the treaty would be foolish. When you were cast into the den of a beast, you did not provoke it to anger.

Yarda joined him beside the window, looking out, leaning one elbow against the side of the frame. He was a harsh man born of harsh winters. But he was also as good a man as Dedelin had ever known—a part of the king longed to marry Vivenna to the general’s own son.

That was foolishness. Dedelin had always known this day would come. He’d crafted the treaty himself, and it demanded he send his daughter to marry the God King. The Hallandren needed a daughter of the royal blood to reintroduce the traditional bloodline into their monarchy. It was something the depraved and vainglorious people of the lowlands had long coveted, and only that specific clause in the treaty had saved Idris these twenty years.

That treaty had been the first official act of Dedelin’s reign, negotiated furiously following his father’s assassination. Dedelin gritted his teeth. How quickly he’d bowed before the whims of his enemies. Yet he would do it again; an Idrian monarch would do anything for his people. That was one big difference between Idris and Hallandren.

“If we send her, Yarda,” Dedelin said, “we send her to her death.”

“Maybe they won’t harm her,” Yarda finally said.

“You know better than that. The first thing they’ll do when war comes is use her against me. This is~\textit{Hallandren.}~They invite Awakeners into their palaces, for Austre’s sake!”

Yarda fell silent. Finally, he shook his head. “Latest reports say their army has grown to include some forty thousand Lifeless.”

\textit{Lord God of Colors,}~Dedelin thought, glancing at the letter again. Its language was simple. Vivenna’s twenty-second birthday had come, and the terms of the treaty stipulated that Dedelin could wait no longer.

“Sending Vivenna is a poor plan, but it’s our only plan,” Yarda said. “With more time, I know I can bring the Tedradel to our cause—they’ve hated Hallandren since the Manywar. And perhaps I can find a way to rile Vahr’s broken rebel faction in Hallandren itself. At the very least, we can build, gather supplies, live another year.” Yarda turned to him. “If we don’t send the Hallandren their princess, the war will be seen as our fault. Who will support us? They will demand to know why we refused to follow the treaty our own king wrote!”

“And if we do send them Vivenna, it will introduce the royal blood into their monarchy, and that will have an even~\textit{more}~legitimate claim on the highlands!”

“Perhaps,” Yarda said. “But if we both know they’re going to attack anyway, then what do we care about their claim? At least this way, perhaps they will wait until an heir is born before the assault comes.”

More time. The general always asked for more time. But what about when that time came at the cost of Dedelin’s own child?

\textit{Yarda wouldn’t hesitate to send one soldier to die if it would mean time enough to get the rest of his troops into better position to attack,}~Dedelin thought.~\textit{We are Idris. How can I ask anything less of my daughter than I’d demand of one of my troops?}

It was just that thinking of Vivenna in the God King’s arms, being forced to bear that creature’s child~.~.~. it nearly made his hair bleach with concern. That child would become a stillborn monster who would become the next Returned god of the Hallandren.

\textit{There is another way,}~a part of his mind whispered.~\textit{You don’t have to send Vivenna.~.~.~.}

A knock came at his door. Both he and Yarda turned, and Dedelin called for the visitor to enter. He should have been able to guess who it would be.

Vivenna stood in a quiet grey dress, looking so young to him still. Yet she was the perfect image of an Idrian woman—hair kept in a modest knot, no makeup to draw attention to the face. She was not timid or soft, like some noblewomen from the northern kingdoms. She was just composed. Composed, simple, hard, and capable. Idrian.

“You have been in here for several hours, Father,” Vivenna said, bowing her head respectfully to Yarda. “The servants speak of a colorful envelope carried by the general when he entered. I believe I know what it contained.”

Dedelin met her eyes, then waved for her to seat herself. She softly closed the door, then took one of the wooden chairs from the side of the room. Yarda remained standing, after the masculine fashion. Vivenna eyed the letter sitting on the desk. She was calm, her hair controlled and kept a respectful black. She was twice as devout as Dedelin, and—unlike her youngest sister—she never drew attention to herself with fits of emotion.

“I assume that I should prepare myself for departure, then,” Vivenna said, hands in her lap.

Dedelin opened his mouth, but could find no objection. He glanced at Yarda, who just shook his head, resigned.

“I have prepared my entire life for this, Father,” Vivenna said. “I am ready. Siri, however, will not take this well. She left on a ride an hour ago. I should depart the city before she gets back. That will avoid any potential scene she might make.”

“Too late,” Yarda said, grimacing and nodding toward the window. Just outside, people scattered in the courtyard as a figure galloped through the gates. She wore a deep brown cloak that bordered on being too colorful, and—of course—she had her hair down.

The hair was yellow.

Dedelin felt his rage and frustration growing. Only Siri could make him lose control, and—as if in ironic counterpoint to the source of his anger—he felt his hair change. To those watching, a few locks of hair on his head would have bled from black to red. It was the identifying mark of the royal family, who had fled to the Idris highlands at the climax of the Manywar. Others could hide their emotions. The royals, however, manifested what they felt in the very hair on their heads.

Vivenna watched him, pristine as always, and her poise gave him strength as he forced his hair to turn black again. It took more willpower than any common man could understand to control the treasonous Royal Locks. Dedelin wasn’t sure how Vivenna managed it so well.

\textit{Poor girl never even had a childhood,}~he thought. From birth, Vivenna’s life had been pointed toward this single event. His firstborn child, the girl who had always seemed like a part of himself. The girl who had always made him proud; the woman who had already earned the love and respect of her people. In his mind’s eye he saw the queen she could become, stronger even than he. Someone who could guide them through the dark days ahead.

But only if she survived that long.

“I will prepare myself for the trip,” Vivenna said, rising.

“No,” Dedelin said.

Yarda and Vivenna both turned.

“Father,” Vivenna said. “If we break this treaty, it will mean war. I am prepared to sacrifice for our people. You taught me that.”

“You will~\textit{not}~go,” Dedelin said firmly, turning back toward the window. Outside, Siri was laughing with one of the stablemen. Dedelin could hear her outburst even from a distance; her hair had turned a flame-colored red.

\textit{Lord God of Colors, forgive me,}~he thought.~\textit{What a terrible choice for a father to make. The treaty is specific: I must send the Hallandren my daughter when Vivenna reaches her twenty-second birthday. But it doesn’t actually say}~which~\textit{daughter I am required to send.}

If he didn’t send Hallandren one of his daughters, they would attack immediately. If he sent the wrong one, they might be angered, but he knew they wouldn’t attack. They would wait until they had an heir. That would gain Idris at least nine months.

\textit{And~.~.~.}~he thought,~\textit{if they were to try to use Vivenna against me, I know that I wouldn’t be able to stop myself from giving in.}~It was shameful to admit that fact, but in the end, it was what made the decision for him.

Dedelin turned back toward the room. “Vivenna, you will not go to wed the tyrant god of our enemies. I’m sending Siri in your place.”

