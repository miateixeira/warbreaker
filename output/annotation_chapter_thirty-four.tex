\section{Annotation Chapter Thirty-Four}

\subsection*{Lightsong and Blushweaver Watch the Priests in the Arena}

I’m not sure why, honestly, but this discussion of Blushweaver trying to prove that God exists through the use of her breasts is one of my favorite in the book. Perhaps that’s because it exemplifies the way that she and Lightsong look at the word.

Anyway, as I’ve said before, I wasn’t intending this to be a book to parallel the state of the United States and the war in Iraq. It just kind of came out as it has, and I think the main reason is one of pulling a reversal upon myself. You see, \textit{Mistborn} was a book about a bloody revolution instigated by the protagonists. We don’t see a lot of the death it caused, and fortunately much of the bloodshed was averted by a timely speech given by a certain young nobleman, but the fact remains that I wrote about overthrowing a government.

That seems to be a popular topic for novels. And anytime I notice that something is popular in the genre, I start wondering if I could write a book about the opposite. In this case, I began thinking of a book where the protagonists were trying to \textit{stop} a war instead of start one. Where they wanted to stabilize the government instead of destabilize it. The opposite story of \textit{Mistborn}, in some ways.

I had the name of the book, \textit{Warbreaker}, long before I even knew who the Warbreaker would be or what the rest of the book would be about. I’m glad we were able to keep it, though my editor complained just a tad that he thought it didn’t indicate the right sense of epic fantasy for the book.

\subsection*{Siri Watches the Priests}

I took a bit of a risk here, having a little scene where Siri admits that all of the troubles and problems in Hallandren excite her. I hope this doesn’t seem out of character; I think I foreshadowed well that she’d react this way. Back in Idris, she was always making trouble, partially (even though she wouldn’t admit it) because she found it exciting. I think that’s common for those who end up in trouble a lot of the time.

Here, what she feels is that same sense—only a more mature version. She’s excited by politics, by being in the middle of things, by having a chance to change the future of the city. I think this is a valuable attribute for one in her position, as long as it isn’t taken too far. By having Vivenna constantly frustrated by her situation and Siri thrilled by hers, I wanted to show a contrast and have the reader come to the same conclusion Siri does in this scene: that Vivenna wouldn’t have made a very good queen to the God King. She’d have made the \textit{expected} queen, and would have done what everyone anticipated her doing. But she would have let herself be a martyr the entire time, which would have been a self-centered way of approaching her duty.

\subsection*{Siri Is Confronted by Blushweaver}

This is one of those little scenes you put into a book that isn’t foreshadowing anything specific. I don’t mind if people home in on this confrontation and worry that Blushweaver will take action against Siri, but I don’t go there with the book. Blushweaver here is just jealous. She knows enough to recognize that in herself, however, and won’t let it push her much farther than her little warning here.

I like what this shows about Blushweaver’s character, and I like that it illustrates how she sees Lightsong. Yes, she’s in love with him. Quite deeply, in fact. She brought him into her plots and schemes here partially because she trusts him, and partially because she wanted to show off for him—and perhaps finally convince him to accept her as a lover.



<p></p>
<p>\subsection*{What Bluefingers Knows}</p>
<p>Siri meets with Bluefingers, who surprises her in the bath yet again. In this little exchange, Bluefingers is being very careful, as he doesn’t want to let on how much he knows. As well as Siri is learning to deal with court, she has \textit{nothing} on Bluefingers, who has spent his entire life there—and who was trained by a Pahn Kahl steward before him. He has been planning his coup for a long time and was actually very frustrated when Vahr started his little rebellion—drawing eyes toward the Pahn Kahl. It was partially due to Bluefingers’s manipulations and information leaks to the Returned that Vahr was captured in the first place.</p>
<p>Here, he lets Siri think he doesn’t know that the God King is mute (he does know, and has known for most of his life) and that he is worried about the replacement of the Pahn Kahl servants. (That would be a setback, but not really the main problem.) What he wants most to do is drive a spike between Siri and the priests, and he’s succeeding gloriously. He almost leaped for joy when Siri offered her little “You get the God King and me out of the palace” offer. It makes his job a lot easier if/when he decides to assassinate the God King himself.</p>
<p></p>



