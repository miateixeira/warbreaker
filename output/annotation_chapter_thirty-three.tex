\section{Annotation Chapter Thirty-Three}

\subsection*{Vivenna and the Mercenaries Wait in the Safe House after the Lifeless Attack on the Slumlords}

Why does Jewels bother sewing up Clod? Why fix Lifeless at all? Denth’s answer is a fairly good one, but it could use some more explanation.

You see, when one makes a Lifeless, the reason the Breath stays and won’t come back is because the body of a recently deceased person is too “sticky” for Breaths. One Breath attaches to it, and because the body so clearly remembers being alive, it can use that Breath to power it. (Assuming you have the right Commands and can picture them correctly in your head when you make the Lifeless.)

However, the more the Lifeless is damaged, the less like the shape of a living person it is, and the more difficult it is for the Breath to keep that body going. Powering a body with only one Breath is hard—it requires the body to work mostly on its own. When you power a cloak or something like that, the Breaths need to provide a lot of energy, since there’s no real muscles to use or skeletal structure to rely on.

So the more wounded a Lifeless becomes, the less well its Breath can keep it going. Eventually you’ll need to stick a second Breath into it, then a third, all the way up until that Lifeless is nothing more than a bunch of bones you’ve Awakened. At that point, you might as well be using sticks or cloth.

\subsection*{Vivenna Admits the Real Reason She Came to Hallandren}

I’ve been pushing toward this for a long time in the narrative. Vivenna didn’t come to Hallandren to save her sister—that’s a front. That’s what she told herself. But the real reasons are more deep, more personal, and less noble. She \textit{had} to come because of how much of her life had been focused on the city. Beyond that, she came because of her hatred of Hallandren. She wanted to find ways to hurt it for what it had done to her.

It was partially her pride. \textit{She} was the one who was supposed to deal with Hallandren. Her pride wouldn’t let her stay away, wouldn’t let Siri do the job that Vivenna was certain she could do better.

She has kept her hatred in check quite well, but it’s always been there, driving her. I hope my readers always thought that coming to save Siri was a flimsy reason for Vivenna to come to T’Telir. The term \textit{love/hate relationship} has become a cliché, but I honestly think there is some real psychology to it, and I feel that I explained one aspect of it here, for Vivenna.

\subsection*{Vivenna Agrees to Learn Awakening}

This has been a long time coming. Sorry to make you wait; in some of my books, I like to use a lot of magic from the start. In others, I like to build out the setting first, letting the characters learn and explore more slowly.

I’ve long wanted to call a magic system “Awakening,” by the way. I tried it out originally in \textit{The Way of Kings} as the name for the transformation-based magic system in that book. But it never worked. You didn’t really “awaken” things. I was just using the term because it sounded good.

So I put it back in the file to be recycled someday. As I began to plan this book, I developed a magic system that \textit{could} be called Awakening. Bringing objects to life seems to fit that just perfectly. So yes, the word came first—though I’m not sure how much I grew the magic system around that one word, or if I was feeling I wanted to do a “bring objects to life” magic system and realized I could use that great name I’d come up with earlier.

\subsection*{Vasher Takes Vivenna Captive}

Now things are finally starting to move! My books, I know, can be kind of slow sometimes. That comes from the fact that I, myself, like to read books that are kind of slow. These two chapters were very important ones. Vivenna admitted something very important about herself, then in a way took the wrong sort of responsibility for her life. Siri realized something about herself, then took the right sort of responsibility for her life. A little bit of reversal going on, as the two sisters live their parallel—yet so different—lives in T’Telir.

But it was certainly time for a shake-up. The next Vivenna chapters turn a lot of things on their heads.



