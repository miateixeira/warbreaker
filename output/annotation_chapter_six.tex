\section{Annotation Chapter Six}

\subsection*{Siri Is Bathed, Then Sent to the God King}

This was a strange sequence of chapters to write. I’ve spoken before on writing characters of the opposite gender. This has grown easier and easier for me over the years, partially—I think—because I started out so bad at it that I insisted on forcing myself to practice and practice. Now, it’s usually as easy for me as writing men. In fact, I don’t even think about the gender of the character when I’m writing—I think about who the character \textit{is}. What their motivations and conflicts are. How they see the world and how they react to things. True, their gender does influence this—just as it influences their personalities. But I don’t sit down and say, “I’m going to write a woman now.” I sit down and say, “I’m going to write Siri.” I know who Siri is, so I can see through her eyes and show how she reacts.

All that said, I’d never before tried writing a wedding night from the viewpoint of a woman. It presented a few interesting challenges. For one, there’s a whole lot more nudity in this book than in my other books. I don’t shy away from this (even though I myself am probably more conservative than most of my readers in areas of sexuality), as I feel that what you do with your imagination is your own business. This scene could be done in a PG way, a PG-13 way, or an R way. It’s completely up to you how you want to imagine it.

One interesting thing to note is that my own wedding happened during the process of writing this book. I wrote this chapter before then, but I \textit{was} engaged at the time. While working on the novel I got to go through the entire progression of awkward moments of a wedding night myself. (Yes, it was our first time, by choice.)

I think that probably colored how I wrote Siri’s viewpoints throughout the entire book.

\subsection*{The Royal Locks}

A group of people whose hair changes color based on their emotions is another one of those little story seeds that had been bouncing around in my head for years before I wrote this book. I even did a few test chapters in other settings with characters who had this physical attribute. (\textit{Dark One}, which I don’t know if I’ll ever finish, toyed with it. As did a book set in the \textit{Aether} world.)

Eventually, this attribute slid into \textit{Warbreaker}. I’m glad I found a good home for it; I love how it adds a little bit of flavor to Siri and Vivenna, making them distinctive in a way that doesn’t have much of anything to do with the plot. I always talk about making things connected, and that’s very important. But you have to be careful not to make everything too neat. That leads to its own problems, as I mentioned in an earlier annotation.

The Royal Locks do work into the worldbuilding, as you’ll find out eventually in the book. However, mostly they’re around to give a distinctive feel to the world and the royal line, to show you that there \textit{is} something unique about the royals. It hopefully enhances your understanding of why Hallandren would work so hard to bring them back into their own line of kings.



