\section{Annotation Chapter Twenty-Two}

\subsection*{Lightsong Plays Terachin With Three Other Gods}

This is the newest scene in the book, added in the last revision before the novel went to copyedit. I added it for two reasons. My editor wanted to see another chapter between the previous Lightsong chapter and the next one. He felt that the god made up his mind to help Blushweaver too easily, and wanted to spend more time with Lightsong mulling over the decision.

I reacted quickly to the suggestion, as I’d been wanting to show Lightsong interacting with some of the other gods. It’s sometimes too easy for me to build my books around a small core cast and rarely involve any others, and I have to force myself to include more characters to round things out. This book had a distinct lack of scenes with “ordinary” gods. We got to see a lot of the exceptions, but never the run-of-the-mill divinities who make up the ranks.

I wanted to show how they schemed and how they acted. Putting Lightsong with three of them here helps the book quite a bit, I think. It makes the world feel more real and helps his character by providing contrast.

The game is something I developed in order to make this scene work. I wanted a divine game—one that wouldn’t require too much effort, would require a lot of preparation and extravagance, but would still qualify as a sport. So, we have a game where the gods can sit on a balcony attended by a fleet of servants and scribes tallying their throws.

When my editor read the scene, he loved it instantly. He called to tell me it was one of his favorites in the book, partially because of some particularly good Lightsong quips. He says that he fully expects some Sanderson book readers to develop the rules for the game someday, then play it at a con.



\subsection*{Vivenna Goes to Two Restaurants to Meet with Crime Lords}

Can you tell that I hate seafood? How does anyone eat that stuff? I mean, honestly. I’ve been forced to choke down raw clams before, and it was just about one of the most traumatic events in my life.

\subsection*{Only Potential Heirs of Idris Have Royal Locks}

This is true. It’s not a matter of genetics, but lineage. That’s a subtle distinction. Only the children of the person who ends up inheriting will have the Royal Locks. (Though there are a couple of notable exceptions to this, they won’t show up in this book, as it will take another novel to explain why and how the Royal Locks really work. If I ever write a sequel, that should be in it.)

This factoid about the Royal Locks should be one of several hints about the lineage of the Idrian crown. There is something odd about their heritage.



\subsection*{Lightsong Plays Terachin With Three Other Gods}

This is the newest scene in the book, added in the last revision before the novel went to copyedit. I added it for two reasons. My editor wanted to see another chapter between the previous Lightsong chapter and the next one. He felt that the god made up his mind to help Blushweaver too easily, and wanted to spend more time with Lightsong mulling over the decision.

I reacted quickly to the suggestion, as I’d been wanting to show Lightsong interacting with some of the other gods. It’s sometimes too easy for me to build my books around a small core cast and rarely involve any others, and I have to force myself to include more characters to round things out. This book had a distinct lack of scenes with “ordinary” gods. We got to see a lot of the exceptions, but never the run-of-the-mill divinities who make up the ranks.

I wanted to show how they schemed and how they acted. Putting Lightsong with three of them here helps the book quite a bit, I think. It makes the world feel more real and helps his character by providing contrast.

The game is something I developed in order to make this scene work. I wanted a divine game—one that wouldn’t require too much effort, would require a lot of preparation and extravagance, but would still qualify as a sport. So, we have a game where the gods can sit on a balcony attended by a fleet of servants and scribes tallying their throws.

When my editor read the scene, he loved it instantly. He called to tell me it was one of his favorites in the book, partially because of some particularly good Lightsong quips. He says that he fully expects some Sanderson book readers to develop the rules for the game someday, then play it at a con.



\subsection*{Vivenna Goes to Two Restaurants to Meet with Crime Lords}

Can you tell that I hate seafood? How does anyone eat that stuff? I mean, honestly. I’ve been forced to choke down raw clams before, and it was just about one of the most traumatic events in my life.

\subsection*{Only Potential Heirs of Idris Have Royal Locks}

This is true. It’s not a matter of genetics, but lineage. That’s a subtle distinction. Only the children of the person who ends up inheriting will have the Royal Locks. (Though there are a couple of notable exceptions to this, they won’t show up in this book, as it will take another novel to explain why and how the Royal Locks really work. If I ever write a sequel, that should be in it.)

This factoid about the Royal Locks should be one of several hints about the lineage of the Idrian crown. There is something odd about their heritage.



