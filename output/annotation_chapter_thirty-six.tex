\section{Annotation Chapter Thirty-Six}

\subsection*{Siri Grows Her Hair for Susebron and Talks about Seduction}

I think these two chapters best show off the tone reversals I was trying for in this book—and explain partially why I was all right with those early chapters being so different from the prologue. Following Vivenna’s biggest chapter for shocks, surprises, and failings, we come here—to what is one of the most flirtatious and calm of the Siri chapters.

You should have been able to notice some changes about Siri, one of the most subtle being her ability to control her hair. The hair is, in a way, an extension of the metaphor. In the beginning chapters, Siri wasn’t able to control it at all, and it always changed back right after she tried to make it go to a specific color. It did what it wanted, reflecting her attitudes, and kind of represented her ability (or lack of ability, in her case) to control the world around her.

Now, she’s able to manipulate things around her slightly to her liking. In contrast, Vivenna’s life is completely out of control. And her hair will respond.



<p></p>
<p>\subsection*{Susebron’s Priests}</p>
<p>Susebron is right to trust his priests. At least, he’s somewhat right. They aren’t evil men, and they do want what is best for him—as long as that doesn’t include going against their traditions and rules. They believe they have the charge to protect Peacegiver’s Treasure, and the God King holds that treasure. They do feel bad for what they are required to do to him.</p>
<p>Their interpretation is extreme, but what would you do, if your god (Peacegiver) commanded you that the Breaths be held and protected, but never used? Cutting out a man’s tongue to keep him from using that terrible power is the way they decided to deal with it. Harsh, but effective.</p>
<p>Either way, they aren’t planning to kill him. One of the big reversals I planned for this book from the concept stage was a world where the priests were good and the thieving crew was evil—a complete turnabout from \textit{Mistborn}. Denth and his team were developed in my mind as an “anti-Kelsier’s Crew.” The priesthood, then, was to turn out to be maligned by the characters and actually working for their best interests.</p>
<p>In the end, I went with the evil crew idea, but the priests aren’t 100% without their flaws.</p>
<p></p>

\subsection*{Siri and Susebron Eat a Midnight Meal}

This is a scene lifted almost from my own life. While on my honeymoon, Emily and I thought we were being so indulgent by ordering room service at three a.m. It was on a cruise ship, and you can do that kind of thing without having to pay extra for it. It kind of felt like the entire ship’s kitchens were there for our whims. And so, a variation on the event popped up in this book.

That doesn’t happen to me very often in books. Usually, it’s hard to point toward one event in my life that inspired a scene. But those sorts of things are peppered throughout this book. Another one is the scene where Siri tries to look seductively at Susebron, then bursts into laughter. My wife is absolutely \textit{terrible} at looking seductive—not because she isn’t pretty, but because whenever she tries, she ends up having a fit of laughter at how ridiculous she thinks she looks.

What else . . . oh, Susebron’s taste buds. A couple of people have e-mailed me about this. From my research (which could be wrong), I’ve come to understand that the old teaching that certain parts of your mouth have taste buds that focus on certain tastes is wrong. The conventional wisdom is that your “sweet” taste buds are on your tongue, and if it is removed, you won’t be able to taste sugar. (Which is why people e-mail me.)

That’s apparently an urban legend. There \textit{are} different kinds of taste buds, but each kind appears in clusters alongside the other kinds. And while most of your taste buds are on the tongue, many are on the roof of the mouth too. So Susebron could taste sweets as well as he tastes anything else.



