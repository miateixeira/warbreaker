\section{Annotation Chapter Ten}

\subsection*{Vivenna Meets the Mercenaries in the Restaurant}

Denth was planned as an important figure in this book from the early going. I was looking for a type of character I’d never written, someone who could be interesting, but not steal the show too much from Vivenna. But I also wanted someone who would provide some good verbal sparring (a theme of this book) without simply replicating the way that Lightsong makes word plays.

Denth’s and Tonk Fah’s personalities grew out of this. I wanted them to offer a more lowbrow sort of humor, conversations that dealt with more base types of joking. They aren’t supposed to be laugh-out-loud funny, but hopefully they’re amusing and colorful as characters.

\subsection*{Vivenna Visits Lemex}

In the very early planning of this book, I intended Lemex to live. He was going to become a mentor figure for Vivenna, and have the very personality that she described him as having in her imagination. Spry, quick-witted, intelligent.

So I decided to kill him off.

Why? Well, it’s complicated. On one hand, I felt that he was too much of a standard character from one of my books. The witty mentor is not only a stereotype of fantasy, but something I rely upon a lot in my writing. (Though, granted, many of those haven’t been published—however, Grandpa Smedry from the Alcatraz books is a great example of this kind of character.)

I also felt that Lemex could too easily be a crutch for Vivenna in the same way that Mab could have been for Siri. The idea was to keep these sisters consistently out of their elements, to force them to stretch and grow.

Instead, I upped the competence of the mercenaries and decided to have them play a bigger part.



<p></p>
<p>\subsection*{Denth the Traitor}</p>
<p>Denth was always going to betray Vivenna. In fact, this is one of the very early concepts for the book—the idea that I wanted a bad guy who was not only likable, but funny. Too often, villains are portrayed as simply despicable people. If they laugh, it’s evil laughter.</p>
<p>But people just aren’t like that, not most of them. They’re real, they have goals and motivations, but they also laugh, cry, and feel. Denth is a mercenary. More than that, he’s a man who has caused a lot of pain and death in his long lifetime, and he copes with it by letting himself be hired to do important tasks. So that he doesn’t have to feel as responsible.</p>
<p>In a lot of ways, I imagined Denth as the anti-Kelsier. Glib, smart, and hired to do impossible tasks. Only in this book he works for the wrong team. In this scene in particular, he was doing his best to nudge Vivenna to give him the Breaths. His job was only to hold her, to keep her captive and in reserve just in case the plots with Siri failed. That way, there would be a second princess to use in the plots. He was assigned to work for Lemex originally just to give him an in with the Idrians in the city, so that he could rile them up to incite the war further. But when he found that Vivenna was coming, he realized that she would be a much better pawn, and so he poisoned Lemex and took her instead. His employers were very happy to have a backup princess.</p>
<p>So, anyway, Lemex’s Breaths were secondary. Denth wanted them, but he knew that the most important thing to do here was get Vivenna to trust him. So he tried to subtly manipulate her into giving them to him. (He intentionally acted reluctant to take them in order to goad her.)</p>
<p>In some ways, even though he doesn’t have a viewpoint, a big theme of this book is the tragedy of the man Denth. He could have been more. At one time, he was a much better man than most who have lived.</p>
<p>Tonk Fah is a waste of flesh, though. Even if he is funny sometimes.</p>
<p></p>



