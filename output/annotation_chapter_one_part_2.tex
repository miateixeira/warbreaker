\section{Annotation Chapter One Part 2}

\subsection*{Ramblemen}

Ramblemen are more than simple traveling jugglers or storytellers. They’re merchants who specialize in bringing news (for a price) and stories as well as goods and services.

Readers latched onto this word, and I’ve had a lot of people say, “I love that term! Why don’t we get to see a rambleman in the book?”

Because some things in books are just there to hint at the greater world. Sometimes a keen, cool word like that can evoke so much more when used in passing than it would if developed into a side plot or attached to a character.

\subsection*{Idris’s Drabness}

One thing to realize is that the Idrians’ attempts to make their city colorless are more superstition than they are effective. It’s much harder to get colors away from an Awakener than the Idrians think. For instance, black is one of the most powerful colors to use for fueling Awakening—but the Idrians don’t even consider it a color. Their browns and tans would also work for Awakening.

However, a lot of times, the traditions of a culture don’t have much to do with factual reality. The determination to avoid colors grew out of a desire to contrast with Hallandren and their devilish Awakeners. It got taken to the extreme, however, and as the centuries passed, the Idrians grew confused about just what Awakening is and what it can do. Of course, there are some who know—Hallandren isn’t \textit{that} far away. But there’s also a lot of rumor and misinformation.

\subsection*{Mab the Cook}

If it sounds to you like Mab knows a lot about Awakening and Hallandren, then you’ve picked up on something. Mab actually used to live in T’Telir. (She was born in Idris, but ran away during her teens.) During her twenties, she was a courtesan of some repute in the city. She had some fairly high-profile clients—so she was more than just a poor, street-corner prostitute. She fell in love with one of the men, however, and he convinced her to give him her Breath. Then he left her.

As a Drab, she had much more trouble finding work. She’d lost a bit of her sparkle, and whatever she’d used to capture the hearts of men, she’d lost that too. She ended up as a madam, running a much poorer whorehouse, using her old contacts and reputation to get clients.

As soon as she made enough, she bought another Breath and returned to Idris, where she got a job in the king’s kitchens. To this day, she bears a lot of ill will toward the Hallandren upper crust, and Awakeners in particular.

\subsection*{The King and Yarda Discuss Sending Vivenna}

I go back and forth on this scene. Sometimes I think it’s too long. Other times I worry that it’s not long enough.

Through the history of the book, this particular scene inched longer and longer as I tried very hard to explain why a good man like Dedelin would send Siri to die in Hallandren. (And also, I wanted to be sure to explain why he was sure she would die there.) There’s a whole lot of setup going on in this sequence between the king and his general.

And I worry that there should be more. While what they do makes intrinsic sense to me, a lot of readers have been confused about the tactics here. Why is the king doing what he’s doing? Is it really needed? Isn’t there another way? This section is the only answer we get to a lot of those questions, since it’s the one and only scene in the book from Dedelin’s viewpoint.

That said, I think this scene might also be too long. The more space I dedicate to Dedelin, the more readers are going to think that he might be a main character. Some are surprised to read on and find out that the king doesn’t make another appearance in the novel. (Well, okay, he makes one more—but he doesn’t have a viewpoint.) I don’t want to put too much here or have readers focus too much on the tactics of his decision, since really all that matters is that readers understand that Siri has been sent unexpectedly to marry the God King.

I’m still iffy on the scene. Some test readers wanted to see the scene where Dedelin says farewell to Siri. (We skip it; the next scene begins with Siri riding away.) They feel they missed a chapter. But I eventually decided that I needed to keep this beginning flowing quickly, because the longer we spend in Idris, the longer it will take us to get to the real plots in Hallandren. If it weren’t so important to set up Siri and Vivenna ahead of time (so that their reversal has impact), I would have just started the book with Siri arriving in Hallandren.



