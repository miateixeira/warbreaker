\section{Annotation Prologue}

\subsection*{The Origins of the Prologue}

This began as a first chapter; I only later turned it into the prologue. My worry when I made the change (and it’s still a bit of a worry) was that it was kind of a sneaky way to begin the book. Let me explain.

This novel focuses primarily on Siri, Vivenna, and Lightsong. Vasher, as the fourth viewpoint, is only in there fairly sparsely. True, he drives a lot of what is happening from behind the scenes, but he’s a mysterious figure, and we don’t know a lot about him. This prologue is pretty much the most extensive, lengthy, and in-depth scene we get of him.

Therefore, it’s kind of sneaky to begin the book with him. I did it for a couple of reasons. First off—and this is the most important one—this scene is just a great hook. It shows off the magic system and the setting of the novel (most of the action takes place in T’Telir, even though the first few chapters are over in Idris). It’s full of conflict and tension, with a mysterious character doing interesting things. In short, it’s exactly how you want to begin a book.

My worries aren’t about this prologue so much as they are about the following three chapters, where things slow down a lot. I was tempted to cut this scene and put it in later, but I eventually decided that giving it the mantle of a prologue was enough. A lot of times, particularly in fantasy, we writers use a prologue to highlight a character or conflict that might not show up again for a while.

\subsection*{Naming Vasher}

Vasher’s name has interesting origins. I first began toying with the ideas that became \textit{Warbreaker} back in 2005. I was hanging out with my then girlfriend (not Emily, but Heather, the girl I dated before I met Emily). We were up at Heather’s family’s cabin in Island Park, Idaho, and I had just met her father for the first time. His name was Vance.

The name intrigued me. Yes, I’d heard it before, but for some reason at that moment it struck me. Later that day, sitting on the dock of the lake, I pulled out my notebook and began to play around with ideas for a story. I tweaked the name to Vancer, but that just didn’t sound right, though I used it for a while. The next incarnation was Vasher. [\textit{Editor’s note: Brandon had earlier used the name Vasher in 2003 for a different character in the draft of another novel, but he had completely forgotten that by the time he wrote this annotation.}]

I began doing some preliminary prose writing, plugging in a magic system I’d been working on. (I’ll talk more later about how I came up with Awakening.) It became a story about a guy who was thrown into prison, then used his Awakening magic to get out of it. (Along with the help of his longtime sidekick, whose name escapes me right now.)

It wasn’t very long. I’ll have to dig it out sometime—it’s only handwritten and wasn’t something I ever intended to publish. Just a quick character sketch. It did have the first line, however, of what eventually became this book: “Why does it always have to end up with me getting thrown into prison?”

\subsection*{First Line Origins}

Of course, this line got a tweak of its own in later drafts. I was fond of this first line, as I’d used it in the original short story with Vancer. However, in that story, he’d been thrown into prison for other reasons. In \textit{Warbreaker}, I began the book with Vasher getting himself purposefully tossed into prison.

So, in the end, my editor pointed out that the line no longer worked quite right. We had to change it—why would Vasher complain about getting thrown into prison if he had done it to himself on purpose? So, it became “It’s funny how many things begin with my getting thrown into prison.”



