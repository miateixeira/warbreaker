\section{Annotation Chapter Forty-Nine}

\subsection*{Vasher’s Temper}

Giving Vasher temper issues is part of a minor quest on my part to find more realistic conflicts and personality traits for my characters. It seems that much of the time, the flaws that writers give their heroes are really just backhanded talents. A hero is “too bold” or “too much of a bookworm.” I’m guilty of this as much as anybody. (Siri’s character flaws are an example.)

It’s a tough balance. Real people tend to have flaws that make them . . . well, unlikable sometimes. Or at least difficult to get along with. We get grumpy, we make bad decisions, we say things we don’t mean. It’s hard to convey this in a story without making the characters unlikable. There are authors who are fantastic at doing so, and Vasher here was me toying with making a person have a more realistic temperament. There’s no hidden advantage for him being like he is; he’s simply got anger issues. Not extreme ones—it’s not like he has to go to therapy. He’s just prone to losing his temper like any number of people out there in the world.

\subsection*{The Attack on the Salt Merchant Was a Cover}

Denth did dozens of things like this, subtle methods of bringing the two kingdoms closer to war. This is the only one that Vivenna and Vasher figure out soon enough to be able to counter.

In this scene, Vivenna’s chapter arc is her struggle with deciding when to make judgments and when not to. It seems that in our society, it’s taboo to judge someone. If you judge, you’re seen as intolerant. And most of us hate being labeled that way. I remember seeing an advertisement online just a few days ago that said something like “Please teach your children tolerance; teach them not to judge others.”

Now, those who read my blog know that I’m big on trying to understand other people’s viewpoints. I don’t like how harsh our dialogues about charged issues tend to be. I’ve said I’m a peacemaker by temperament. However, I think telling someone, “Don’t judge others” is just plain ridiculous. (Of course, maybe it’s all just semantics.)

We have to judge. We do it every day. We decide who we want to be friends with. We judge which candidates we want to vote for. We judge which activities we want to be a part of. A lot of these judgments are influenced by our thoughts on the people involved in them.

It’s not good to be racist. Skin color is a terrible reason to judge someone. But that doesn’t mean that you shouldn’t sometimes make judgments about people for other reasons. I think maybe we’ve become hypersensitive to this sort of thing.

\subsection*{Vasher Uses Straw Figures to Find the Tunnel}

I wanted to bring the straw men back into the book, as I felt I needed to show you—and Vivenna—just how capable Vasher is with Breath. He’s leaps and bounds above most people. I think this book gives a skewed perspective, since we don’t see any ordinary Awakeners. We see those just learning (Vivenna) and we see one of the greatest masters of the art to ever live (Vasher).

With his practice and years of Awakening, he’s able to get Awakened objects to do things that others wouldn’t be able to. The straw men are a good example. As for why he apologizes, well, he doesn’t even know that himself. I think it’s because he realizes that Breath \textit{can} make something sentient and aware, like Nightblood, and worries that the straw creatures become (even just slightly) more than just mindless automatons.

\subsection*{Vasher Kills}

As I said, he has a temper. He tends to lose it when he fights. He’s not a berserker or anything like that; he simply lets his passion get in the way when he’s in battle. It makes him worse at fighting, particularly when dueling. It also makes him a lot more dangerous sometimes.

Vivenna looking back at him, his hand on Nightblood’s hilt, slowly pulling it forth as the bodies lie around on the ground is one of my favorite scenes in the book.



<p></p>
<p>This is how Vasher lost his Breath before, by the way. If you recall, he began the book with barely enough to Awaken in weak ways. He remembers having much more Breath. Beyond feeding on one Breath a week, slowly eating away his supply, he drew Nightblood a few months back. That drained away his Breath and left him with only a few remaining. As for who he killed that time . . . I’m going to hold off on saying, just in case I decide to incorporate it into a future book.</p>
<p>Notice how he grows in size here when he isn’t paying attention. That’s his Returned nature beginning to manifest, much like Vivenna’s hair reacts to her emotions, because of the moment of great passion from him during the fight.</p>
<p>In this chapter, we also get the first hints that children and animals like Vasher. That’s another hint about his nature—though a very, very subtle one, since I haven’t talked about how animals and children all like Returned. They can sense the divine Breath within him, and it comforts them.</p>
<p></p>

\subsection*{Priests as Scapegoats}

I do think that someone being a different religion from yourself makes them a good scapegoat. We tend to be put off by anyone who is too devout toward religion, even if their passion for it mimics our own passion for something we are dedicated to. It’s easy to divide ourselves along religious lines.

Once again, I think I need to mention that I didn’t write this comment (or the ones about not judging) into the book as an intentional message. It just seemed appropriate for the characters to say or consider, and I happen to agree with them. What I think is important influences the book. How can it help but do otherwise?



