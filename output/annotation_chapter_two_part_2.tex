\section{Annotation Chapter Two Part 2}

\subsection*{Vivenna and Her Father Chat about Siri}

King Dedelin does love Siri. He’s a good man, but not quite as great a man as his daughters probably think. He \textit{does} put Vivenna first. He loves her more. Perhaps because he relates to her; the two of them are very similar in many ways.

The girls’ mother passed away over a decade ago, by the way, in a riding accident. Siri doesn’t remember it, but Vivenna does, and that is one of the sources of tension between them. Siri’s flagrant rides remind both Vivenna and her father of the way the queen died. Memory of their mother is also part of what makes Vivenna more controlled and “perfect.” Siri grew up with very little supervision, while Vivenna had much more of it in the person of her mother.

Anyway, Dedelin loves Vivenna more. When he says that he sent Siri, “Not because of personal preference, whatever people say,” he’s being truthful as he sees things—but he’s deluding himself. He’s convinced himself that he did it primarily because Vivenna’s leadership is important to Idris and she can’t be risked. That is important to him, true, but his love for Vivenna is the primary reason.

Now, he’s not callous or hateful of Siri. He loves her. But . . . well, Vivenna reminds him of his wife. I guess you can’t blame the guy too much for what he did.

\subsection*{Vivenna Picks Berries}

One aspect of the worldbuilding I barely get to talk about is the Idrian monks. I really liked the concept of a group of monks whose duties weren’t very religious. Rather than sitting in a monastery all day, their duties are essentially to act as servants to the kingdom’s poor. (Not to say that monks in our world don’t do that. However, I liked the concept of it being much more formalized.)

In Idris, if a man breaks his leg and can’t work the field, a monk will come and take his place on the job. The wages for that work still go to the family of the man who has been hurt. Sometimes, if a father dies and cannot support his family, a monk is assigned permanently to take his place at work duties and provide for that man’s family.

They go wherever they are needed, forbidden to own or possess anything themselves, giving all they have to the people. Now, of course, not everyone who becomes a monk fits the ideal. Without the pressures of needing to feed one’s self or acquire goods, some of them can be kind of lazy. But many are very diligent, like Fafen.

\subsection*{One Last Note on Reversals}

We have a nice moment in this chapter rotating around a single word. Siri begins the chapter thinking about how she was supposed to be useless, and how she wishes that she still were. Then Vivenna ends her section thinking about how she’s \textit{become} useless. That terrifies her.



