\section{Introduction}

Welcome! My name is Brandon Sanderson. Before anything else, I’d like to thank you for your interest in my books. I hope you enjoy~\textit{Warbreaker}.

In case you don’t know, I’m a professional fantasy novelist. My first book,~\textit{Elantris}, was published in some thirteen languages, earned me a Campbell nomination, and got starred reviews in Publisher’s Weekly and the Library Journal. It was also picked by Barnes and Noble editors as the best fantasy or science fiction book of the year.

My second book,~\textit{Mistborn}, is out in paperback.~\textit{Mistborn 2: The Well of Ascension}~was published in August of 2007. Book three is out in October of this year. I also have a kid’s book~\textit{Alcatraz versus the Evil Librarians}~out from Scholastic Press. You can find sample chapters of these books at the end of this file. If you like Warbreaker, consider buying those!

As many of you might already know, I was chosen in December of 2007 to complete Robert Jordan’s epic masterpiece The Wheel of Time. I’m hard at work on the twelfth and final novel in this series, titled A MEMORY OF LIGHT. It should be out sometime in the fall of 2009. Coincidentally, that should be the same year~\textit{Warbreaker}~is released.

How this Book Came About

\textit{Warbreaker}~is something of an experiment for me. For a long time, I’ve wanted to release an e-book on my website. My first inclination was to grab one of my old, unpublished books and offer it.

And yet, one of my main reasons for releasing said e-book would be for publicity reasons. I wanted something I could give away for free which would show what I’m capable of writing and therefore (hopefully) encourage people to look into my other books. I figure that if people give my books a try, they’ll be hooked and read the other ones.

That made me want to offer something new. Something that showed off the very best of my abilities. Why offer an inferior product as your free sample? If it wasn’t good enough to get published on its own, then wouldn’t that lead people to think of my books as inferior?

That leads us to~\textit{Warbreaker}. This is not an old work. In fact, this is my newest work. It has been purchased by Tor (who gave me permission to try this experiment) and will be published in hardcover in 2009, with a paperback release to follow the next year.

I like to have a lot of contact with my fans, and as I contemplated releasing a new book (rather than an old one) on my website, I had a chance to do something rarely seen. I could release drafts of the book~\textit{as I wrote them}, allowing my readers to catch a glimpse of the writing process. They could see the evolution of the book, maybe even offer feedback on early drafts, allowing them to have a much closer connection to me as a writer and this book in specific. By doing this, I could make~\textit{Warbreaker}a project which would engage my already existing readers as well as people who’d never tried my books before.

I decided to go ahead and give this a try. That was back in June of 2006.

My Worries

Releasing the book this way is a gamble for two reasons.

First off, there’s the perennial fear that I think all artist get when they give away their art for free. A part of me worries that by giving this book away, it will end up selling dreadfully when it’s actually released. Poor sales like that on one book (even a 10% drop) could set a bad tempo for future books.

I don’t think this is likely. I, personally, feel very differently about art and the public than certain record executives appear to feel. I think that people WILL pay money for something they’ve already read if they liked it enough. They can always get books for free via the library anyway. Besides, I’m not trying to recruit people to buy one book; I’m trying to recruit lifelong fans who will still be reading Brandon Sanderson novels twenty years from now.

On top of all that, I believe that releasing at least one novel for free will bring my work to many readers who wouldn’t otherwise be familiar with my work. The potential gains far outweigh the potential losses.

Still, I worry a little bit. But artists tend to do that.

The second fear I have relates not to releasing a work on-line, but releasing an~\textit{Unfinished}~work on line. Though this is the fourth draft of the book, my novels usually see somewhere near eight drafts before they go to press. This is still a work in progress. What if readers pick this up, read through it, and judge me flawed as a writer because their only experience with me comes from an unpolished work?

This one really bothers my agent. He’s got a good point. Still, I think the opportunity that this affords my readers—particularly the aspiring writers among them—was too great to ignore. It is done, and I intend to stick to my original plan. I will post every draft as I complete them, then will eventually post comparisons of the drafts so that readers can follow the changes made to the book.

Know, however, that this is~\textit{still a work in progress}. Don’t judge me too harshly based on its flaws.

Conclusion\textit{}

My hope is still to let readers collaborate a little bit on this book. Feel free to visit my forums and email me with your impressions of the novel. Your feelings and questions are important to me and can help this book grow better.

I hope that you enjoy this book. If you do, the best thing you can do to say thanks would be to purchase a copy when it is released! (It’s looking like Spring 2009.) You are also welcome to share it with friends (see the rights explanation below.)

Remember my published novels as well. They are far more polished, and if you want to make certain that I write more fantasy novels in the future, the best thing to do is indicate your will to Tor by purchasing my novels. You can find sample chapters of each of my published works at the end of this document. If you’re reading in Microsoft Word, you can use the Document Map to jump to them. (Or to jump between chapters of WARBREAKER.) Using this, you can also find a list of revisions (spoiler warning!) that were made in this draft.

Most of all, I want to thank you for reading. I think the primary motivation of all artists is the desire to express themselves. My books are not complete until you read them and add your imagination to the events they contain. For me, sales are secondary to that.

Enjoy.

Brandon Sanderson, January 2008

