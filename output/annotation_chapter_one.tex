\section{Annotation Chapter One}

\subsection*{Tone}

You can probably guess why I was worried about the transition from the prologue with Vasher to this chapter with Siri. The tone shift is quite dramatic. Actually, one of the things my agent complained a lot about with this book \textit{was} the tone. Not just for this chapter shift, but for the entire book.

In his opinion, there were too many different tone shifts going on. We have Vasher’s plot, which is dark and sometimes violent. We have the Siri plotline, which is romantic and sometimes whimsical. We have Lightsong, whose chapters are glib and smell faintly of an old comedic murder mystery. Then we have Vivenna, whose tone bounces around across \textit{all} of these.

That’s one of the things I like about the book. My agent complained, but I know he likes things more streamlined than I sometimes do. He loved the \textit{Mistborn} books, and I do think they are excellent novels—but they are very focused. The characters are distinctive, but their plots are all centered on many of the same types of goals.

With \textit{Warbreaker}, one of the main things I’m trying to do is contrast it to \textit{Mistborn}. To do something different, something that harkens a little more back to \textit{Elantris}, with its three very different viewpoints.

I want there to be a lot of different tones and feels to this book. It’s part of the theme of the novel—that of vibrant Hallandren and its many wonders. I want it to feel like a lot is going on, and that in different parts of the city, very different stories can be told.

\subsection*{The Origins of Siri and Vivenna}

Back around the year 2000 or 2001 I started writing a book called \textit{Mythwalker}. It was an epic fantasy novel, an attempt to go back to basics in the genre. I’d tried several genre-busting epics (one of which was \textit{Elantris}) that focused on heroes who weren’t quite the standards of the genre. I avoided peasant boys, questing knights, or mysterious wizards. Instead I wrote books about a man thrown into a leper colony, or an evil missionary, or things like that.

I didn’t sell any of those books. (At least, not at first.) I was feeling discouraged, so I decided to write a book about a more standard fantasy character. A peasant boy who couldn’t do anything right, and who got caught up in something larger than himself and inherited an extremely powerful magic.

It was boring.

I just couldn’t write it. I ended up stopping about halfway through—it’s the only book of mine that I never finished writing. It sits on my hard drive, not even spellchecked, I think, half finished like a skyscraper whose builder ran out of funds.

One of the great things about \textit{Mythwalker}, however, was one of the subplots—about a pair of cousins named Siri and Vivenna. They switched places because of a mix-up, and the wrong one ended up marrying the emperor.

My alpha readers really connected with this storyline. After I abandoned the project, I thought about what was successful about that aspect of the novel. In the end, I decided it was just the characters. They \textit{worked}. This is odd because, in a way, they were archetypes themselves.

The story of the two princesses, along with the peasant/royalty swap, is an age-old fairy tale archetype. This is where I’d drawn the inspiration from for these two cousins. One wasn’t trained in the way of the nobility; she was a distant cousin and poor by comparison. The other was heir to her house and very important. I guess the idea of forcing them to switch places struck some very distinct chords in my readers.

Eventually, I decided that I wanted to tell their story, and they became the focus of a budding book in my mind. I made them sisters and got rid of the “accidental switch” plotline. (Originally, one had been sent by mistake, but they looked enough alike that nobody noticed. Siri kept quiet about it for reasons I can’t quite remember.) I took a few steps away from the fairy tale origins, but tried to preserve the aspects of their characters and identities that had worked so well with readers.

I’m not sure why using one archetype worked and the other didn’t. Maybe it was because the peasant boy story is \textit{so} overtold in fantasy, and I just didn’t feel I could bring anything new to it. (At least not in that novel.) The two princesses concept isn’t used nearly as often. Or maybe it was just that with Siri and Vivenna I did what you’re supposed to—no matter what your inspiration, if you make the characters live and breathe, they will come alive on the page for the reader. Harry Potter is a very basic fantasy archetype—even a cliché—but those books are wonderful.

You have to do new things. I think that fantasy \textit{needs} a lot more originality. However, not every aspect of the story needs to be completely new. Blend the familiar and the strange—the new and the archetypal. Sometimes it’s best to rely on the work that has come before. Sometimes you need to cast it aside.

I guess one of the big tricks to becoming a published author is learning when to do which.



