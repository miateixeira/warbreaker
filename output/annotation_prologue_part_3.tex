\section{Annotation Prologue Part 3}

What else to say about the prologue? I’ll talk about Nightblood in a future annotation. Let’s see . . .

\subsection*{Vasher Awakens the Cloak}

He doesn’t end up using it. A lot of people point this out. Him not needing it was intentional. I know it raises a question in the prologue, and seems kind of useless, but it’s there to give some added depth to the scene and the magic. Plus, it was just a smart thing to do. Awakening the cloak to protect him was a precaution—one that didn’t end up being needed, but one of the things that annoys me about books is when every single thing the heroes do ends up being important, useful, or even a hindrance. Sometimes you pack yourself a lunch, but then just don’t end up needing it.

\subsection*{The Straw Figure Returns with the Keys}

Vasher couldn’t have used a thread to unlock the door here, by the way. I know a certain person manages to pull it off later in the book, but that doesn’t happen in the God King’s dungeons.

One thing to remember about designing magic systems—particularly those as important to their societies as mine—is that the people in the world \textit{live} with this magic. They use it and see it being used regularly. They think of it and consider it.

It’s not hard to design a lock that an Awakened thread can’t unlock easily. It \textit{is} more expensive to buy a lock like that, and so not all locks have such precautions. These ones do, however.

If you’ve read the book through, then you know that Vasher’s simple-sounding Command of “Fetch Keys” given to the straw man is incredibly complex. In fact, it’s probably one of the most complicated Commands given to any Awakened object in the entire book. It’s kind of cool to me that Vasher uses it here, showing off incredible mastery of the magic, before anyone reading will even realize how much skill saying those two words correctly really takes.

\subsection*{Vasher Confronts Vahr}

Vahr’s original name was Pahn. You can find it used in earlier drafts of the book. I liked the sound and look of that so much, in fact, that I based the name of the people he came from on his own name.

That made for a problem, though. That’s like having a person named America. It happens, but it’s kind of confusing in a book. So, I eventually had to change his name to something that had a similar look and feel, but which wouldn’t lead to so much confusion.

Vahr dies here, and one of the major revisions I made to the book was to bring out more of his influence throughout the book. I didn’t want it to be \textit{too} in your face. However, he was a very important man. We see only the very tail end of his life here, but he worked for over a decade as a Pahn revolutionary, trying to inspire his people to rebel against Hallandren oppression. (Or at least what he saw as Hallandren oppression.) He eventually became such a popular figure that he raised an army, with monetary support from several of Hallandren’s trade competitors across the sea.

We see here the end of that—Vahr, captured and being tortured. He’s a lot more important than he seems, both to the world and to the novel itself.



<p></p>
<p>Also, if you look, I’ve inputted in the last drafts a little hint here of Vasher being a Returned. He says he could have the Fifth Heightening if he wanted it, which is true. He has his Returned Breath suppressed, but if he let it out, he could instantly have the Fifth Heightening. However, he’d be instantly recognizable as Returned the moment he did that. Plus, he couldn’t use that Returned Breath for Awakening things.</p>
<p></p>



