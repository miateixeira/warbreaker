\section{Annotation Chapter Fifty-One}

\subsection*{Vasher Considers Killing Lightsong}

I remember reading a book a few years back where the heroes are separated from one another. One group of them is doing something clandestine, while another group is traveling in the area posing as ordinary peasants. Neither knows what the other is up to.

Well, some soldiers capture the ones posing as peasants, then go and talk to the main group of heroes. The main group says, “Well, I guess we’ll have to kill those poor peasants who inadvertently passed by and discovered we have an army here.” It’s supposed to be dramatic irony, I believe. The protagonists nearly end up killing one another through a cruel twist of fate. (The group posing as peasants avoid death, however, for reasons I can’t quite remember.)

Anyway, I put the book down shortly after. I didn’t remember the scene I’d read until writing this particular one. Why wouldn’t Vasher just kill Lightsong, thereby ending the war?

Because that’s not a good solution. It’s a shortsighted one. If you do terrible things in the name of trying to do what is right, I think you’ll just end up creating bigger problems. Vasher couldn’t have killed Lightsong, not and remain the man he wants to be. He knows this, I think. Even a man with the reputation of Lightsong is not someone you can kill just because he’s inconvenient to you. Not if you want your conscience to go untarnished.

And if innocent peasants happen to spy your good-guy army, there are \textit{much better} actions to take than deciding to execute them in the name of the greater good. You do that, and you stop being heroes. (That’s not necessarily a book killer. It’s only one if you expect me to keep on reading and still consider your characters heroic.)

\subsection*{Nightblood and Vasher Interact as He Sneaks into the God King’s Palace}

Note that Nightblood is capable of more change than Vasher assumes. Vasher has a bit of a blind spot when it comes to Nightblood. He makes assumptions he wouldn’t make regarding other people or elements of Awakening. It’s hard for him to regard the sword without bias. If you want to know more about this, read the sequel. (Er, if I ever write it.) Which is tentatively named \textit{Nightblood}.

Anyway, Nightblood is named for the smoke he leaks, and he originally had a different name when he was created. Vasher himself dubbed the sword Nightblood after he had used it to kill the woman he loved. The blackness that leaks out is actually corrupted and consumed Breaths, the ones that Nightblood leeches off anyone who draws him.

In this scene, there are some small hints of what it was like during the Manywar—with people Awakening ropes to toss boulders and things like that. It was a pretty dramatic conflict, the first one where Lifeless and Awakeners were put to a great deal of use in battle.



<p></p>
<p>\subsection*{Denth Captures Vasher}</p>
<p>So, between this chapter and the previous one, Denth’s mercenaries—who were being hidden in the tunnels beneath the palace—quietly killed the two soldiers who were standing guard at Siri’s door. They are also, along with the Lifeless that Bluefingers broke, securing the Lifeless compound, grabbing Blushweaver, and taking control of the palace.</p>
<p>The priests get wind of this, though, and react by marshaling their own forces. For most of the night, the priests assume that they’re struggling with Idrian rebels who have tried to take the palace and rescue Siri.</p>
<p></p>



