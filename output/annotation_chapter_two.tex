\section{Annotation Chapter Two}

\subsection*{Siri Rides South, Stunned}

Already, you should be able to see another tone shift in the book. We’ve gone from lazy highland romping to frustration and terror. My goal with this book was to keep that up—to always have things moving and the characters being pulled out of their comfortable lives into situations that force them to stretch.

One fun thing you can research yourself by looking at the first draft chapters of \textit{Warbreaker} I posted. In them, I toyed with having Mab the cook be sent with Siri to be a lady’s maid.

I didn’t intend this while planning the book, but after writing Mab—and having so much fun with her character—I wanted to keep hold of her and let her add some color to Siri’s sections. However, I cut this idea out pretty quickly. (Though a draft of this chapter exists with Mab accompanying Siri—I think in that draft, Mab is the one Siri is complaining to, rather than the poor guard outside the window.)

Why cut Mab? Well, a couple of reasons. First off, Siri’s plotline was much more dramatic and emotional if she was forced to leave behind everything she’d known. Giving her a support character like Mab undermined Siri’s plot and growth as a character. Beyond that, Siri’s plots didn’t \textit{need} more color. We’ve got plenty of interesting characters and experiences coming for her, so the addition of another character wasn’t needed.

I tried the chapter, but then realized that my original instincts had been right. I was forced to cut Mab out.

\subsection*{Character Shifts}

This is a fun chapter, formatwise. It looks simple—we’ve got two alternating sequences with Siri and Vivenna. But what’s going on here is that I’m trying to pull the first of many reversals in this book.

A reversal is more than just a plot twist—it’s a swap. (Or at least that’s how I define it in my head.) Just like \textit{Elantris}’s substructure was that of the chapter triads, \textit{Warbreaker}’s substructure is that of reversals. People change places or do 180-degree turns. This presented a challenge to me, as I had to work hard to make such often-abrupt changes well foreshadowed and rational. That’s rather difficult to pull off. Most twists take characters in a slightly new direction; spinning them around completely required a lot more groundwork.

If you’ve read other annotations of mine, you’ll probably know that I love twists—but I love them only in that I love to make them \textit{work}. A good twist has to be rational and unexpected at the same time. Pulling off that balance is one of the great pleasures in writing.

In this chapter, we have the beginnings of the first big reversal in this book. It’s more gradual—not an abrupt one-eighty, but a slow and purposeful one-eighty. But the seeds are here, even in this early chapter. If you look at it, we have this:

Scene One: Siri acts just like we expect Siri to. Blustering and emotional.

Scene Two: Vivenna acts just like we expect Vivenna to. Calm, rational, in control, and willing to do as she is told.

Scene Three: Siri grows calm, considers her situation with more care, and acts a little bit like a queen should in deciding to send her soldiers back.

Scene Three: Vivenna is very bothered by what is happening and acts just a little bit like Siri would—she decides upon a plan that is impetuous.

I’m very excited by the underlying structure of the chapter, even though I’m aware that most people probably wouldn’t be. I’m just a screwy author type. <img draggable="false" role="img" class="emoji" alt="😉" src="https://s.w.org/images/core/emoji/14.0.0/svg/1f609.svg"> I like how the changes are very subtle, and yet already there are hints at the way the characters are heading in life.

I like reversals and tone changes, but I still think that readers deserve to have an understanding of what the major plots and arcs for a character will be. There will be twists, but I don’t want to just twist needlessly or endlessly. The characters are the most important part of the story, and one thing I rarely twist (particularly late in a book) is a character’s personal arc. I keep personal arcs steady, as they’re the foundation of a reader’s attachment to the book.



