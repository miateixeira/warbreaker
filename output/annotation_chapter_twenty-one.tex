\section{Annotation Chapter Twenty-One}

\subsection*{Vasher Awakens His Clothing, Then Leaps off the Palace}

One of the subtle, yet drastic, changes to Awakening that happened in this story came during the middle drafts. Originally, the Command was part of the process of Awakening—but wasn’t as important as I eventually made it. I had intended for very difficult things to be accomplished through the use of very long and intricate Commands. However, as I wrote the first draft, I felt this was bulky. What it meant that was if you wanted to use a powerful Awakening in battle, you’d have to stop and spout several paragraphs of instructions. It really cut down on the tension of the battle sequences. (And Awakening was already slower than I preferred, with the need for all of the steps—Breathing, finding color, then Commanding.)

So during revisions, I changed this. Instead of requiring a lengthy Command to create a powerful Awakening, the strength and skill of the Awakener is instead determined by their ability to visualize what they want the Command to do. The Command is a focus, the spoken words an important part of the process, but the real trick is getting the right mental picture.

This way, someone can practice a lot, and still use simple Commands—like “grab things”—yet have them do very powerful things. It also allows me to have Commands be easier to learn and use, yet still require skill to master.

\subsection*{Vasher Pretends to Be Crazy, Approaches the Guards}

This line about gods attracting the unhinged comes a little bit from personal experience. Many of you may know that in the LDS church, we often serve missions during the early part of our twenties or our late teens. I did this, moving to Korea for two years and doing service, teaching about the church, and generally having a blast living among and learning from another culture.

One thing I learned, however, is that when you’re associated with anything religious in a formal way like that, you tend to attract people of . . . interesting inclinations. I got to listen to a surprising number of people who weren’t all there tell me about things they’d seen or decided upon. (And note, this isn’t me trying to make fun of other religions or other beliefs—I, of course, got to speak with a lot of people who believed differently from myself. No, in this case, I’m referring to the mentally challenged people who—for whatever reason—liked to search out missionaries and talk to them.)

It was a lot of fun, don’t get me wrong. But it was also weird.

Anyway, I would assume these guards are accustomed to dealing with the unbalanced. Though entry into the Court of Gods is restricted, it’s hardly impossible to get in. With the lottery, and with the numbers of performers and artists coming into the place every day, you can sneak in without too much difficulty. At least up until what happens this night, after which things become a lot more strict.

I imagine that Mercystar, somewhat vain though she is, intentionally hired men to be her guards who were of a kindly disposition. She’s a good woman, if a bit of a drama queen. In my mind, most of the people working in the Court of Gods are generally good people. But perhaps that’s my personal bias that religion—when it’s not being manipulated and used for terrible purposes—does wonderful things for people.

\subsection*{Vasher Fights the Guards, Then Creates a Lifeless Squirrel}

I wanted to show the creation of a Lifeless somewhere in this book, as I think the process is interesting. The draining of color happens in a slightly different way than in regular Awakening, though it’s similar. In this case, the creature draws color from its own body in order to come to life.

The better your imagining of the Command when you make it (not the orders you give it, but the one when you give it the Breath), the more intelligent and capable of following orders the Lifeless is. Later in the book, for instance, people are surprised at how good this little squirrel is at doing what it is told.



<p></p>
<p>The contact Vasher mentions in this scene is Bluefingers. The little scribe is working very hard to push the court toward war, and he thinks that if Vasher sneaks into the hidden tunnels, he might do something dangerous like kill a few guards. More than that, Bluefingers is hoping that by giving away that tidbit of information, he might be able to get Vasher to trust him, and therefore get the chance to manipulate him toward fomenting the war.</p>
<p>At this point, Vasher has contacted Bluefingers pretending that he’s interested in the politics of the court and the war. Bluefingers inaccurately assumes—from intelligence he’s gathered, from what Denth has said, and from some faint awareness of who Vasher might be—that Vasher wants to drive Hallandren back to war with Idris. At the very least, Bluefingers assumes that Vasher will want to kill and destroy, since death and destruction have often been his wake.</p>
<p>And so, Bluefingers sells to Vasher a little tidbit that he assumes is innocent (the presence of the tunnels). This gives Vasher an unexpected edge. He now knows that it’s possible to get to the Lifeless garrison, and into the court itself, through ways nobody knows about. That makes him suspect that something greater might be going on, perhaps a coup of some sort.</p>
<p>I apologize for only showing little pieces of this in the book. But, to be honest, I don’t think it’s that interesting—mostly because everybody is so wrong about what they’re assuming. And the assumptions are rational enough that I think it would be confusing in the book. Vasher is wrong about the coup, and Bluefingers is wrong about Vasher’s motives. Denth only cares about getting a chance to punish Vasher for the death of his sister.</p>
<p></p>



