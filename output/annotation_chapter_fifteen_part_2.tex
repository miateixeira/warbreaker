\section{Annotation Chapter Fifteen Part 2}

\subsection*{Vivenna Sees Vasher}

I’m sorry that we don’t get to see Vasher as much as you all want. I considered adding more chapters in. I considered it several times during several rewrites. In the end, I just decided that his viewpoints had to remain as they were in the early part of the book. If you see too much of what he’s doing, it will give away things I don’t want to give away.

I don’t like having viewpoints that fail to reveal things about the characters and their emotions and plans. It feels like I’m lying to the reader when I hide things the viewpoint character knows. I avoid it when I can (though I can’t always—reference Kelsier in \textit{Mistborn}).

Either way, I just decided to keep Vasher as he was, with only occasional appearances.

\subsection*{Life Sense as Part of the Magic}

The ability of the Heightenings and Breath to give people an added dose of life sense was part of my attempts to make Awakening, as a magic system, feel more visceral and real. Allomancy is a great magic system, but I wanted a different feel here. In Allomancy, the powers granted are more like superpowers; with Awakening, I wanted something that felt . . . well, closer to what people already do.

Perfect pitch and perfect color recognition are two things that I think resonate this way; the ability to bring inanimate objects to life may seem wildly superpowerish, but I think it’s a part of our own superstition and mythology—or at least the superstition and mythology of our past. Life from things inanimate, like spontaneous generation, was long assumed as something real. Witches were often thought to be able to bring sticks or bundles of cloth to life.

I think that there’s still a lot of superstition in our modern world regarding how it feels to have someone watching you. We are more aware of our surroundings, sometimes, than we realize. I think we attribute a supernatural connection to some of these things. Who knows? Maybe there is one. I don’t know, perhaps I’ve got a bit of it myself.

Enhancing this and making it part of the magic was a way to get the visceral feel I was looking for. It also plays off the idea that by giving up your Breath, you give up part of your life. The fact that Drabs can’t be noticed by life sense allows me to show that they have taken one more step toward being objects themselves.

BioChroma. It turns objects into living things, but turns living things into objects as well.



