\section{Chapter Three}

Lightsong didn’t remember dying.

His priests, however, assured him that his death had been extremely inspiring. Noble. Grand. Heroic. One did not Return unless one died in a way that exemplified the great virtues of human existence. That was why the Iridescent Tones sent the Returned back; they acted as examples, and gods, to the people who still lived.

Each god represented something. An ideal related to the heroic way in which they had died. Lightsong himself had died displaying extreme bravery. Or, at least, that was what his priests told him. Lightsong couldn’t remember the event, just as he couldn’t remember anything of his life before he became a god.

He groaned softly, unable to sleep any longer. He rolled over, feeling weak as he sat up in his majestic bed. Visions and memories pestered his mind, and he shook his head, trying to clear away the fog of sleep.

Servants entered, responding wordlessly to their god’s needs. He was one of the younger divinities, for he’d Returned only five years before. There were some two dozen deities in the Court of Gods, and many were far more important—and far more politically savvy—than Lightsong. And above them all reigned Susebron, the God King of Hallandren.

Young though he was, he merited an enormous palace. He slept in a room draped with silks, dyed with bright reds and yellows. His palace held dozens of different chambers, all decorated and furnished according to his whims. Hundreds of servants and priests saw to his needs—whether he wanted them seen to or not.

\textit{All of this,}~he thought as he stood,~\textit{because I couldn’t figure out how to die.}~Standing made him just a bit dizzy. It was his feast day. He would lack strength until he ate.

Servants approached carrying brilliant red and gold robes. As they entered his aura, each servant—skin, hair, clothing, and garments—burst with exaggerated color. The saturated hues were far more resplendent than any dye or paint could produce. That was an effect of Lightsong’s innate BioChroma: he had enough Breath to fill thousands of people. He saw little value in it. He couldn’t use it to animate objects or corpses; he was a god, not an Awakener. He couldn’t give—or even loan—his deific Breath away.

Well, except once. That would, however, kill him.

The servants continued their ministrations, draping him with gorgeous cloth. Lightsong was a good head and a half taller than anyone else in the room. He was also broad of shoulder, with a muscular physique that he didn’t deserve, considering the amount of time he spent idle.

“Did you sleep well, Your Grace?” a voice asked.

Lightsong turned. Llarimar, his high priest, was a tall, portly man with spectacles and a calm demeanor. His hands were nearly hidden by the deep sleeves of his gold and red robe, and he carried a thick tome. Both robes and tome burst with color as they entered Lightsong’s aura.

“I slept fantastically, Scoot,” Lightsong said, yawning. “A night full of nightmares and obscure dreams, as always. Terribly restful.”

The priest raised an eyebrow. “Scoot?”

“Yes,” Lightsong said. “I’ve decided to give you a new nickname. Scoot. Seems to fit you, the way you’re always scooting around, poking into things.”

“I am honored, Your Grace,” Llarimar said, seating himself on a chair.

\textit{Colors,}~Lightsong thought.~\textit{Doesn’t he ever get annoyed?}

Llarimar opened his tome. “Shall we begin?”

“If we must,” Lightsong said. The servants finished tying ribbons, doing up clasps, and draping silks. Each bowed and retreated to a side of the room.

Llarimar picked up his quill. “What, then, do you remember of your dreams?”

“Oh, you know.” Lightsong flopped back onto one of his couches, lounging. “Nothing really important.”

Llarimar pursed his lips in dis pleasure. Other servants began to file in, bearing various dishes of food. Mundane, human food. As a Returned, Lightsong didn’t really need to eat such things—they would not give him strength or banish his fatigue. They were just an indulgence. In a short time, he would dine on something far more~.~.~. divine. It would give him strength enough to live for another week.

“Please try to remember the dreams, Your Grace,” Llarimar said in his polite, yet firm, way. “No matter how unremarkable they may seem.”

Lightsong sighed, looking up at the ceiling. It was painted with a mural, of course. This one depicted three fields enclosed by stone walls. It was a vision one of his predecessors had seen. Lightsong closed his eyes, trying to focus. “I~.~.~. was walking along a beach,” he said. “And a ship was leaving without me. I don’t know where it was going.”

Llarimar’s pen began to scratch quickly. He was probably finding all kinds of symbolism in the memory. “Were there any colors?” the priest asked.

“The ship had a red sail,” Lightsong said. “The sand was brown, of course, and the trees green. For some reason, I think the ocean water was red, like the ship.”

Llarimar scribbled furiously—he always got excited when Lightsong remembered colors. Lightsong opened his eyes and stared up at the ceiling and its brightly colored fields. He reached over idly, plucking some cherries off a servant’s plate.

Why should he begrudge the people his dreams? Even if he found divination foolish, he had no right to complain. He was remarkably fortunate. He had a deific BioChromatic aura, a physique that any man would envy, and enough luxury for ten kings. Of all the people in the world, he had the least right to be difficult.

It was just that~.~.~. well, he was probably the world’s only god who didn’t believe in his own religion.

“Was there anything else to the dream, Your Grace?” Llarimar asked, looking up from his book.

“You were there, Scoot.”

Llarimar paused, paling just slightly. “I~.~.~. was?”

Lightsong nodded. “You apologized for bothering me all the time and keeping me from my debauchery. Then you brought me a big bottle of wine and did a dance. It was really quite remarkable.”

Llarimar regarded him with a flat stare.

Lightsong sighed. “No, there was nothing else. Just the boat. Even that is fading.”

Llarimar nodded, rising and shooing back the servants—though, of course, they remained in the room, hovering with their plates of nuts, wine, and fruit, should any of it be wanted. “Shall we get on with it then, Your Grace?” Llarimar asked.

Lightsong sighed, then rose, exhausted. A servant scuttled forward to redo one of the clasps on his robe, which had come undone as he sat.

Lightsong fell into step beside Llarimar, towering at least a foot over the priest. The furniture and doorways, however, were built to fit Lightsong’s increased size, so it was the servants and priests who seemed out of place. They passed from room to room, using no hallways. Hallways were for servants, and they ran in a square around the outside of the building. Lightsong walked on plush rugs from the northern nations, passing the finest pottery from across the Inner Sea. Each room was hung with paintings and gracefully calligraphed poems, created by Hallandren’s finest artists.

At the center of the palace was a small, square room that deviated from the standard reds and golds of Lightsong’s motif. This one was bright with ribbons of darker colors—deep blues, greens, and blood reds. Each was a true color, directly on hue, as only a person who had attained the Third Heightening could distinguish.

As Lightsong stepped into the room, the colors blazed to life. They became brighter, more intense, yet somehow remained dark. The maroon became a more true maroon, the navy a more powerful navy. Dark yet bright, a contrast only Breath could inspire.

In the center of the room was a child.

\textit{Why does it always have to be a child?}~Lightsong thought.

Llarimar and the servants waited. Lightsong stepped forward, and the little girl glanced to the side, where a couple of priests stood in red and gold robes. They nodded encouragingly. The girl looked back toward Lightsong, obviously nervous.

“Here now,” Lightsong said, trying to sound encouraging. “There’s nothing to fear.”

And yet, the girl trembled.

Lecture after lecture—delivered by Llarimar, who had claimed that they were~\textit{not}~lectures, for one did not lecture gods—drifted through Lightsong’s head. There was nothing to fear from the Returned gods of the Hallandren. The gods were a blessing. They provided visions of the future, as well as leadership and wisdom. All they needed to subsist was one thing.

Breath.

Lightsong hesitated, but his weakness was coming to a head. He felt dizzy. Cursing himself quietly, he knelt down on one knee, taking the girl’s face in his oversized hands. She began to cry, but she said the words, clear and distinct as she had been taught. “My life to yours. My Breath become yours.”

Her Breath flowed out, puffing in the air. It traveled along Lightsong’s arm—the touch was necessary—and he drew it in. His weakness vanished, the dizziness evaporated. Both were replaced with crisp clarity. He felt invigorated, revitalized,~\textit{alive.}

The girl grew dull. The color of her lips and eyes faded slightly. Her brown hair lost some of its luster; her cheeks became more bland.

\textit{It’s nothing,}~he thought.~\textit{Most people say they can’t even tell that their Breath is gone. She’ll live a full life. Happy. Her family will be well paid for her sacrifice.}

And Lightsong would live for another week. His aura didn’t grow stronger from Breath upon which he fed; that was another difference between a Returned and an Awakener. The latter were sometimes regarded as inferior, man-made approximations of the Returned.

Without a new Breath each week, Lightsong would die. Many Returned outside of Hallandren lived only eight days. Yet with a donated Breath a week, a Returned could continue to live, never aging, seeing visions at night which would supposedly provide divinations of the future. Hence the Court of the Gods, filled with its palaces, where gods could be nurtured, protected, and—most importantly—fed.

Priests hustled forward to lead the girl out of the room.~\textit{It is nothing to her,}~Lightsong told himself again.~\textit{Nothing at all.~.~.~.}

Her eyes met his as she left, and he could see that the twinkle was gone from them. She had become a Drab. A Dull, or a Faded One. A person without Breath. It would never grow back. The priests took her away.

Lightsong turned to Llarimar, feeling guilty at his sudden energy. “All right,” he said. “Let’s see the Offerings.”

Llarimar raised an eyebrow over his bespectacled eyes. “You’re accommodating all of a sudden.”

\textit{I need to give something back,}~Lightsong thought.~\textit{Even if it’s something useless.}

They passed through several more rooms of red and gold, most of which were perfectly square with doors on all four sides. Near the eastern side of the palace, they entered a long, thin room. It was completely white, something very unusual in Hallandren. The walls were lined with paintings and poems. The servants stayed outside; only Llarimar joined Lightsong as he stepped up to the first painting.

“Well?” Llarimar asked.

It was a pastoral painting of the jungle, with drooping palms and colorful flowers. There were some of these plants in the gardens around the Court of Gods, which was why Lightsong recognized them. He’d never actually been to the jungle—at least, not during this incarnation of his life.

“The painting is all right,” Lightsong said. “Not my favorite. Makes me think of the outside. I wish I could visit.”

Llarimar looked at him quizzically.

“What?” Lightsong said. “The court gets old sometimes.”

“There isn’t much wine in the forest, Your Grace.”

“I could make some. Ferment~.~.~. something.”

“I’m sure,” Llarimar said, nodding to one of his aides outside the room. The lesser priest scribbled down what Lightsong had said about the painting. Somewhere, there was a city patron who sought a blessing from Lightsong. It probably had to do with bravery—perhaps the patron was planning to propose marriage, or maybe he was a merchant about to sign a risky business deal. The priests would interpret Lightsong’s opinion of the painting, then give the person an augury—either for good or for ill—along with the exact words Lightsong had said. Either way, the act of sending a painting to the god would gain the patron some measure of good fortune.

Supposedly.

Lightsong moved away from the painting. A lesser priest rushed forward, removing it. Most likely, the patron hadn’t painted it himself, but had instead commissioned it. The better a painting was, the better a reaction it tended to get from the gods. One’s future, it seemed, could be influenced by how much one could pay one’s artist.

\textit{I shouldn’t be so cynical,}~Lightsong thought.~\textit{Without this system, I’d have died five years ago.}

Five years ago he~\textit{had}~died, even if he still didn’t know what had killed him. Had it really been a heroic death? Perhaps nobody was allowed to talk about his former life because they didn’t want anyone to know that Lightsong the Bold had actually died from a stomach cramp.

To the side, the lesser priest disappeared with the jungle painting. It would be burned. Such offerings were made specifically for the intended god, and only he—besides a few of his priests—was allowed to see them. Lightsong moved along to the next work of art on the wall. It was actually a poem, written in the artisans’ script. The dots of color brightened as Lightsong approached. The Hallandren artisans’ script was a specialized system of writing that wasn’t based on form, but on color. Each colored dot represented a different sound in Hallandren’s language. Combined with some double dots—one of each color—it created an alphabet that was a nightmare for the colorblind.

Few people in Hallandren would admit to having~\textit{that}~particular ailment. At least, that was what Lightsong had heard. He wondered if the priests knew just how much their gods gossiped about the outside world.

The poem wasn’t a very good one, obviously composed by a peasant who had then paid someone else to translate it to the artisans’ script. The simple dots were a sign of this. True poets used more elaborate symbols, continuous lines that changed color or colorful glyphs that formed pictures. A lot could be done with symbols that could change shape without losing their meaning.

Getting the colors right was a delicate art, one that required the Third Heightening or better to perfect. That was the level of Breath at which a person gained the ability to sense perfect hues of color, just as the Second Heightening gave someone perfect pitch. Returned were of the Fifth Heightening. Lightsong didn’t know what it was like to live without the ability to instantly recognize exact shades of color and sound. He could tell an ideal red from one that had been mixed with even one drop of white paint.

He gave the peasant’s poem as good a review as he could, though he generally felt an impulse to be honest when he looked at Offerings. It seemed his duty, and for some reason it was one of the few things he took seriously.

They continued down the line, Lightsong giving reviews of the various paintings and poems. The wall was remarkably full this day. Was there a feast or celebration he hadn’t heard about? By the time they neared the end of the line, Lightsong was tired of looking at art, though his body—fueled by the child’s Breath—continued to feel strong and exhilarated.

He stopped before the final painting. It was an abstract work, a style that was growing more and more popular lately—particularly in paintings sent to him, since he’d given favorable reviews to others in the past. He almost gave this one a poor grade simply because of that. It was good to keep the priests guessing at what would please him, or so some of the gods said. Lightsong sensed that many of them were far more calculating in the way that they gave their reviews, intentionally adding cryptic meanings.

Lightsong didn’t have the patience for such tricks, especially since all anyone ever really seemed to want from him was honesty. He gave this last painting the time it deserved. The canvas was thick with paint, every inch colored with large, fat strokes of the brush. The predominant hue was a deep red, almost a crimson, that Lightsong immediately knew was a red-blue mixture with a hint of black in it.

The lines of color overlapped, one atop another, almost in a progression. Kind of like~.~.~. waves. Lightsong frowned. If he looked at it right, it looked like a sea. And could that be a ship in the center?

Vague impressions from his dream returned to him. A red sea. The ship, leaving.

\textit{I’m imagining things,}~he told himself. “Good color,” he said. “Nice patterns. It puts me at peace, yet has a tension to it as well. I approve.”

Llarimar seemed to like this response. He nodded as the lesser priest—who stood a distance away—recorded Lightsong’s words.

“So,” Lightsong said. “That’s it, I assume?”

“Yes, Your Grace.”

\textit{One duty left,}~he thought. Now that Offerings were done, it would be time to move on to the final—and least appealing—of his daily tasks. Petitions. He had to get through them before he could get to more important activities, like taking a nap.

Llarimar didn’t lead the way toward the petition hall, however. He simply waved a lesser priest over, then began to flip through some pages on a clipboard.

“Well?” Lightsong asked.

“Well what, Your Grace?”

“Petitions.”

Llarimar shook his head. “You aren’t hearing petitions today, Your Grace. Remember?”

“No. I have~\textit{you}~to remember things like that for me.”

“Well, then,” Llarimar said, flipping a page over, “consider it officially remembered that you have no petitions today. Your priests will be otherwise employed.”

“They will?” Lightsong demanded. “Doing what?”

“Kneeling reverently in the courtyard, Your Grace. Our new queen arrives today.”

Lightsong froze.~\textit{I really need to pay more attention to politics.}~“Today?”

“Indeed, Your Grace. Our lord the God King will be married.”

“So soon?”

“As soon as she arrives, Your Grace.”

\textit{Interesting,}~Lightsong thought.~\textit{Susebron getting a wife.}~The God King was the only one of the Returned who could marry. Returned couldn’t produce children—save, of course, for the king, who had never drawn a breath as a living man. Lightsong had always found the distinction odd.

“Your Grace,” Llarimar said. “We will need a Lifeless Command in order to arrange our troops on the field outside the city to welcome the queen.”

Lightsong raised an eyebrow. “We plan to attack her?”

Llarimar gave him a stern look.

Lightsong chuckled. “Fledgling fruit,” he said, giving up one of the Command phrases that would let others control the city’s Lifeless. It wasn’t the core Command, of course. The phrase he’d given to Llarimar would allow a person to control the Lifeless only in noncombat situations, and it would expire one day after its first use. Lightsong often thought that the convoluted system of Commands used to control the Lifeless was needlessly complex. However, being one of the four gods to hold Lifeless Commands~\textit{did}~make him rather important at times.

The priests began to chat quietly about preparations. Lightsong waited, still thinking about Susebron and the impending wedding. He folded his arms and rested against the side of the doorway.

“Scoot?” he asked.

“Yes, Your Grace?”

“Did I have a wife? Before I died, I mean.”

Llarimar hesitated. “You know I cannot speak of your life before your Return, Lightsong. Knowledge of your past won’t do anyone any good.”

Lightsong leaned his head back, resting it against the wall, looking up at the white ceiling. “I~.~.~. remember a face, sometimes,” he said softly. “A beautiful, youthful face. I think it might have been her.”

The priests hushed.

“Inviting brown hair,” Lightsong said. “Red lips, three shades shy of the seventh harmonic, with a deep beauty. Dark tan skin.”

A priest scuttled forward with the red tome, and Llarimar started writing furiously. He didn’t prompt Lightsong for more information, but simply took down the god’s words as they came.

Lightsong fell silent, turning away from the men and their scribbling pens.~\textit{What does it matter?}~he thought.~\textit{That life is gone. Instead, I get to be a god. Regardless of my belief in the religion itself, the perks are nice.}

He walked away, trailed by a retinue of servants and lesser priests who would see to his needs. Offerings done, dreams recorded, and petitions canceled, Lightsong was free to pursue his own activities.

He didn’t return to his main chambers. Instead, he made his way out onto his patio deck and waved for a pavilion to be set up for him.

If a new queen was going to arrive today, he wanted to get a good look at her.

