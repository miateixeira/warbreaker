\section{Annotation Chapter Five}

\subsection*{Vasher Watches Siri}

We get back to Vasher. Showing his viewpoints in the book was tricky. I wanted him to remain an enigma—I didn’t want you to know if you were rooting for him or not. But I did want you to think he was interesting.

\subsection*{Nightblood Origins}

I’ve been wanting to do a book with a talking sword for some time. Sentient objects are a favorite theme of mine from fantasy books I’ve read, and I think you’ll probably see more of them in future books from me.

The magic sword is its own archetype in fantasy, even if there haven’t been any good magic sword books among the big fantasy novels of recent years. Perhaps that’s because Saberhagen and Moorcock did such a good job with their books in the past. I’m not sure. (I don’t count appearances of magic swords like Callandor in the \textit{Wheel of Time}. I mean books with major parts played by swords.)

Anyway, that’s a tangent, and I’m certain that half the people reading this can think of examples and exceptions to what I just said. Either way, this is a theme I wanted to tackle, and the magic system of this world lent me the opportunity.

Nightblood is another favorite character of the readers. I think his personality works the best out of any non-viewpoint character I’ve ever written. He doesn’t get that much dialogue in the book, but it is so distinctive that it just works.

\subsection*{Vasher Meets Bebid the Priest for Food}

Restaurants. They didn’t really exist in a lot of medieval cultures. Now, most of my books don’t take place in medieval times—they’re more preindustrial uchronias, late renaissance if you will. \textit{Warbreaker} is no exception.

T’Telir seems the kind of place that would have restaurants. Places to sit idly, eating and chatting. It is a successful port city with a lot of trade and a great deal of wealth. There’s even something of a middle class, another concept that didn’t exist during a lot of periods in time.

Originally, I had Vasher make an oblique comment about Bebid’s daughter as a way to get him to talk. However, I shied away from this in later drafts, moving to more nebulous indiscretions instead. I felt that a comment about a daughter might sound too much like kidnapping on Vasher’s part, even though I was thinking that his daughter had done something embarrassing that, if revealed, would get the priest into trouble.



