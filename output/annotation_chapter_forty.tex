\section{Annotation Chapter Forty}

\subsection*{Siri Bullies Treledees}

Treledees, by the way, used to be called Tridees. My editor didn’t like the name, so I swapped it.

I wanted a good, strong scene where we could see that Siri made the decision to keep her hair in check. Again, I’m moving her and Vivenna into different roles, but I want it to be natural, an evolution of their characters brought on by who they are and how their surroundings affect them.

In this case, living in the Court of Gods, there is a very good reason to learn to control your hair. If many are like Treledees, who is of the Third Heightening, then even the most minute changes in your hair color will tip them off.

This is one of the interactions of the magic system that was nice to connect, an interaction I didn’t expect or anticipate. With a lot of Breath, you can perceive very slight changes in color. With the Royal Locks, your hair responds to even your slightest emotions. Put the two together, and you get this scene. It was, in a way, inevitable from the beginning of the book.

Siri has come a long way. She’s still stumbling about and making a lot of mistakes. But she’s also winning some victories. There’s nothing hidden to learn about this chapter; she really did just one-up Treledees and get what she wanted.

\subsection*{The God King’s Priests}

Treledees explains, finally, why it is that the God King’s tongue was removed. I hope this makes sense. Or, more accurately, I hope that Treledees’s explanation and rationalizations make sense. I don’t want the priesthood to come off as too evil in these books. In fact, because we’re seeing through the eyes of so many Idrians, I work very hard to show the Idrians (and the reader) their prejudices.

This isn’t because I wanted to write a book about prejudice. It’s because I wanted to tell a good story, and I believe that a good story works to show all sides of a conflict. Since we don’t have any viewpoints from the priests, I felt I needed several reminders (like the confrontation between Vivenna and Jewels) to explain the Hallandren viewpoint.



