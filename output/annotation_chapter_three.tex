\section{Annotation Chapter Three}

\subsection*{Similarities Between \textit{Warbreaker} and \textit{Elantris}}

And finally, we arrive at my personal favorite character in the book. Lightsong the Bold, the god who doesn’t believe in his own religion.

I had the idea for Lightsong a number of years ago. My first book, \textit{Elantris}, dealt with the concept of men who were made gods. However, in that book, we never actually get to see men living \textit{as} gods. The gods have lost their powers and have been locked away.

This time I wanted to tell a different story, a story about what it is like to live as a member of a pantheon of deities. Yet I didn’t want them to be too powerful. Or even powerful at all.

I realize that there is some resonance here with \textit{Elantris}. I hope that the concepts don’t seem too much alike. What I wanted to do with this story was look at some of the same ideas in \textit{Elantris}, but turn them about completely. Instead of dealing with gods who had fallen, I wanted to look at gods at the height of their political power. Instead of dealing with people who were ridiculously powerful, I wanted gods who were more about prophecy and wisdom.

I made it so that the Returned couldn’t remember their old lives as a way to distinguish them from the Elantrians. However, I can’t help the fact that the ideas had the same (yet opposite) seed. But I’m confident that there’s plenty of room in the idea to explore it in a different direction, and I think this book comes out feeling very much its own novel.

\subsection*{First Line and Lightsong’s Origins}

Lightsong’s character came from a one-line prompt I had pop into my head one day. “Everyone loses something when they die and Return. An emotion, usually. I lost fear.”

Of course, it changed a \textit{lot} from that one line. Still, I see that as the first seed of his character. The idea of telling a story about someone who has died, then come back to life, losing a piece of himself in the return intrigued me.

The other inspiration for him was my desire to do a character who could fit into an Oscar Wilde play. I’m a big fan of Wilde’s works, particularly the comedies, and have always admired how he can have someone be glib and verbally dexterous without coming across as a jerk. Of course, a character like this works differently in a play than in a book. For a story to be epic, you need depth and character arcs you don’t have time for in a play.

So, think of Lightsong as playing a part. When he opens his mouth, he’s usually looking for something flashy to say to distract himself from the problems he feels inside. I think the dichotomy came across very well in the book, as evidenced by how many readers seem to find him to be their favorite character in the novel.



