\section{Annotation Chapter Fifteen}

\subsection*{Siri Sees the God King}

I think this is my favorite plotline of the book. The Siri/God King one, I mean. It’s hard to choose, but this is the one that I felt most interested in. (Though Lightsong’s ending chapters are powerful too.)

I wanted the God King to be an enigma, much like Vasher is, at the beginning of the book. Well . . . that’s not quite true. Right at the beginning, I wanted him to be scary and dangerous. I wanted the reader to perceive him as Siri did.

By now, however, you should be wondering more. Who is he? What are his motives? Is he angry with her or not?

The driving force behind this, actually, is the Lord Ruler. In \textit{Mistborn}, a part of me always felt that he was just a little too stereotypical an evil emperor. True, I worked hard to round him out, particularly through the later books. But writing him made me want to take an evil emperor archetype in a very different direction.

I’ve spoken on the reversals in this book. Well, one thing I realized after the fact is that the novel is—in a lot of ways—about reversals of my own writing. Things I’ve done before, but taken the opposite direction. Almost like I need to react against myself and explore things in new ways, particularly in cases where (like the Lord Ruler) I did things that were more conventional to the genre.

I think that’s why this book has so much resonance with my previous books. Or maybe it doesn’t really, and I’m just seeing something that doesn’t exist. A lot of my ideas in writing, however, come from seeing something done in a movie or a book (or even in one of my own books) and wondering if I could take it a new and different direction. I hope that doesn’t make me feel like I’m repeating myself.

\subsection*{Lightsong Kneels before the God King}

My vote for most thoughtful line of the first chunk of the book? Lightsong’s comment that he’d found that make-believe things were often the only things of substance in people’s lives. (Not quoted directly.)

It’s a little bit cynical, yet somewhat hopeful as well. As Lightsong perceives it, it’s true.

\subsection*{Kalad’s Phantoms}

Kalad used to be Khlad, by the way. I didn’t want his name to sound so Pahn Kahlish, which I signify with the extra \textit{h} sounds to give them an airy feel to their words. I added the mythology of Kalad’s Phantoms to the book late in the process, wishing to give some more depth to the superstitions of the world. And perhaps do some other things too. . . .



