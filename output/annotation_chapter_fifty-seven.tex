\section{Annotation Chapter Fifty-Seven}

\subsection*{Siri Is Led Up to a Room with an Altar}

Well, welcome to my favorite chapter in the book. (Of course, I do tend to say that about the climactic chapters of each book.) For me, this is the kind of chapter that drives one to write a book. The chance to get to it, and to have things start coming together, is the biggest thrill I get in writing.

The “sacrifice Siri on the altar” image was one of the original ones I’d planned for this book, but by the time I got to this place in the novel, it just felt lame to go through with that. It’s such a clichéd image. That’s kind of the point—Bluefingers is trying for something visceral and exaggerated—but I felt that the imagery of it could undermine the entire scene.

I think I did one draft with her tied down to it, but I revised that out pretty quickly. It was far too Snidely Whiplash for me. I like this version much better, where we find out what Bluefingers is going to do, but Siri stands up to him and bullies him into letting her die with dignity. I also went back and seeded the stories about Hallandren and killing people on altars as a superstitious rumor that some Idrians believe. (There were stories about the Mormons, back in the day, claiming that they sacrificed women on the altars of their temples then threw the corpses out the window into the Great Salt Lake. Sounds ridiculous, I know, but in eras without as much media, people can believe some pretty crazy things.)

\subsection*{Vasher and Denth Spar; Vasher Gets Stabbed}

I love scenes in books (when I read them) that imply a great weight of history that we don’t get to see between characters. It gives me a sense that the story is \textit{real}. That these characters lived before the story, and that they’ll continue to live afterward (or, well, the ones who survive).

When I built this book, I knew that the Vasher/Denth relationship needed a lot of groundwork to give it that sense. I wanted them both to be complicated characters who have a twisted past. It all comes to head here, in this chapter, and we get the ending of a story over three centuries old. Will I ever tell those stories? Probably not. Like the story of Alendi and Rashek in \textit{Mistborn}, I think the story between Vasher and Denth is stronger as it stands—as something to lend weight to this book. We \textit{will} go more into the Vasher/Arsteel relationship (particularly as we deal with Yesteel) in the next book, if I write it.

By this point, you should be wondering just who Vasher is. He’s been alive since the Manywar, and Denth implies that Vasher himself caused the conflict. There’s obviously a lot more going on with him than you expect.

\subsection*{Lightsong’s Climactic Scene, with His Vision of the Boat}

Lightsong’s vision and eventual death in this chapter are another of the big focus scenes. In fact, I’d say that this little scene here is my absolute favorite in the book. It’s hard to explain why, but I get a chill whenever I read it. It’s the chill of something you planned that turned out even better than you expected. (As opposed to the planning for the Siri/altar image, which turned out poorly and so had to be cut.)

I worked hard to bring this scene in my head to fruition. No other section of the book has been tweaked more in drafting—everything from changing it so Lightsong grabs the God King’s hand as opposed to his foot, to reworking the imagery of the ocean. (That imagery, by the way, came from my honeymoon while standing on the cruise ship at night and staring into the churning white froth above deep black water.)

Many people on my forums called this event ahead of time—Lightsong healing the God King. I’m fine with that. It \textit{did} seem like a very obvious setup. One character with powers he cannot use until healed, another with the power to heal someone one time. Sometimes it’s okay to give people what they expect—particularly when the result is this scene. I hope they didn’t expect it to be as powerful as it is (assuming readers like the scene as much as I do). I want this one to be very moving.

It’s the final fulfillment of Lightsong’s character. Note that even in the end, his sarcasm and irony come through. He told Siri not to depend on him because he would let her down. Well, Lightsong, you’re a better man than you wanted us to believe. There’s a reason why so many are willing to rely upon you.

\subsection*{Vasher and Denth’s Climax}

I wanted to offer Denth the chance for redemption here, though there was no way he was going to let himself take it. His response is honest. He doesn’t feel he deserves it. He has done terrible things; to wipe away the memory of them would be cheating. Better to just get it over with.

There’s a very good chance that after killing Vasher, Denth would have walked over, picked up Nightblood, and let the sword drain his life away. He wouldn’t have been able to live with the guilt.

But that doesn’t happen. When I first designed this magic system, I added to it the idea that taking a lot of Breath shocks you and sends you into a small seizure of pleasure. This is lifted from the magic system in \textit{Mythwalker}, the story from which I drew Siri and Vivenna. I added the component to Awakening not only because it fit, but because I liked giving one more little nod to \textit{Mythwalker}.

However, the moment I began writing it, I knew that this twist of giving someone Breath, then killing them, would be an awesome way to pull a reversal with the magic. So I built into the story the entire arc of Vasher beating Arsteel mysteriously, and Denth wanting to duel him to prove that he couldn’t win a duel.

Denth was right. Vasher cheated.

Both of these scenes end with a transfer of Breath. That’s intentional; I placed these scenes together on purpose. I love the parallel of one transfer bringing life and hope, the other bringing death.

And by the way, we don’t see Tonk Fah, Jewels, or Clod again in the book. They’ll come back in the sequel. Without Denth’s control, Tonks is off to start murdering and killing wantonly; by the next book, he’ll have changed quite dramatically.

Jewels, on the other hand, is taking Arsteel (Clod) to his brother, who is a master of Lifeless Commands. (Yesteel invented ichor-alcohol.) She hopes to find a way to restore to Arsteel some of his memories and personality.

\subsection*{Siri Is Rescued}

And here we have a big scene that a lot of readers have been waiting for. I apologize for making Siri need to be rescued like this, but I felt it was appropriate to the story. It’s because of her teaching the God King and helping him become the man he is that he’s able to do this.

Remember that the Seventh Heightening (I think that’s the one) grants a person Instinctive Awakening, meaning that once you reach that Heightening, you don’t need any practice to learn to Awaken. Your Commands are obeyed instinctively. This doesn’t mean that everything you try will work, but that you can make most basic Commands (grab things, that sort) work without having to try. In fact, figuring out most of the more complicated, previously unknown Commands requires a person to be of the Seventh Heightening.

This power grew out of me wanting the upper Heightenings to do some very dramatic things. I do worry that this scene is a little deus ex machina. That keeps me from liking it quite as much as the Lightsong climax or the Denth/Vasher climax. But I feel that a story needs a great variety of climactic moments—some internal character moments, some external skill moments, some great twists, some expected payoffs, some big reveals, and some dramatic rescues. This chapter and the next take a shot at trying to cover a lot of those different types.



