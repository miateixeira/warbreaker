\section{Annotation Chapter Forty-Three}

\subsection*{Vivenna Awakens in Vasher’s Care}

Vivenna, as a character, was divided into two parts in my head. There was the Vivenna of the first half of the book, who was haughty and misled, though determined and self-confident. Then there was the break in the middle, where everything was taken away from her. Now we’re into Vivenna’s second half, the confused and uncertain Vivenna who has to essentially start all over.

Her plot is a contrast to Siri’s plot. Siri’s growth is more gradual; she doesn’t have an event like Vivenna’s time on the streets to make a focus for her plotline. The depth of growth the changes afford Vivenna made her a very interesting character to write; I’m sorry that she’s generally people’s least favorite character. But that wasn’t all that unanticipated. When presented with a large group of characters, many of whom were amusing or mysterious, then dropping one major character in who had a serious growth arc but started out less likable . . . well, you expect readers to latch on to other characters. By this point in the story, they’re not used to caring about Vivenna as much as the others, so I think that her drama isn’t as powerful for them—which means she doesn’t have time to earn their affection, even when she starts changing and growing.

Of course, part of me still sees the Vivenna of the sequel, where she can continue her growth and learning. I think she’ll be a great character for that book, if I ever write it. Though I worry about doing so and making people disappointed that I’m writing about her rather than Siri.

\subsection*{Vasher the Hero}

We finally start to get a sense here of Vasher’s true motivations. When designing him as a character, one of my goals was to force myself to stretch. I wanted to tell a story about a hero who was very different from my standard. A person who \textit{wasn’t} glib, who wasn’t good with people. The opposite of Kelsier or Raoden—a man who had trouble expressing himself, who let his anger get the better of him, and who was rough around the edges. You really get to see who he is in this chapter as he shoves Vivenna around and bullies the Idrians.

Vasher tries, and his heart is good, but he just doesn’t have a delicate bone in his body. He doesn’t know how to influence people. He made for a fascinating hero to write for that reason, but it also led me to want to keep him more mysterious from the beginning. I felt that if we spent too much time with him, we wouldn’t be as interested in him. The way people who read the book kept crying for more Vasher and more Nightblood made me think I was right in keeping their chapters sparse—it meant that by the time you reached this point in the book, you were (hopefully) very interested in what he was doing.

\subsection*{Vivenna’s Thoughts on Being a Drab}

A lot of what happened to Vivenna—how she saw the world and how she acted—was influenced by being a Drab. As I’ve said before, the Hallandren aren’t right when they say losing your Breath does nothing to you. Most Drabs struggle with depression, and the fact that they’re almost always sick doesn’t help either.

And so, Vivenna’s time on the streets was artificially made more dreary and terrible than it truly was. Being a Drab, being sick, the shock of being betrayed—these things combined to give you the person you saw in the previous two chapters. It’s a way to cut a corner. I wanted Vivenna to feel like she’d been on the streets for months, but for it only to have been a few weeks.

She is able to make her hair change colors again. This is a representation of the fact that she has started to pull out of the nightmare. She’s slightly in control of her world again, and the roughest time for her has passed. There’s also a clue in that hair, one that Vasher mentions. Because of it, and her heritage, and something very mysterious in the past, every member of the royal line has a fraction of a divine Returned Breath in them. That makes it much easier for them to learn to Awaken than a normal person.

\subsection*{Everyone Is the Hero in Their Own Story}

Another of the big plot events I wanted for this book was to have a character work for the wrong team for a long period of time without realizing it. I’d rarely seen this plot twist in a book, and even more rarely seen anything like it pulled off with any skill. So I wanted to try my hand at it.

Vasher is right here. Denth was playing with her when he told her that line about heroes. He said it partially because he was trying to justify what he was doing, and partially because he was amused that she thought she was doing what was right—when she was a major motivating force driving her people toward destruction.

Vivenna thought she was the hero, but she was the villain—at least for a good chunk of the book.

\subsection*{Vivenna and Vasher Meet with the Idrian Workers}

Now we get to see the other part of what Vasher has been doing all this time—the part that I couldn’t show you earlier, since it would have made it too obvious that he had good intentions. (And that, in turn, might have spoiled the surprise that Denth was manipulating Vivenna.) He’s been trying very hard to convince the Idrians not to get themselves into trouble. He’s been only mildly successful.

Vivenna listening here has some things to work through. Some alpha readers had difficulty with how easily she started helping Vasher, so I’ve reworked that in the final draft. Hopefully you now see her struggle and her reasoning.

What she sees here is something real. She notices that most of Hallandren doesn’t care about Idris or the Idrians. When I lived in Korea, I sensed a lot of resentment from the Koreans toward the Japanese. The Japanese had done some pretty terrible things to the Koreans during the various wars throughout the history of the two countries, and the anger the Koreans felt was quite well justified. The thing is, most Japanese I meet are surprised to hear how much resentment there is. It’s kind of like Americans are sometimes surprised to hear how much dislike there is for them in Mexico.

When you’re the bigger country, the one who historically won conflicts and wars, you often don’t much notice the people you’ve stepped on along the way. While the smaller country may create a rivalry with you, you may not even realize that you \textit{have} a rival. This is what happened with Hallandren and Idris. While some people push for war, the general populace doesn’t even think about Idris—except as that poor group of people up in the highlands who sell them wool and do jobs they, the Hallandren, don’t want to do.

This can be very frustrating for someone from the smaller country, like Vivenna, when confronted not with anger, but with indifference, about your feelings.



