\section{Chapter Thirty-Two}

He was always gone when Siri awoke.

She lay in the deep, well-stuffed bed, morning light streaming through the window. Already, the day was growing warm, and even her single sheet was too hot. She threw it off but remained on the bed, looking up at the ceiling.

She could tell from the sunlight that it was nearly noon. She and Susebron tended to stay up late talking. That was probably a good thing. Some might see that she was getting up later and later each morning, and think that it was due to other activities.

She stretched. At first, it had been strange to communicate with the God King. As the days progressed, however, it was feeling more and more natural to her. She found his writing—uncertain, unpracticed letters that explained such interesting thoughts—to be endearing. If he spoke, she suspected that his voice would be kindly. He was so tender. She’d never have expected that.

She smiled, sinking back into her pillow, idly wishing for him to still be there when she awoke. She was happy. That, also, was something she’d never expected from Hallandren. She did miss the highlands, and her inability to leave the Court of Gods frustrated her, particularly considering the politics.

And yet there were other things. Marvelous things. The brilliant colors, the performers, the sheer~\textit{overwhelming}~experience of T’Telir. And there was the opportunity to speak with Susebron each night. Her brashness had been such a shame and an embarrassment to her family, but Susebron found it fascinating, even alluring.

She smiled again, letting herself dream. However, real life began to intrude. Susebron was in danger. Real, serious danger. He refused to believe that his priests could bear him any sort of malice or be a threat. That same innocence which made him so appealing was also a terrible liability.

But what to do? Nobody else knew of his predicament. There was only one person who could help him. That person, unfortunately, wasn’t up to the task. She had ignored her lessons, and had come to her fate wholly unprepared.

\textit{So what?}~a part of her mind whispered.

Siri stared at the ceiling. She found it hard to summon her customary shame at having ignored her lessons. She’d made a mistake. How much time was she going to spend moping, annoyed at herself for something done and gone?

\textit{All right,}~she told herself.~\textit{Enough excuses. I might not have prepared as well as I should, but I’m here, now, and I need to do something.}

\textit{Because nobody else will.}

She climbed out of bed, running her fingers through her long hair. Susebron liked it long—he found it as fascinating as her serving women did. With them to help her care for it, the length was worth the trouble. She folded her arms, wearing only her shift, pacing. She needed to play their game. She hated thinking of it that way. “Game” implied small stakes. This was no game. It was the God King’s life.

She searched through her memory, dredging up what scraps she could from her lessons. Politics was about exchanges. It was about giving what you had—or what you implied that you had—in order to gain more. It was like being a merchant. You started with a certain stock, and by the end of the year, you hoped to have increased that stock. Or maybe even have changed it into a completely different and better stock.

\textit{Don’t make too many waves until you’re ready to strike,}~Lightsong had told her.~\textit{Don’t appear too innocent, but don’t appear too smart either. Be average.}

She stopped beside the bed, then gathered up the bedsheets and towed them over to the smoldering fire to burn, as was her daily chore.

\textit{Exchanges,}~she thought, watching the sheets catch fire in the large hearth.~\textit{What do I have to trade or exchange? Not much.}

It would have to do.

She walked over and pulled open the door. As usual, a group of serving women waited outside. Siri’s standard ladies moved around her, bringing clothing. Another group of servants, however, moved to tidy the room. Several of these wore brown.

As her servants dressed her, she watched one of the girls in brown. At a convenient moment, Siri stepped over, putting a hand on the girl’s shoulder.

“You’re from Pahn Kahl,” Siri said quietly.

The girl nodded, surprised.

“I have a message I want you to give to Bluefingers,” Siri whispered. “Tell him I have vital information he needs to know. I’d like to trade. Tell him it could change his plans drastically.”

The girl paled, but nodded, and Siri stepped back to continue dressing. Several of the other serving women had heard the exchange, but it was a sacred tenet of the Hallandren religion that the servants of a god weren’t to repeat what they heard in confidence. Hopefully that would hold true. If it didn’t, then she hadn’t really given that much away.

Now she just had to decide just what “vital information” she had, and why exactly Bluefingers should care about it.

\bigskip \hrule \bigskip

“My dear queen!” Lightsong said, actually going so far as to embrace Siri as she stepped into his box at the arena.

Siri smiled as Lightsong waved for her to seat herself in one of his chaise longues. Siri sat with care—she was coming to favor the elaborate Hallandren gowns, but moving gracefully in them took quite a bit of skill. As she settled, Lightsong called for fruit.

“You treat me too kindly,” Siri said.

“Nonsense,” Lightsong said. “You’re my queen! Besides, you remind me of someone of whom I was very fond.”

“And who is that?”

“I honestly have no idea,” Lightsong said, accepting a plate of sliced grapes, then handing them to Siri. “I can barely remember her. Grapes?”

Siri raised an eyebrow, but she knew by now not to encourage him~\textit{too}~much. “Tell me,” she asked, using a little wooden spear to eat her grape slices. “Why do they call you Lightsong the Bold?”

“There is an easy answer to that,” he said, leaning back. “It is because of all the gods, only I am bold enough to act like a complete idiot.”

Siri raised an eyebrow.

“My station requires true courage,” he continued. “You see, I am~\textit{normally}~quite a solemn and boring person. At nights my fondest desire is to sit and compose interminably periphrastic lectures on morality for my priests to read to my followers. Alas, I cannot. Instead, I go out each evening, abandoning didactic theology in favor of something which requires true courage: spending time with the other gods.”

“Why does~\textit{that}~take courage?”

He looked at her. “My lady. Have you~\textit{seen}~how positively tedious they all can be?”

Siri laughed. “No, really,” she said. “Where did the name come from?”

“It’s a complete misnomer,” Lightsong said. “Obviously you’re intelligent enough to see that. Our names and titles are assigned randomly by a small monkey who has been fed an exceedingly large amount of gin.”

“Now you’re just being silly.”

“Now?” Lightsong asked. “\textit{Now?}” he raised a cup of wine toward her. “My dear, I am\textit{always}~silly. Please be good enough to retract that statement at once!”

Siri just shook her head. Lightsong, it appeared, was in rare form this afternoon.~\textit{Great,}~she thought.~\textit{My husband is in danger of being murdered by unknown forces and my only allies are a scribe who’s afraid of me and a god who makes no sense.}

“It has to do with death,” Lightsong finally said as the priests began to file into the arena floor below for this day’s round of arguments.

Siri looked toward him.

“All men die,” Lightsong said. “Some, however, die in ways that exemplify a particular attribute or emotion. They show a spark of something greater than the rest of mankind. That is what is said to bring us back.”

He fell silent.

“You died showing great bravery, then?” Siri asked.

“Apparently,” he said. “I don’t know for sure. Something in my dreams suggests that I may have insulted a very large panther. That sounds rather brave, don’t you think?”

“You don’t know how you died?”

He shook his head. “We forget,” he said. “We awake without memories. I don’t even know what work I did.”

Siri smiled. “I suspect that you were a diplomat or a salesman of some sort. Something that required you to talk a lot, but say very little!”

“Yes,” he said quietly, seeming unlike himself as he stared down at the priests below. “Yes, no doubt that was it exactly~.~.~.” He shook his head, then smiled at her. “Regardless, my dear queen, I have provided a surprise for you this day!”

\textit{Do I want to be surprised by Lightsong?}~She glanced about nervously.

He laughed. “No need to fear,” he said. “My surprises rarely cause bodily harm, and never to beautiful queens.” He waved his hand, and an elderly man with an extraordinarily long white beard approached.

Siri frowned.

“This is Hoid,” Lightsong said. “Master storyteller. I believe you had some questions you wished to ask~.~.~.”

Siri laughed in relief, remembering only now her request to Lightsong. She glanced at the priests below. “Um, shouldn’t we be paying attention to the speeches?”

Lightsong waved indifferently. “Pay attention? Ridiculous! That would be far too responsible of us. We’re gods, for the Colors’ sake. Or, well, I am. You’re close enough. A god-in-law, one might say. Anyway, do~\textit{you}~really want to listen to a bunch of stuffy priests talk about sewage treatment?”

Siri grimaced.

“I thought not. Besides, neither of us have votes pertaining to this issue. So let us spend our time wisely. We never know when we will run out!”

“Of time?” Siri asked. “But you’re immortal!”

“Not run out of time,” Lightsong said, holding up his plate. “Of grapes. I~\textit{hate}~listening to storytellers without grapes.”

Siri rolled her eyes, but continued to eat the grape slices. The storyteller waited patiently. As she looked more closely, she could tell that he wasn’t quite as old as he seemed at first glance. The beard must be a badge of his profession, and while it didn’t appear to be fake, she suspected that it had been bleached. He was really much younger than he wanted to appear.

Still, she doubted Lightsong would have settled for anyone other than the very best. She settled back in her chair—which, she noticed, had been crafted for someone of her size.~\textit{I should be careful with my questions,}~she thought.~\textit{I can’t ask directly about the deaths of the old God Kings; that would be too obvious.}

“Storyteller,” she said. “What do you know of Hallandren history?”

“Much, my queen,” he said, bowing his head.

“Tell me of the days before the division between Idris and Hallandren.”

“Ah,” the man said, reaching into a pocket. He pulled out a handful of sand and began to rub it between his fingers, letting it drop in a soft stream toward the ground, its grains blown slightly in the wind. “Her Majesty wishes one of the~\textit{deep}~stories, from long before. A story before time began?”

“I wish to know the origins of the Hallandren God Kings.”

“Then we begin in the distant haze,” the storyteller said, bringing up another hand, letting powdery black sand drop from it, mixing with the sand that fell from the first hand. As Siri watched, the black sand turned white, and she cocked her head, smiling at the display.

“The first God King of Hallandren is ancient,” Hoid said. “Ancient, yes. Older than kingdoms and cities, older than monarchs and religions. Not older than the mountains, for~\textit{they}~were already here. Like the knuckles of the sleeping giants below, they formed this valley, where panthers and flowers both make their home.

“We speak of just ‘the valley’ then, a place before it had a name. The people of Chedesh still dominated the world. They sailed the Inner Sea, coming from the east, and it was they who first discovered this strange land. Their writings are sparse, their empire has long since been taken by the dust, but memory remains. Perhaps you can imagine their surprise upon arriving here? A place with beaches of fine, soft sand, with fruits aplenty, and with strange, alien forests?”

Hoid reached into his robes and pulled out a handful of something else. He began to drop it before him—small green leaves from the fronds of a fern.

“Paradise, they called it,” Hoid whispered. “A paradise hidden between the mountains, a land with pleasant rains that never grew cold, a land where succulent food grew spontaneously.” He threw the handful of leaves into the air, and in the center of them puffed a burst of colorful dust, like a tiny flameless firework. Deep reds and blues mixed in the air, blowing around him.

“A land of color,” he said. “Because of the Tears of Edgli, the striking flowers of such brilliance that could yield dyes that would hold fast in any cloth.”

Siri had never really thought about how Hallandren would look to people who came across the Inner Sea. She’d heard stories from the ramblemen who came into Idris, and they spoke of distant places. In other lands, one found prairies and steppes, mountains and deserts. But not jungles. Hallandren was unique.

“The First Returned was born during this time,” Hoid said, sprinkling a handful of silver glitter into the air before him. “Aboard a ship that was sailing the coast. Returned can now be found in all parts of the world, but the first one—the man whom you call Vo, but we name only by his title—was born here, in the waters of this very bay. He declared the Five Visions. He died a week later.

“The men of his ship founded a kingdom upon these beaches, then called Hanald. Before their arrival, all that had existed in these jungles was the people of Pahn Kahl, more a mere collection of fishing villages than a true kingdom.”

The glitter ran out, and Hoid began to drop a powdery brown dirt from his other hand as he reached into another pocket. “Now, you may wonder why I must travel back so far. Should I not speak of the Manywar, of the shattering of kingdoms, of the Five Scholars, of Kalad the Usurper and his phantom army, which some say still hides in these jungles, waiting?

“Those are the events we focus upon, the ones men know the best. To speak only of them, however, is to ignore the history of three hundred years that led up to them. Would there have been a Manywar without knowledge of the Returned? It was a Returned, after all, who predicted the war and prompted Strifelover to attack the kingdoms across the mountains.”

“Strifelover?” Siri interrupted.

“Yes, Your Majesty,” Hoid said, switching to a black dust. “Strifelover. Another name for Kalad the Usurper.”

“That sounds like the name of a Returned.”

Hoid nodded. “Indeed,” he said. “Kalad~\textit{was}~Returned, as was Peacegiver, the man who overthrew him and founded Hallandren. We haven’t arrived at~\textit{that}~part yet. We are still back in Hanald, the outpost-become-kingdom founded by the men of the First Returned’s crew. They were the ones who chose the First Returned’s wife as their queen, then used the Tears of Edgli to create fantastic dyes which sold for untold riches across the world. This soon became a bustling center of trade.”

He removed a handful of flower petals and began to let them fall before him. “The Tears of Edgli. The source of Hallandren wealth. Such small things, so easy to grow here. And yet, this is the only soil where they will live. In other parts of the world, dyes are very difficult to produce. Expensive. Some scholars say that the Manywar was fought over these flower petals, that the kingdoms of Kuth and Huth were destroyed by little drips of color.”

The petals fell to the floor.

“But only~\textit{some}~of the scholars say that, storyteller?” Lightsong said. Siri turned, having almost forgotten that he was with her. “What do the rest say? Why was the Manywar fought in\textit{their}~opinions?”

The storyteller fell silent for a moment. And then he pulled out two handfuls and began to release dust of a half-dozen different colors. “Breath, Your Grace. Most agree that the Manywar was not~\textit{only}~about petals squeezed dry, but a much greater prize.~\textit{People}~squeezed dry.

“You know, perhaps, that the royal family was growing increasingly interested in the process by which Breath could be used to bring objects to life. Awakening, it was then first being called. It was a fresh and poorly understood art, then. It still is, in many ways. The workings of the souls of men—their power to animate ordinary objects and the dead to life—is something discovered barely four centuries ago. A short time, by the accounting of gods.”

“Unlike a court proceeding,” Lightsong mumbled, glancing over at the priests who were still talking about sanitation. “Those seem to last an eternity, according to the accounting of this god.”

The storyteller didn’t break stride at the interruption. “Breath,” he said. “The years leading up to the Manywar, those were the days of the Five Scholars and the discovery of new Commands. To some, this was a time of great enlightenment and learning. Others call them the darkest days of men, for it was then we learned to best exploit one another.”

He began to drop two handfuls of dust, one bright yellow, the other black. Siri watched, amused. He seemed to be slanting what he said toward her, careful not to offend her Idrian sensibilities. What did she really know of Breath? She’d rarely even seen any Awakeners in the court. Even when she did, she didn’t really care. The monks had spoken against such things, but, well, she had paid about as much attention to them as she had to her tutors.

“One of the Five Scholars made a discovery,” Hoid continued, dropping a handful of white scraps, small torn pieces of paper with writing on them. “Commands. Methods. The means by which a Lifeless could be created from a~\textit{single Breath.}

“This, perhaps, seems a small thing to you. But you must look at the past of this kingdom and its founding. Hallandren began with the servants of a Returned and was developed by an expansive mercantile effort. It controlled a uniquely lucrative region which, through the discovery and maintenance of the northern passes—combined with increasingly skillful navigation—was becoming a jewel coveted by the rest of the world.”

He paused and his second hand came up, dropping little bits of metal, which fell to the stonework with a sound not unlike falling rain. “And so the war came,” he said. “The Five Scholars split, joining different sides. Some kingdoms gained the use of Lifeless while others did not. Some kingdoms had weapons others could only envy.

“To answer the god’s question, my story claims one other reason for the Manywar: the ability to create Lifeless so cheaply. Before the discovery of the single-Breath Command, Lifeless took fifty Breaths to make. Extra soldiers—even a Lifeless one—are of limited use if you can gain only one for every fifty men you already have. However, being able to create a Lifeless with a single Breath~.~.~. one for one~.~.~. that will double your troops. And half of them won’t need to eat.”

The metal stopped falling.

“Lifeless are no stronger than living men,” Hoid said. “They are the same. They are not more skilled than living men. They are the same. However, not having to~\textit{eat}~like regular men? That advantage was enormous. Mix that with their ability to ignore pain and never feel fear~.~.~. and suddenly you had an army that others could not stand against. It was taken even further by Kalad, who was said to have created a new and more powerful type of Lifeless, gaining an advantage even more frightening.”

“What kind of new Lifeless?” Siri asked, curious.

“Nobody remembers, Your Majesty,” Hoid explained. “The records of that time have been lost. Some say they were burned intentionally. Whatever the true nature of Kalad’s Phantoms, they were frightening and terrible—so much so that even though the details have been lost in time, the phantoms themselves live on in our lore. And our curses.”

“Do they really still exist out there?” Siri asked, shivering slightly, glancing toward the unseen jungles. “Like the stories say? An unseen army, waiting for Kalad to return and command them again?”

“Alas,” Hoid said, “I can tell only stories. As I said, so much from that time is lost to us now.”

“But we know of the royal family,” Siri said. “They broke away because they didn’t agree with what Kalad was doing, right? They saw moral problems with using Lifeless?”

The storyteller hesitated. “Why, yes,” he finally said, smiling through his beard. “Yes, they did, Your Majesty.”

She raised an eyebrow.

“Psst,” Lightsong said, leaning in. “He’s lying to you.”

“Your Grace,” the storyteller said, bowing deeply. “I beg your pardon. There are diverging explanations! Why, I am a teller of stories—all stories.”

“And what do other stories say?” Siri asked.

“None of them agree, Your Majesty,” Hoid said. “Your people speak of religious indignation and of treachery by Kalad the Usurper. The Pahn Kahl people tell of the royal family working hard to gain powerful Lifeless and Awakeners, then being surprised when their tools turned against them. In Hallandren, they tell of the royal family aligning themselves with Kalad, making him their general and ignoring the will of the people by seeking war with bloodlust.”

He looked up, and then began to trail two handfuls of black, burned charcoal. “But time burns away behind us, leaving only ash and memory. That memory passes from mind to mind, then finally to my lips. When all is truth, and all are lies, does it matter if some say the royal family sought to create Lifeless? Your belief is your own.”

“Either way, the Returned took control of Hallandren,” she said.

“Yes,” Hoid said. “And they gave it a new name, a variation on the old one. And yet, some still speak regretfully of the royals who left, bearing the blood of the First Returned to their highlands.”

Siri frowned. “Blood of the First Returned?”

“Yes, of course,” Hoid said. “It was his wife, pregnant with his child, who became the first queen of this land. You are his descendant.”

She sat back.

Lightsong turned, curious. “You didn’t know this?” he asked, in a tone lacking his normal flippancy.

She shook her head. “If this fact is known to my people, we do not speak of it.”

Lightsong seemed to find that interesting. Down below, the priests were moving on to a different topic—something about security in the city and increasing patrols in the slums.

She smiled, sensing a subtle way to get to the questions she~\textit{really}~wanted to ask. “That means that the God Kings of Hallandren carried on~\textit{without}~the blood of the First Returned.”

“Yes, Your Majesty,” Hoid said, crumbling clay out into the air before him.

“And how many God Kings have there been?”

“Five, Your Majesty,” the man said. “Including His Immortal Majesty, Lord Susebron, but not including Peacegiver.”

“Five kings,” she said. “In three hundred years?”

“Yes, Your Majesty,” Hoid said, bringing out a handful of golden dust, letting it fall before him. “The dynasty of Hallandren was founded at the conclusion of the Manywar, the first one gaining his Breath and life from Peacegiver himself, who was revered for dispelling Kalad’s Phantoms and bringing a peaceful end to the Manywar. Since that day, each God King has fathered a stillborn son who then Returned and took his place.”

Siri leaned forward. “Wait. How did Peacegiver create a new God King?”

“Ah,” Hoid said, switching back to sand with his left hand. “Now~\textit{there}~is a story lost in time. How indeed? Breath can be passed from one man to another, but Breath—no matter how much—does not make one a god. Legends say that Peacegiver died by granting his Breath to his successor. After all, can a god not give his life away to bless another?”

“Not exactly a sign of mental stability, in my opinion,” Lightsong said, waving for some more grapes. “You don’t encourage confidence in our predecessors, storyteller. Besides, even if a god gives away his Breath, it doesn’t make the recipient divine.”

“I only tell stories, Your Grace,” Hoid repeated. “They may be truths, they may be fictions. All I know is that the stories themselves exist and that I must tell them.”

\textit{With as much flair as possible,}~Siri thought, watching him reach into yet another pocket and pull free a handful of small bits of grass and earth. He let bits fall slowly between his fingers.

“I speak of foundations, Your Grace,” Hoid said. “Peacegiver was no ordinary Returned, for he managed to stop the Lifeless from rampaging. Indeed, he sent away Kalad’s Phantoms, which formed the main bulk of the Hallandren army. By doing so, he left his own people powerless. He did so in an effort to bring peace. By then, of course, it was too late for Kuth and Huth. However, the other kingdoms—Pahn Kahl, Tedradel, Gys, and Hallandren itself—were brought out of the conflict.

“Can we not assume more from this god of gods who was able to accomplish so much? Perhaps he~\textit{did}~do something unique, as the priests claim. Leave some seed within the God Kings of Hallandren, allowing them to pass their power and divinity from father to son?”

\textit{Heritage which would give them a claim to rule,}~Siri thought, idly slipping a sliced grape into her mouth.~\textit{With such an amazing god as their progenitor, they could become God Kings. And the only one who could threaten them would be~.~.~.}

\textit{The royal family of Idris, who can apparently trace their line back to the First Returned. Another heritage of divinity, a challenger for rightful rule in Hallandren.}

That didn’t tell her how the God Kings had died. Nor did it tell her why some gods—such as the First Returned—could bear children, while others could not.

“They’re immortal, correct?” Siri asked.

Hoid nodded, smoothly dropping the rest of his grass and dirt, moving into a different discussion by bringing forth a handful of white powder. “Indeed, Your Majesty. Like all Returned, the God Kings do not age. Agelessness is a gift for all who reach the Fifth Heightening.”

“But why have there been five God Kings?” she asked. “Why did the first one die?”

“Why do any Returned pass on, Your Majesty?” Hoid asked.

“Because they are loony,” Lightsong said.

The storyteller smiled. “Because they tire. Gods are not like ordinary men. They come back for~\textit{us,}~not for themselves, and when they can no longer endure life, they pass on. God Kings live only as long as it takes them to produce an heir.”

Siri started. “That’s commonly known?” she asked, then cringed slightly at the potentially suspicious comment.

“Of course it is, Your Majesty,” the storyteller said. “At least, to storytellers and scholars. Each God King has passed from this world shortly after his son and heir was born. It is natural. Once the heir has arrived, the God King grows restless. Each one has sought out an opportunity to use up his Breath to benefit the realm. And then~.~.~.”

He threw up a hand, snapping his fingers, throwing up a little spray of water, which puffed to mist.

“And then they pass on,” he said. “Leaving their people blessed and their heir to rule.”

The group fell silent, the mist evaporating in front of Hoid.

“Not exactly the most pleasant thing to inform a newlywed wife, storyteller,” Lightsong noted. “That her husband is going to grow bored with life as soon as she bears him a son?”

“I seek not to be charming, Your Grace,” Hoid said, bowing. At his feet, the various dusts, sands, and glitters mixed together in the faint breeze. “I only tell stories. This one is known to most. I should think that Her Majesty would like to be aware of it as well.”

“Thank you,” Siri said quietly. “It was good of you to speak of it. Tell me, where did you learn such an~.~.~. unusual method of storytelling?”

Hoid looked up, smiling. “I learned it many, many years ago from a man who didn’t know who he was, Your Majesty. It was a distant place where two lands meet and gods have died. But that is unimportant.”

Siri ascribed the vague explanation to Hoid’s desire to create a suitably romantic and mysterious past for himself. Of far more interest to her was what he’d said about the God Kings’ deaths.

\textit{So there is an official explanation,}~she thought, stomach twisting.~\textit{And it’s actually a pretty good one. Theologically, it makes sense that the God Kings would depart once they had arranged for a suitable successor.}

\textit{But that doesn’t explain how Peacegiver’s Treasure—that wealth of Breath—passes from God King to God King when they have no tongues. And it doesn’t explain why a man like Susebron would get tired of life when he seems so excited by it.}

The official story would work fine for those who didn’t know the God King. It fell flat for Siri. Susebron would never do such a thing. Not now.

Yet~.~.~. Would things change if she bore him a son? Would Susebron grow tired of her that easily?

“Maybe we should be~\textit{hoping}~for old Susebron to pass, my queen,” Lightsong said idly, picking at the grapes. “You were forced into all this, I suspect. If Susebron died, you might even be able to go home. No harm done, people healed, new heir on the throne. Everyone is either happy or dead.”

The priests continued to argue below. Hoid bowed, waiting for dismissal.

\textit{Happy~.~.~. or dead.}~Her stomach twisted. “Excuse me,” she said, rising. “I would like to walk about a bit. Thank you for your storytelling, Hoid.”

With that—entourage in tow—she quickly left the pavilion, preferring that Lightsong not see her tears.

