\section{Annotation Chapter Nine}

\subsection*{Vivenna as a Viewpoint Character}

Generally, Vivenna is the readers’ least favorite character in the book. I can see why that is. Siri gets to be the flamboyant younger sister, Lightsong the pithy courtier, and Vasher the mysterious unknown. Vivenna, then, is saddled with the responsibility of being the older sister trying to do what is right. She’s not as dynamic as the others, particularly from the start.

Perhaps this should have made me want to put more into her viewpoints. Change her to be more dynamic, perhaps. However, I resisted that. Of the four, Vivenna is the most like me. The older sibling who gets into other people’s business, ostensibly for their own good. I was a lot like that when I was younger.

For me, Vivenna is the most interesting character in the book. Yes, Lightsong was the most fun to write—but Vivenna is the one who has the most potential for growth and change. Particularly because she isn’t instantly appealing like the other three. Much like Hrathen in \textit{Elantris}, Vivenna begins very far from where she would need to go if she wanted to gain the rooting interest of readers. You’ll have to read on and see if she actually gets there.

\subsection*{Vivenna Watches the City}

One of the reasons I knew that I had to make Vivenna a viewpoint character was that there was such a wonderful contrast between her and Siri. The way they look at the world is so different that it provides excellent opportunities for the story. The way they each respond to their first visit to T’Telir is an example of this.

Beyond that, with Siri and Lightsong locked in the court, and with Vasher doing whatever the heck Vasher is doing, we didn’t have any characters who could experience the city itself consistently with a sympathetic viewpoint.

As I’ve stated, this book began as one about the two sisters who are forced into each other’s roles, and how they deal with those changes in their lives. Vivenna is an integral part of this process.

\subsection*{Parlin as a Character}

Any of you who followed the development of \textit{Warbreaker} as a novel through the early stages know that Parlin, as a character, changed dramatically across revisions. He began with a different name (Peprin) and was much more bumbling and innocent. He provided some comic relief and often said dumb things.

This just didn’t work. For one thing, we already have the mercenaries in Vivenna’s viewpoint to give us some fun lines. (More on them later.) For another, Peprin was just \textit{too} dense. I didn’t like how stupid he came off. He seemed ridiculous rather than funny. So, I chopped him out and replaced him with a similar character who was more competent.

For instance, in the original draft, Peprin bought a hat because he thought it was cool—but it just made him look stupid. Parlin buys the same hat, but his reasoning is that if you’re going to go about in the woods, you dress in woodland colors. If you’re going to go about in the city, you want to start dressing in city colors. It’s good reasoning, and you’ll see him follow it more in the future. The two men do the same thing, but in my head the rationale was completely different, and that changed how I wrote them. (I hope.)

Reading through the book again, I still feel that Parlin just isn’t enough of a character. With the mercenaries there to dominate the scene, Parlin gets lost. I feel that if I had the time, I’d probably chop him out again and replace him with yet another character, one who talks more, so that he can be more a part of things. Ah well.



