\section{Annotation Chapter Fifty-Five}

\subsection*{Treledees Almost Tells Siri How to Pass On the God King’s Breath}

We get to see more of Siri taking charge here. In this tense situation, a lot of others would have been reduced to hysterics, but she’s come into her own, taking command, trying to get the information she needs.

Treledees lies to her here about two things. First off, he \textit{does} know how a God King can have a child, but he knows that the secret is also held by a secure group of priests on the islands. He doesn’t think letting Siri in on that one for now is a good idea. But he does want to pass on how to get Susebron’s Breaths away from him, should it become necessary. He knows that those \textit{need} to be passed on, even if the God King does have a child. That’s the greater secret, but the one that needs to be known to Siri. Those Breaths cannot die with Susebron.

So, anyway, he’s lying about the God King not being able to have a child. (Or at least he sidesteps it. He says that the God King can’t sire a child, which is true unless certain steps are taken. He also says that he doesn’t know how the First Returned bore a child, which is true—he doesn’t know for certain if the First Returned used the same method that Treledees knows. He’s also sidestepping the fact that he \textit{does} believe that the blood of the First Returned flows in the veins of the royal Idrian line.)

So why not bring this up in the book? Well, I learned in \textit{Elantris} that it’s easy to overtwist an ending by having too many reveals. This is a very small point, and there is good rationalization for why Treledees doesn’t let on what he knows. So I felt it was better to let the story stand as is, without delving into this.

Of course, there is a hint in the text about it—or at least a question. If they depended only on a Returned child taking Susebron’s place, then why were they worried about Siri having sex with Susebron? They didn’t need her to sleep with him unless they expected that sex to do something.

I’m sorry to leave this issue a mystery, and I’m even more sorry to not explain how Susebron can give away his Breaths. It’s not important to this book, and so I felt that having Treledees give the explanation here would just bog things down. I’d rather wait until a sequel, where I detail the magic system in a more complete form, to give you these explanations.

That leaves us with the cliché of someone who \textit{almost} passes on information, then dies. As I said, I am sorry to do this. I nearly didn’t put it in, but I felt it very important to include something that let you know that the priests \textit{did} have a way to get those Breaths.

Note that Treledees is not lying about letting Susebron live out his life with Siri in peace. They have allowed previous God Kings to do that, once they had a successor in place.

\subsection*{The Priests Sacrifice Themselves}

As I said, one of the reversals for this book is a reversal of my own books, where priests have traditionally been the bad guys. Here, Treledees and his people throw their lives away in an attempt to save Susebron. They’re zealous; I would say too zealous. But they’re good men, trying their best to serve their god. They go to their graves in that service.

\subsection*{Vivenna Sneaks (Poorly) into the Court of Gods}

We also show Vivenna being very proactive. Both of the sisters are having a much larger effect on events here than their male counterparts. There are a few interesting things to note.

First is Nightblood’s mention of Yesteel. I believe this is the first mention of him in the book. If you’ve been paying attention, you probably realized that there was one person missing out of the Five Scholars. Vasher, Denth, Arsteel, Shashara . . . and this guy. You’ll see him in the sequel. (And yes, he’s much better at sneaking than Vasher or Vivenna.)

Another note here is that Nightblood can sense where Vasher is. This is because Nightblood has ingested and fed off Vasher’s Breaths in the past. When he does that, it connects him to someone. It’s also, by the way, one of the secrets as to why Vasher doesn’t get sick when holding Nightblood, even though he’s a good person. It’s not simply familiarity (though that is part of it). Nightblood has a built-in test. If he feeds off you and you survive, then you become somewhat immune to his powers.

\subsection*{Vivenna Throws Nightblood at the Soldiers}

These men in soldier uniforms, as hinted at by how they react to Nightblood, are just a bunch of Denth’s mercenaries wearing uniforms to hide them. The guards at the front gates, however, are actual court guards. They don’t know that insurgents are now in control of the palace; they’re confused and are taking orders from Bluefingers, whom they see as someone with respect and authority.

The priests of the various gods are not so accommodating. There’s mass chaos among them, though many parts of the city don’t even know something strange is going on. The tunnels out of the Court of Gods are clogged with priests getting their various deities out of danger, which is why Bluefingers is slightly frustrated in the Siri scenes. He can’t get the God King out to the boat he has waiting. (He wants to keep him as a prisoner. Executing him as he outlines to Siri is a backup plan, one he decides to implement.)

\subsection*{Lightsong Notices the Pahn Kahl are Imitating Priests}

If you’re still confused about this, most of the priests you’ve seen in the Lightsong sections are Pahn Kahl scribes imitating priests to increase confusion. The skin tone is the clue, and Lightsong noticed it a chapter or two back, but couldn’t figure out what exactly was bothering him.

The previous chapters of the book—everything before the evening of Lightsong’s infiltration of the tunnels—never included Pahn Kahl imitating priests. We’ve only seen them a couple of places, mostly in the Lightsong sections here.

\subsection*{Bluefingers Explains That He Has to Execute Siri}

Bluefingers is right when he says that there’s a good chance Idris will do better in the war than everyone assumes. Of course, the main reason they’ll do better is because of how the Lifeless were launched without support or planning.

If this war were allowed to progress, Idris would be able to draw allies from across the mountains (as I mentioned earlier), and Yesteel’s ability to create swords like Nightblood would end with T’Telir falling and then the entire world being cast into chaos and destruction.



