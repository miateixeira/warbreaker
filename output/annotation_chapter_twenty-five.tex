\section{Annotation Chapter Twenty-Five}

\subsection*{Vivenna and the Mercenaries Attack the Salt Merchant’s Shop}

A very quick and efficient way to hurt the future Hallandren war. Denth doesn’t mention that there are ways of preserving meat (drying and smoking) that don’t require salt—but even in most jerky methods, one uses a brine solution, so his argument is reasonably satisfactory.

Vivenna notices the Tears of Edgli here, the flowers that drive Hallandren wealth and trade. I added these in an early draft, as I realized that there needed to be a cheap, easy source for all of those dyes the Hallandren use. (This was pointed out to me by my friend Jeff Creer, I believe.) The Tears offer something else as well—a reason for the wealth of the people. In early times, dye trades were extremely lucrative, and being able to control a method by which unusual dyes could be created would have been a very good basis for an economy.

I also like what it does for the flavor of Hallandren as a whole. This story happens in the place that is, in most fantasy books, far away. A lot of fantasy novels like to make their setting someplace akin to rural England, and they’ll talk of distant countries that have exotic spices, dyes, and trade goods.

Well, in this world, Hallandren is that place. It’s at the other end of the silk road, so to speak.

\subsection*{Vivenna Talks to Jewels about Religion}

I’m very conscious of the fact that all of my major viewpoint characters in this book—Lightsong included—don’t believe in the Hallandren religion. That worries me because the book presents a very one-sided view of their beliefs.

Religion isn’t a simple thing. In my books so far, I fear that I’ve presented the religions in a far too one-sided way. Hrathen with his Shu-Dereth, the Lord Ruler and his religion—these were not the types of religions that are very enticing to readers. The characters, even those viewpoint characters who followed the religions, didn’t present them very well. (And, in truth, the Lord Ruler’s religion—the Steel Ministry—was a pretty despicable religion.)

In this book, I wanted to present several different viable religions. There is something to be said for Austrism, with its goodly monks and teachings on humility through the Five Visions. But it’s a very superstitious and xenophobic religion at the same time, and it is very biased against the magic of the world. The Hallandren religion has more going for it than the characters would like to accept.

So, even though most readers might consider this a throwaway scene between Vivenna and Jewels, is a very important one to me. It is the place where we get to see a follower of the Iridescent Tones really stand up for what she believes. Vivenna deserves to be smacked down here, I think.

\subsection*{Vivenna Talks to Denth, and Considers Her Faith}

Vivenna’s line here—to believe is to be arrogant—is something I’ve thought about a lot myself. How do you believe that you’re right, yet also not be dismissive of others or arrogant about it?

This applies to more than just religion. It bothers me that in things like religion or politics, our natural inclination as human beings is to assume the worst about the other guy. If you look at the recent political elections in the United States, it seems that the other side—whichever side—can never do anything right. There is no candidate that the Republicans could have chosen who the Democrats wouldn’t have dismissed completely, and vice versa.

Isn’t it possible for you to think that you’re right without deciding that any who believe differently are stupid and corrupt?

I believe that my religion is true. And, by the definition of that religion, it means that everyone else is wrong. And yet my religion teaches me to be humble. I think there’s a way to do that and hold to your belief, but it seems to require more effort than a lot of people are willing to make.



