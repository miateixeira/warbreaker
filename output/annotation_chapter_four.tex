\section{Annotation Chapter Four}

\subsection*{Naming in This Book}

The names in this novel, particularly in Hallandren and Idris, follow the concept of repeated consonant sounds.



So, instead, in \textit{Mistborn} I chose names that were much easier to say, and gave everyone a simple nickname. When it came time for \textit{Warbreaker}, I wanted to try something else, to take a step back toward distinctiveness in the language, but not go as far as I had in \textit{Elantris}.

I’ve long toyed with using double consonants as a naming structure. I played with a lot of different ways of writing these. I could either use the letters doubled up, with no break (\textit{Ttelir}). I could slip a vowel in the middle and hope people pronounced it as a schwa sound (\textit{Tetelir}). Or I could use the fantasy standard of an apostrophe (\textit{T’telir}).

In the end, I decided to go with all three. I felt that writing all the names after one of the ways would look repetitive and annoying. By using all three, I could have variety, yet also have a theme. So, you have doubles in names like Llarimar. You have inserted vowels like in Vivenna. And you have apostrophes like in T’Telir.

I think it turned out well. Some members of my writing group complained about fantasy novels and their overuse of apostrophes in names. My answer: Tough. Just because English doesn’t like to do it doesn’t mean we have to eschew it in other languages. I like the way T’Telir looks with an apostrophe, and the way people will say it. So it stays. <img draggable="false" role="img" class="emoji" alt="😉" src="https://s.w.org/images/core/emoji/14.0.0/svg/1f609.svg">

\subsection*{Siri Approaches T’Telir}

And we finally get to see T’Telir. I’m still a tad bothered that it’s chapter four before we get to see the city. I worry that people will read the book and have trouble getting grounded in it, since we’ve now had five viewpoints across five chapters and have been in a lot of different locations.

However, I think that the groundwork in the first four chapters is needed to make the book work. I just couldn’t figure out a way to cut it all out and still have things work. Perhaps (just perhaps) I could have moved the Vasher prologue into the middle and made it a regular chapter, then moved the original Siri/Dedelin chapter to a prologue. Then, with the decision to send Siri into the city made, I could have jumped straight to this one. However, we’d have lost too much in that. Doing it this way isn’t perfect either, but I think it’s still the best way the book could have been done.



