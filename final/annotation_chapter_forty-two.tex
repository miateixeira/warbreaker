\section{Annotation Chapter Forty-Two}

\subsection*{Lightsong Irritates Allmother’s Priestesses}

I believe I mentioned earlier why the squirrel is more capable than Llarimar expected it to be. The quality of the Commands (meaning the skill of the Awakener) is very important to Lifeless, particularly those who might not have been very skilled in life to begin with.

In the original draft, I wasn’t sure what kind of person I was going to make Allmother. I hadn’t planned for her and Lightsong to have any kind of history together; these are just connections I worked into it as I wrote. (Along with his relationship with Calmseer.) He needed something to intertwine him better with the court, and so as I was drafting, these things kind of just fit together. Sometimes readers ask me what I plan and what I don’t. Well, the honest truth is that it’s hard to look at a book and give clear guidelines on what was planned and what was developed during the process of writing. In this case, Allmother as a character was done completely on the fly.

Of course, once she was developed, I went back in the next draft and built in some references to her in the Lightsong sections so that I could hint at previous interaction.

\subsection*{Lightsong Meets Allmother}

This was a tough scene to get right. The trick is, I knew by this point that I wanted Allmother to be one of those who disliked Lightsong. She thinks that he’s a useless god, and she isn’t one of those who saw hidden depth in him.

I also knew that I wanted to give a twist here by having Lightsong offer up his Commands and give himself a way out, so to speak. What he does here is rather honorable. He knows that Allmother is a clever woman and perhaps one of the only gods capable of going toe-to-toe with Blushweaver. By giving her his Commands, he does a good job of countering Blushweaver without having to resist her.

But he couldn’t get away with it. He \textit{had} to stay in the middle of it all, for the good of the story and for the good of him as a character. So the question became, “Why in the world would Allmother give him her Commands?”

The prophetic dreams came to my rescue a couple of times in this book. I know that they’re cheating slightly, but since I’ve built them into the story, I might as well use them. Having her having dreamed of his arrival gives me the out for why she’d do something as crazy as give up her Commands. I think her visions, mixed with the knowledge that Calmseer trusted Lightsong, would be enough to push her over the edge.

Oh, and that \textit{was} an Idrian kneeling before Allmother—the guy with only one leg. He’s not the only one to have converted. A lot of them have, in fact, when confronted by gods you can visit and see. The other Idrians call them “scrapes,” an epithet that refers to a scratch on a person’s arm revealing the colorful blood underneath.

Once, all the gods and goddesses did as Allmother now does—if someone came to them with a petition, they tried their best to find a way to help them without giving up their Breath. The modern gods consider this far too much trouble, and it has fallen out of practice. Everyone says that the gods of this day are weaker than the previous ones. They’re right, though \textit{weaker} isn’t the right word. They’re just not as high quality a group of people, partially because of their traditions and expectations.



