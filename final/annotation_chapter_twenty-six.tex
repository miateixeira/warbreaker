\section{Annotation Chapter Twenty-Six}

\subsection*{Lightsong Gets Up Early, Excited}

Let this be a lesson to aspiring writers. People’s reactions to these Lightsong sections—where he goes to investigate the murder—are proof of a long-standing rule of writing. Characters who \textit{do} things are more interesting than those who don’t.

Now, this may seem obvious to you. But let me assure you, when you start to write, you will often be tempted to include viewpoint characters with internal conflicts. Many times, poorly written, these conflicts result in the character being inactive. They can’t decide about things, or they’re a coward, or they’re depressed or indifferent. All of these things are flaws the characters are going to grow out of during the story, and you’re very tempted to build them into the character as a way of giving the character more growth and things to overcome.

That’s not a bad instinct, but it’s much more difficult to pull off than you think. The problem is that a lot of characters like that don’t really \textit{do} anything for the first part of the story. They’re reactive, and they don’t care about the plot, which makes the reader not care about the plot.

Until you’ve practiced a while, might I suggest that you stick with characters who are passionate about what they’re doing and who try consistently to achieve their goals? Give them different internal conflicts, things that don’t keep them from acting. Learning to write a good book is tough enough without tackling an inactive character in your first few stories.

\subsection*{Lightsong Sees the Painting of the Red Battle}

This is our first major clue (though a subtle one at the same time) that there might be something to the religion of the Iridescent Tones. Lightsong \textit{does} see something in this painting that an ordinary person wouldn’t be able to. A well-crafted piece of art, made by a person channeling the Tones and connected to them via Breath, can speak to a Returned. Now, in this case, it doesn’t work quite like Llarimar says it does—Lightsong doesn’t actually prophesy about the black sword in the way the priest thinks. In other words, Lightsong isn’t prophesying that he’ll see the Black Sword (Nightblood) in the day’s activities.

Instead, Lightsong is seeing an image of a previous war, which is prophetic in that another Manywar is brewing—and in both cases, Nightblood will be important to the outcome of the battle.

To continue, I go into spoilers.



<p></p>
<p>The person Lightsong sees in the abstract painting is Shashara, Denth’s sister, one of the Five Scholars and a Returned also known as Glorysinger by the Cult of the Returned. She is seen here in Lightsong’s vision as she’s drawing Nightblood at the battle of Twilight Falls. It’s the only time the sword was drawn in battle, and Vasher was horrified by the result.</p>
<p>It’s because of her insistence on using the sword in battle, and on giving away the secret to creating more, that Vasher and she fought. He ended up killing her with Nightblood, which they’d created together during the days they were in love—he married her a short time before their falling out. That marriage ended with him slaying his own wife to keep her from creating more abominations like Nightblood and loosing them upon the world.</p>
<p>Nightblood is part of a much larger story in this world. He’s dropped casually into this particular book, more as a side note than a real focus of what’s going on, but his own role in the world is much, much larger than his supporting part here would indicate.</p>
<p>Also, just so you know, the second person who snuck into the palace was Denth—tailing Vasher, trying to decide what he was up to. Bluefingers let Denth know that Vasher would try to enter, but warned him \textit{not} to attack the man. Not while it could expose Denth and possibly Bluefingers.</p>
<p>Denth would have attacked anyway, if he’d decided he had a good opportunity. But he didn’t, and he decided it was better to watch.</p>
<p>And yes, he’d hidden away his Breath so that Vasher couldn’t sense him following.</p>
<p></p>

\subsection*{Lightsong Inspects the Murder Scene Again}

The interesting thing about this scene is that it reveals almost nothing about what happened. At least, it doesn’t reveal anything to the readers.

However, it reveals a whole lot about Lightsong as a character. I waited until he’d been established before starting to bring up questions like the ones in this chapter, where I begin to dig deeply into who he was before he died. In a way, he’s not investigating the murder so much as he is investigating himself—and that’s why the scene works, even though we know the information about the murders he reveals. (Though we don’t know who that second person was. Unless you read the spoiler above, of course.)



