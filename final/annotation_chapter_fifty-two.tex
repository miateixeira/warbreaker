% \section{Annotation Chapter Fifty-Two}
\section*{Annotations}

\subsection*{Lightsong Gathers His Finery in His Palace}

Is there a lesson in all of this, as Lightsong accuses Llarimar of teaching? Perhaps. The value of something is indeed in how you treat it. All of the riches in the world could be piled in one place, and they would be unimportant unless you ascribed value to them. I think this is one of the reasons Lightsong has been so flippant all of his life as a god. Before Returning, the things he valued were far more intangible. People, his life’s work, intellectual freedom—all these things were taken from him, then replaced with gold and baubles. To him, they’re inferior replacements, and he can’t help but chafe—unknowingly—at his confines.

I wanted a chance for Llarimar to take off his hat and be just a friend for a time. His belief system is complex, since he knew Lightsong ahead of time. He sees the divine mantle, but he also sees the man.

\textcolor{red}{
The man who was his younger brother, the daring and gregarious one, the one who didn’t always do what he was supposed to. One of the subtle twists of this book is that Llarimar and Lightsong’s relationship is supposed to be a parallel of Vivenna and Siri’s. They were closer than those two ever were, and as both were middle-aged, they interacted differently. But Lightsong (or Stennimar as he was then known) never married. He liked traveling too much, and enjoyed his bachelor lifestyle. Llarimar was the one who always did what he should, but he also always admired his brother for his sense of adventure, his proactiveness, and his simple kindness toward other people.
}

\subsection*{Siri Is Locked Up, and Her Guards Change}

Just a quick reminder here of what’s going on with Siri. I worry about her next few sequences looking too “damsel in distress.” I tried to counteract this in several places, which I’ll mention. Still, I had a problem here. Once things turn to combat and fighting, there is very little that Siri can do. She’s not Vin—she can’t approach things the same way.

However, since Elend got to play damsel in distress fairly often in the \textit{Mistborn} books, I think I’ve earned the right to put a female protagonist into that role here. It’s appropriate to the plot, and I don’t think it could have worked any other way.

\subsection*{Lightsong Sneaks into Mercystar’s Palace}

Here’s the other big place where I cheated just a tad and added Lightsong’s dreams of the tunnels and the moon as a reason to get him into the right place at the right time. I added this in a later draft; originally, this was one of my big personal problems with the book: the fact that Lightsong got into just the right place at just the right time. It was just too coincidental, and it always bugged me.

I wasn’t paying attention to the tools I’d given myself (as I think I mentioned earlier). If I’m going to go to all this trouble to build a magic system that uses prophecy as a major component of its religion, then I might as well use a few of those prophecies as small plot points. I didn’t want them to solve any major problems, but letting Lightsong dream of where he has to be brings nice closure to the entire “What’s in those tunnels?” plot while at the same time playing into his quest to determine if he really is a god or not.

By the way, the grate that Lightsong closes on the tunnel behind them . . . well, it didn’t do any good. There’s a lever and pulley on the other side, in the room beneath Mercystar’s palace—and the locking mechanism is there too. The grate is there to keep people out of her palace, installed by her priests to keep unsavory elements (if there are any) from sneaking in through the tunnels. Vasher had to pull this very grate up before he could sneak into the tunnels himself. Mercystar’s priests don’t follow because they don’t care that Lightsong snuck in and down; they just want to guard their goddess. So they arrange troops up above, waiting for Lightsong to return.

\textcolor{red}{
A little history on the tunnel complex. It was begun many years ago by some gods who wanted to have a secret way to get between each other’s palaces. They had to get funds for that, however, and so the God King’s steward before Bluefingers (who was also Pahn Kahl) heard of it and was intrigued. Even back then, plans were being laid. He realized that a secret way to get in and out of the Court of Gods would be very useful, so he began to hint to the priests he knew that they might want tunnels themselves. They were very useful in arranging clandestine meetings of the political type, and so some priests got their god to agree to tunnels. They didn’t realize that they were playing into the Pahn Kahl steward’s plans.
}

\textcolor{red}{
Bluefingers continued this work, carefully diverting funds from the projects secretly, then using the digging to mask digging in other places as well. Few priests paid attention to the workers down there, and within several decades, the workers could enter and leave even without passing through the court above. The priests liked having secret ways to enter the court themselves, though most had safety features—like the grate at Mercystar’s place—installed. They saw no danger in the tunnels; they’ve always been too confident of their safety in T’Telir. They didn’t realize the extent to which Bluefingers would eventually be able to manipulate the tunnels to bring in mercenaries and Pahn Kahl Awakeners to slowly begin breaking the Lifeless soldiers.
}

\subsection*{Lightsong Attacks}

And we discover that Lightsong is no good with the sword. I toyed with making him able to use it, but I felt it was too much of a cut corner. Knowing who he was before he died, he’d not have needed to know the sword. Beyond that, I felt it would have been too expected. Lightsong himself built it up so much that I feel it would have been a boring plot twist to have him able to use the sword. Beyond that, it would have been just too convenient.

Reversals. I wanted to reverse what you assume about him, and to reverse how this scene would have probably played out in a lot of fantasy stories. Once again, I’m not reversing just \textit{to} reverse. I’m reversing because it’s appropriate for the characters, setting, and plot—and then finally because it’s more interesting this way.



