\section*{Annotations}

\subsection*{Lightsong Plays Terachin With Three Other Gods}

This is the newest scene in the book, added in the last revision before the novel went to copyedit. I added it for two reasons. My editor wanted to see another chapter between the previous Lightsong chapter and the next one. He felt that the god made up his mind to help Blushweaver too easily, and wanted to spend more time with Lightsong mulling over the decision.

I reacted quickly to the suggestion, as I’d been wanting to show Lightsong interacting with some of the other gods. It’s sometimes too easy for me to build my books around a small core cast and rarely involve any others, and I have to force myself to include more characters to round things out. This book had a distinct lack of scenes with “ordinary” gods. We got to see a lot of the exceptions, but never the run-of-the-mill divinities who make up the ranks.

I wanted to show how they schemed and how they acted. Putting Lightsong with three of them here helps the book quite a bit, I think. It makes the world feel more real and helps his character by providing contrast.

The game is something I developed in order to make this scene work. I wanted a divine game—one that wouldn’t require too much effort, would require a lot of preparation and extravagance, but would still qualify as a sport. So, we have a game where the gods can sit on a balcony attended by a fleet of servants and scribes tallying their throws.

When my editor read the scene, he loved it instantly. He called to tell me it was one of his favorites in the book, partially because of some particularly good Lightsong quips. He says that he fully expects some Sanderson book readers to develop the rules for the game someday, then play it at a con.



\subsection*{Vivenna Goes to Two Restaurants to Meet with Crime Lords}

Can you tell that I hate seafood? How does anyone eat that stuff? I mean, honestly. I’ve been forced to choke down raw clams before, and it was just about one of the most traumatic events in my life.

\subsection*{Only Potential Heirs of Idris Have Royal Locks}

This is true. It’s not a matter of genetics, but lineage. That’s a subtle distinction. Only the children of the person who ends up inheriting will have the Royal Locks. (Though there are a couple of notable exceptions to this, they won’t show up in this book, as it will take another novel to explain why and how the Royal Locks really work. If I ever write a sequel, that should be in it.)

This factoid about the Royal Locks should be one of several hints about the lineage of the Idrian crown. There is something odd about their heritage.



\subsection*{Lightsong Plays Terachin With Three Other Gods}

This is the newest scene in the book, added in the last revision before the novel went to copyedit. I added it for two reasons. My editor wanted to see another chapter between the previous Lightsong chapter and the next one. He felt that the god made up his mind to help Blushweaver too easily, and wanted to spend more time with Lightsong mulling over the decision.

I reacted quickly to the suggestion, as I’d been wanting to show Lightsong interacting with some of the other gods. It’s sometimes too easy for me to build my books around a small core cast and rarely involve any others, and I have to force myself to include more characters to round things out. This book had a distinct lack of scenes with “ordinary” gods. We got to see a lot of the exceptions, but never the run-of-the-mill divinities who make up the ranks.

I wanted to show how they schemed and how they acted. Putting Lightsong with three of them here helps the book quite a bit, I think. It makes the world feel more real and helps his character by providing contrast.

The game is something I developed in order to make this scene work. I wanted a divine game—one that wouldn’t require too much effort, would require a lot of preparation and extravagance, but would still qualify as a sport. So, we have a game where the gods can sit on a balcony attended by a fleet of servants and scribes tallying their throws.

When my editor read the scene, he loved it instantly. He called to tell me it was one of his favorites in the book, partially because of some particularly good Lightsong quips. He says that he fully expects some Sanderson book readers to develop the rules for the game someday, then play it at a con.



\subsection*{Vivenna Goes to Two Restaurants to Meet with Crime Lords}

Can you tell that I hate seafood? How does anyone eat that stuff? I mean, honestly. I’ve been forced to choke down raw clams before, and it was just about one of the most traumatic events in my life.

\subsection*{Only Potential Heirs of Idris Have Royal Locks}

This is true. It’s not a matter of genetics, but lineage. That’s a subtle distinction. Only the children of the person who ends up inheriting will have the Royal Locks. (Though there are a couple of notable exceptions to this, they won’t show up in this book, as it will take another novel to explain why and how the Royal Locks really work. If I ever write a sequel, that should be in it.)

This factoid about the Royal Locks should be one of several hints about the lineage of the Idrian crown. There is something odd about their heritage.

\subsection*{Clod the Lifeless}

Yes, Clod is Arsteel, in case you were wondering. After Vasher killed him, Denth’s team decided to have him made into a Lifeless. Partially because Denth was curious if it was possible, and partially because Arsteel was such a capable warrior that they knew he’d make for an excellently skilled Lifeless. It isn’t as good as having Arsteel himself, of course, but Clod is probably the greatest Lifeless swordfighter in existence right now in the entire world.

Another tidbit that never comes up is that Jewels was in love with Arsteel, which is the primary reason she joined Denth’s team in the first place. Arsteel joined it because he wanted to try to redeem Denth; he felt that a reconciliation between Denth and Vasher was possible, and as a peacemaker, he thought he might be able to make it happen. As for why Vasher killed him . . . well, I’m afraid that’s another story that will have to wait for the sequel.

Jewels is still in love with him. And yes, she still sleeps with him on occasion. And yes, she’s a little bit unhinged emotionally and mentally because of his death.

\subsection*{Vivenna at the Safe House}

Vivenna is right about what happens to a person when they lose their Breath. It \textit{is} a part of your soul, and without one, you are more prone to depression, you get sick much more easily, and you’re generally more irritable.

I included this mention here because I’m betting that most people who read the book side with Denth and assume he’s right when he talks about these things. But don’t be too judgmental about the Idrians—yes, they’re biased, but the Hallandren are too in a lot of ways. It’s not as simple as one side always being right and the other wrong. In this case, the Idrian teachings are correct, and most Hallandren are looking for justifications when they say that giving up one’s Breath isn’t all that damaging to them.

\subsection*{Denth’s Speed}

Yes, Denth is inhumanly fast. He’s a Returned, after all, and has all of the physical enhancements that come with that. Even when he’s chosen not to manifest most of them, he’s still got an edge, just like Vasher does.

How do they hide that they’re Returned? Well, it comes down to mastery of their ability to change their appearance. They can’t shape-shift entirely; they can just alter some things about their appearance. They can change their weight, their hair color, and things like that at will. Vasher doesn’t do this often, but Denth has been known to use it as a disguise. The problem, after you do this once and someone realizes it, your nature becomes very suspect.

They have learned to suppress their divine Breath. This allows them to hide, but they must be careful never to give away all of their Breath. Denth has been a Drab before—he’s not completely lying—but never for longer than a few days. And his divine Breath is always there, suppressed. So he doesn’t know what it’s like to be a true Drab, which is why in this chapter he says he doesn’t think it changes you that much. He’s never felt it.

\subsection*{Tonk Fah Wants to Be the Mean One}

Tonk Fah is a sociopath. He doesn’t feel an emotional connection to other people, nor does he feel their pain when he hurts them. He tortures and kills animals when it strikes his fancy. There’s a dead parrot in the basement of the safe house, which is why Denth keeps Vivenna from going down there. There aren’t any bodies of Idrian soldiers down there currently, though Denth has had a few of them killed already. The fact that he has people watching their house, plus Vivenna’s mention of her father’s soldiers checking Lemex’s house first, are tiny clues. They do indeed go there first, and Denth has his people there watching. That’s how he catches the Idrian soldiers.

By this point in the story, he’s killed about three people who have come looking for Vivenna. The death count will eventually reach several dozen.




