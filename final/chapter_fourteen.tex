% \section{Chapter Fourteen}
\chapter{}

“It’s raining,” Lightsong noted.

“Very astute, Your Grace,” Llarimar said, walking beside his god.

“I’m not fond of rain.”

“So you have often noted, Your Grace.”

“I’m a god,” Lightsong said. “Shouldn’t I have power over the weather? How can it rain if I don’t want it to?”

“There are currently twenty-five gods in the court, Your Grace. Perhaps there are more who desire rain than those who don’t.”

Lightsong’s robes of gold and red rustled as he walked. The grass was cool and damp beneath his sandaled toes, but a group of servants carried a wide canopy over him. Rain fell softly on the cloth. In T’Telir, rainfalls were common, but they were never very strong.

Lightsong would have liked to have seen a true rainstorm, like people said occurred out in the jungles. “I’ll take a poll then,” Lightsong said. “Of the other gods. See how many of them wanted it to rain today.”

“If you wish, Your Grace,” Llarimar said. “It won’t prove much.”

“It’ll prove whose fault this is,” Lightsong said. “And~.~.~. if it turns out that most of us want it to stop raining, perhaps that will start a theological crisis.”

Llarimar, of course, didn’t seem bothered by the concept of a god trying to undermine his own religion. “Your Grace,” he said, “our doctrine is~\textit{quite}~sound, I assure you.”

“And if the gods don’t want it to rain, yet it still does?”

“Would you like it to be sunny all the time, Your Grace?”

Lightsong shrugged. “Sure.”

“And the farmers?” Llarimar said. “Their crops would die without the rain.”

“It can rain on the crops,” Lightsong said, “just not in the city. A few selective weather patterns shouldn’t be too much for a god to accomplish.”

“The people need water to drink, Your Grace,” Llarimar said. “The streets need to be washed clean. And what of the plants in the city? The beautiful trees—even this grass that you enjoy walking across—would die if the rain did not fall.”

“Well,” Lightsong said, “I could just~\textit{will}~them to continue living.”

“And that is what you do, Your Grace,” Llarimar said. “Your soul knows that rain is best for the city, and so it rains. Despite what your consciousness thinks.”

Lightsong frowned. “By that argument, you could claim that~\textit{anyone}~was a god, Llarimar.”

“Not just~\textit{anyone}~comes back from the dead, Your Grace. Nor do they have the power to heal the sick, and they certainly don’t have your ability to foresee the future.”

\textit{Good points, those,}~Lightsong thought as they approached the arena. The large, circular structure was at the back of the Court of Gods, outside of the ring of palaces that surrounded the courtyard. Lightsong’s entourage moved inside—red canopy still held above him—and entered the sand-covered arena yard. Then they moved up a ramp toward the seating area.

The arena had four rows of seats for ordinary people—stone benches, accommodating T’Telir citizens who were favored, lucky, or rich enough to get themselves into an assembly session. The upper reaches of the arena were reserved for the Returned. Here—close enough to hear what was said on the arena floor, yet far enough back to remain stately—were the boxes. Ornately carved in stone, they were large enough to hold a god’s entire entourage.

Lightsong could see that several of his colleagues had arrived, marked by the colorful canopies that sat above their boxes. Lifeblesser was there, as was Mercystar. They passed by the empty box usually reserved for Lightsong and made their way around the ring and approached a box topped by a green pavilion. Blushweaver lounged inside. Her green and silver dress was lavish and re vealing, as always. Despite its rich trim and embroidery, it was little more than a long swath of cloth with a hole in the center for her head and some ties. That left it completely open on both sides from shoulder to calf, and Blushweaver’s thighs curved out lusciously on either side. She sat up, smiling.

Lightsong took a deep breath. Blushweaver always treated him kindly and she certainly did have a high opinion of him, but he felt like he had to be on guard at all times when he was around her. A man could be taken in by a woman such as she.

Taken in, then never released.

“Lightsong, dear,” she said, smiling more deeply as Lightsong’s servants scuttled forward, setting up his chair, footrest, and snack table.

“Blushweaver,” Lightsong replied. “My high priest tells me that you’re to blame for this dreary weather.”

Blushweaver raised an eyebrow, and to the side—standing with the other priests—Llarimar flushed. “I like the rain,” Blushweaver finally said, lounging back on her couch. “It’s~.~.~. different. I like things that are different.”

“Then you should be thoroughly bored by me, my dear,” Lightsong said, seating himself and taking a handful of grapes—already peeled—from the bowl on his snack table.

“Bored?” Blushweaver asked.

“I strive for nothing if not mediocrity, and mediocrity is~\textit{hardly}~different. In fact, I should say that it’s highly in fashion in court these days.”

“You shouldn’t say such things,” Blushweaver said. “The people might start to believe you.”

“You mistake me. That’s why I say them. I figure if I can’t do properly deific miracles like control the weather, then I might as well settle for the lesser miracle of being the one who tells the truth.”

“Hum,” she replied, stretching back, the tips of her fingers wiggling as she sighed in contentment. “Our priests say that the purpose of the gods is not to play with weather or prevent disasters, but to provide visions and ser vice to the people. Perhaps this attitude of yours is not the best way to see to their interests.”

“You’re right, of course,” Lightsong said. “I’ve just had a revelation. Mediocrity~\textit{isn’t}~the best way to serve our people.”

“What is, then?”

“Medium rare on a bed of sweet-potato medallions,” he said, popping a grape in his mouth. “With a slight garnish of garlic and a light white wine sauce.”

“You’re incorrigible,” she said, finishing her stretch.

“I am what the universe made me to be, my dear.”

“You bow before the whims of the universe, then?”

“What else would I do?”

“Fight it,” Blushweaver said. She narrowed her eyes, absently reaching to take one of the grapes from Lightsong’s hand. “Fight with everything, force the universe to bow to you instead.”

“That’s a charming concept, Blushweaver. But I believe that the universe and I are in slightly different weight categories.”

“I think you’re wrong.”

“Are you saying I’m fat?”

She regarded him with a flat glance. “I’m saying that you needn’t be so humble, Lightsong. You’re a~\textit{god}.”

“A god who can’t even make it stop raining.”

“\textit{I}~want it to storm and tempest. Maybe this drizzle is the compromise between us.”

Lightsong popped another grape in his mouth, squishing it between his teeth, feeling the sweet juice leak onto his palate. He thought for a moment, chewing. “Blushweaver, dear,” he finally said. “Is there some kind of subtext to our current conversation? Because, as you might know, I am absolutely~\textit{terrible}~with subtext. It gives me a headache.”

“You can’t get headaches,” Blushweaver said.

“Well I can’t get subtext either. Far too subtle for me. It takes effort to understand, and effort is—unfortunately—against my religion.”

Blushweaver raised an eyebrow. “A new tenet for those who worship you?”

“Oh, not that religion,” Lightsong said. “I’m secretly a worshipper of Austre. His is such a delightfully blunt theology—black, white, no bothering with complications. Faith without any bothersome thinking.”

Blushweaver stole another grape. “You just don’t know Austrism well enough. It’s complex. If you’re looking for something~\textit{really}~simple, you should try the Pahn Kahl faith.”

Lightsong frowned. “Don’t they just worship the Returned, like the rest of us?”

“No. They have their own religion.”

“But everyone knows the Pahn Kahl are practically Hallandren.”

Blushweaver shrugged, watching the stadium floor below.

“And how exactly did we get onto this tangent, anyway?” Lightsong said. “I swear, my dear. Sometimes our conversations remind me of a broken sword.”

She raised an eyebrow.

“Sharp as hell,” Lightsong said, “but lacking a point.”

Blushweaver snorted quietly. “You’re the one who asked to meet with me, Lightsong.”

“Yes, but we both know that you wanted me to. What are you planning, Blushweaver?”

Blushweaver rolled her grape between her fingers. “Wait,” she said.

Lightsong sighed, waving for a servant to bring him some nuts. One placed a bowl on the table; then another came forward and began to crack them for him. “First you imply that I should join with you, now you won’t tell me what you want me to do? I swear, woman. Someday, your ridiculous sense of drama is going to cause cataclysmic problems—like, for instance, boredom in your companions.”

“It’s not drama,” she said. “It’s respect.” She nodded directly across the arena, where the God King’s box still stood empty, golden throne sitting on a pedestal above the box itself.

“Ah. Feeling patriotic today, are we?”

“It’s more that I’m curious.”

“About?”

“Her.”

“The queen?”

Blushweaver gave him a flat stare. “Of course, her. Who else would I be speaking about?”

Lightsong counted off the days. It had been a week. “Huh,” he said to himself. “Her period of isolation is over, then?”

“You really should pay more attention, Lightsong.”

He shrugged. “Time tends to pass you by more quickly when you take no notice of it, my dear. In that, it’s remarkably similar to most women I know.” With that, he accepted a handful of nuts, then settled back to wait.

\orn

Apparently, the people of T’Telir weren’t fond of carriages—not even to carry gods. Siri sat, somewhat bemused, as a group of servants carried her chair across the grass toward a large, circular structure at the back of the Court of Gods. It was raining. She didn’t care. She’d been cooped up for far too long.

She turned, twisting in her chair, looking back over a group of serving women who carried her dress’s long golden train, keeping it off the wet grass. Around them all walked more women, who held a large canopy to shield Siri from the rain.

“Could you~.~.~. move that aside?” Siri asked. “Let the rain fall on me?”

The serving women glanced at one another.

“Just for a little bit,” Siri said. “I promise.”

The women shared frowns, but slowed, allowing Siri’s porters to pull ahead and expose her to the rain. She looked up, smiling as the drizzle fell on her face.~\textit{Seven days is}~far~\textit{too long to spend indoors,}~she decided. She basked for a long moment, enjoying the cool wetness on her skin and clothing. The grass looked inviting. She glanced back again. “I could walk, you know.”\textit{Feel my toes on those green blades.~.~.~.}

The serving women looked very, very uncomfortable about that concept.

“Or not,” Siri said, turning around as the women sped up, again covering the sky with their canopy. Walking was probably a bad idea, considering her dress’s long train. She’d eventually chosen a gown far more daring than anything she’d ever worn before. The neckline was a touch low, and it had no sleeves. It also had a curious design that covered the front of her legs with a short skirt, yet was floor-length in back. She’d picked it partially for the novelty, though she blushed every time she thought of how much leg it showed.

They soon arrived at the arena and her porters carried her up into it. Siri was interested to see that it had no ceiling and had a sand-covered floor. Just above the floor, a colorful group of people were gathering on ranks of benches. Though some of them carried umbrellas, many ignored the light rain, chatting amiably among themselves. Siri smiled at the crowd; a hundred different colors and as many different clothing styles were represented. It was good to see some variety again, even if that variety was somewhat garish.

Her porters carried her up to a large stone cleft built into the side of the building. Here, her women slid the canopy’s poles into holes in the stone, allowing it to stand freely to cover the entire box. Servants scuttled about, getting things ready, and her porters lowered her chair. She stood, frowning. She was finally free of the palace. And yet it appeared she was going to have to sit above everyone else. Even the other gods—who she assumed were in the other canopied boxes—were far away and separated from her by walls.

\textit{How is it that they can make me feel alone, even when surrounded by hundreds of people?}~She turned to one of her serving women. “The God King. Where is he?”

The woman gestured toward the other boxes like Siri’s.

“He’s in one of them?” Siri asked.

“No, Vessel,” the woman said, eyes downcast. “He will not arrive until the gods are all here.”

\textit{Ah,}~Siri thought.~\textit{Makes sense, I guess.}

She sat back in her chair as several servants prepared food. To the side, a minstrel began to play a flute, as if to drown out the sounds of the people below. She would rather have heard the people. Still, she decided not to let herself get into a bad mood. At least she was outside, and she could~\textit{see}~other people, even if she couldn’t interact with them. She smiled to herself, leaning forward, elbows on knees, as she studied the exotic colors below.

What was she to make of T’Telir people? They were just so remarkably diverse. Some had dark skin, which meant they were from the edges of the Hallandren kingdom. Others had yellow hair, or even strange hair colors—blue and green—that came, Siri assumed, from dyes.

All wore brilliant clothing, as if there were no other option. Ornate hats were popular, both on men and women. Clothing ranged from vests and shorts to long robes and gowns.~\textit{How much time must they spend shopping!}~It was difficult enough for her to choose what to wear, and she had only about a dozen choices each day—and no hats. After she’d refused the first few, the servants had stopped offering them.

Entourage after entourage arrived bearing a different set of colors—a hue and a metallic, usually. She counted the boxes. There was room for about fifty gods, but the court had only a couple of dozen. Twenty-five, wasn’t it? In each procession, she saw a figure standing taller than the others. Some—mostly the women—were carried on chairs or couches. The men generally walked, some wearing intricate robes, others wearing nothing more than sandals and skirt. Siri leaned forward, studying one god as he walked right by her box. His bare chest made her blush, but it let her see his well-muscled body and toned flesh.

He glanced at her, then nodded his head slightly in respect. His servants and priests bowed almost to the ground. The god passed on, having said nothing.

She sat back in her chair, shaking her head as one of the servants offered her food. There were still four or five gods left to arrive. Apparently, the Hallandren deities weren’t as punctual as Bluefingers’s schedule-keeping had led her to believe.

\orn

Vivenna stepped through the gates, passing into the Hallandren Court of Gods, which was dominated by a group of large palaces. She hesitated, and small groups of people passed through on either side of her, though there wasn’t much of a crowd.

Denth had been right; it had been easy for her to get into the court. The priests at the gate had waved Vivenna through without even asking her identity. They had even let Parlin pass, assuming him to be her attendant. She turned back, glancing at the priests in their blue robes. She could see bubbles of colorfulness around them, indications of their strong BioChroma.

She’d been tutored about this. The priests guarding the gates had enough Breath to get them to the First Heightening, the state at which a person gained the ability to distinguish levels of Breath in other people. Vivenna had it too. It wasn’t that auras or colors looked different to her. In fact, the ability to distinguish Breath was similar to the perfect pitch she had gained. Other people heard the same sounds she did, she just had the ability to pick them apart.

She saw how close a person had to get to one of the priests before the colors increased, and she saw exactly how much more colorful those hues became. This information let her know instinctively that each of the priests was of the First Heightening. Parlin had one Breath. The ordinary citizens, who had to present papers to gain entrance to the court, also each had only one Breath. She could tell how strong that Breath was, and if the person was sick or not.

The priests each had exactly fifty Breaths, as did the majority of the wealthier individuals entering through the gates. A fair number had at least two hundred Breaths, enough for the Second Heightening and the perfect pitch it granted. Only a couple had more Breaths than Vivenna, who had reached all the way to the Third Heightening and the perfect color perception it granted.

She turned away from her study of the crowd. She’d been tutored about the Heightenings, but she’d never expected to experience one firsthand. She felt dirty. Perverse. Particularly because the colors were just so~\textit{beautiful.}

Her tutors had explained how the court was composed of a wide circle of palaces, but they had not mentioned how each palace was so harmoniously balanced in color. Each was a work of art, utilizing subtle color gradients that normal people just wouldn’t be able to appreciate. These sat on a perfect, uniformly green lawn. It was trimmed carefully, and it was marred by neither road nor walkway. Vivenna stepped onto it, Parlin at her side, and she felt an urge to kick off her shoes and walk barefoot in the dew-moistened grass. That wouldn’t be appropriate at all, and she stifled the impulse.

The drizzle was finally starting to let up, and Parlin lowered the umbrella he’d bought to keep them both dry. “So, this is it,” he said, shaking off the umbrella. “The Court of Gods.”

Vivenna nodded.

“Good place to graze sheep.”

“I doubt that,” she said quietly.

Parlin frowned. “Goats, then?” he said finally.

Vivenna sighed, and they joined the small procession walking across the grass toward a large structure outside the circle of palaces. She’d been worried about standing out—after all, she still wore her simple Idrian dress, with its high neck, practical fabric, and muted colors. She was beginning to realize that there just~\textit{wasn’t}~a way to stand out in T’Telir.

The people around her wore such a stunning variety of costumes that she wondered who had the imagination to design them all. Some were as modest as Vivenna’s and others even had muted colors—though these were usually accented by bright scarves or hats. Modesty in both design and color was obviously unfashionable, but not nonexistent.

\textit{It’s all about drawing attention,}~she realized.~\textit{The whites and faded colors are a reaction against the bright colors. But because everyone tries so hard to look distinctive, nobody does!}

Feeling a little more secure, she glanced at Parlin, who seemed more at peace now that they were away from the larger crowds in the city below. “Interesting buildings,” he said. “The people wear so much color, but that palace is just one color. Wonder why that is.”

“It’s not one color. It’s many different shades of the same color.”

Parlin shrugged. “Red is red.”

How could she explain? Each red was different, like notes on a musical scale. The walls were of pure red. The roof tiles, side columns, and other ornamentations were of slightly different shades, each distinct and intentional. The columns, for instance, formed stepping fifths of color, harmonizing with the base tint of the walls.

It was like a symphony of hues. The building had obviously been constructed for a person who had achieved the Third Heightening, as only such a person would be able to see the ideal resonance. To others~.~.~. well, it was just a bunch of red.

They passed the red palace, approaching the arena. Entertainment was central to the lives of the Hallandren gods. After all, one couldn’t expect gods to do anything useful with their time. Often they were diverted in their palaces or on the courtyard lawn, but for particularly large events, there was the arena—which also served as the location of Hallandren legislative debates. Today, the priests would argue for the sport of their deities.

Vivenna and Parlin waited their turn as the people crowded around the arena entrance. Vivenna glanced toward another gateway, wondering why nobody used it. The answer was made manifest as a figure approached. He was surrounded by servants, some carrying a canopy. All were dressed in blue and silver, matching their leader, who stood a good head taller than the others. He gave off a BioChromatic aura such as Vivenna had never seen—though, admittedly, she’d been able to see them for only a few hours. His bubble of enhanced color was enormous; it extended nearly thirty feet. To her First Heightening senses, the god’s Breath registered as infinite. Immea sur able. For the first time, Vivenna could see that there~\textit{was}~something different about the Returned. They weren’t just Awakeners with more power; it was like they had only a single Breath, but that Breath was so im mensely powerful that it single-handedly propelled them to the upper Heightenings.

The god entered the arena through the open gateway. As she watched him, Vivenna’s sense of awe dissipated. There was an arrogance in this man’s posture, a dismissiveness to the way he entered freely while others waited their turn at an overcrowded entrance.

\textit{To keep him alive,}~Vivenna thought,~\textit{he has to absorb a person’s Breath each week.}

She’d let herself become too relaxed, and she felt her revulsion return. Color and beauty couldn’t cover up such enormous conceit, nor could it hide the sin of being a parasite living on the common people.

The god disappeared into the arena. Vivenna waited, thinking for a time about her own BioChroma and what it meant. She was completely shocked when a man beside her suddenly lifted off the ground.

The man rose into the air, lifted by his unusually long cloak. The cloth had stiffened, looking a little like a hand as it held the man up high so he could see over the crowd.~\textit{How does it do that?}~She’d been told that Breath could give life to objects, but what did “life” mean? It seemed as if the fibers in the cloak were taut, like muscles, but how did it lift something so much heavier than it was?

The man descended to the ground. He muttered something Vivenna couldn’t hear, and his BioChromatic aura grew stronger as he recovered his Breath from the cloak. “We should be moving again soon,” the man said to his friends. “The crowd is thinning up ahead.”

Indeed, soon the crowd started to progress. It wasn’t long before Vivenna and Parlin entered the arena itself. They moved through the stone benches, choosing a place that wasn’t too crowded, and Vivenna looked urgently through the boxes set above. The building was ornate, but not really very big, and so it didn’t take her long to locate Siri.

When she did, her heart sank.~\textit{My~.~.~. sister,}~Vivenna thought with a chill.~\textit{My poor sister.}

Siri was dressed in a scandalous golden dress that didn’t even come down to her knees. It also had a plunging neckline. Siri’s hair, which even she should have been able to keep a dark brown, was instead the golden yellow of enjoyment, and there were deep red ribbons woven through it. She was being attended by dozens of servants.

“Look what they’ve done to her,” Vivenna said. “She must be frightened senseless, forced to wear something like that, forced to keep her hair a color that matches her clothing~.~.~.”~\textit{Forced to be slave to the God King.}

Parlin’s square-jawed face grew hard. He didn’t often get angry, but Vivenna could see it in him now. She agreed. Siri was being exploited; they were carrying her around and displaying her like some kind of trophy. It seemed to Vivenna a statement. They were saying they could take a chaste, innocent Idrian woman and do whatever they wished with her.

\textit{What I’m doing is right,}~Vivenna thought with growing determination.~\textit{Coming to Hallandren}~was~\textit{the best thing to do. Lemex might be dead, but I have to press onward. I have to find a way.}

\textit{I have to save my sister.}

“Vivenna?” Parlin said.

“Hum?” Vivenna asked, distracted.

“Why is everyone starting to bow?”

\orn

Siri played idly with one of the tassels on her dress. The final god was seating himself in his box.\textit{That’s twenty-five,}~she thought.~\textit{That should be all of them.}

Suddenly, out in the audience, people began to rise, then kneel to the ground. Siri stood, searching anxiously. What was she missing? Had the God King arrived, or was this something else? Even the gods had gone down on their knees, though they didn’t prostrate themselves as the mortals did. They all seemed to be bowing toward Siri.~\textit{Some sort of ritual greeting for their new queen?}

Then she saw it. Her dress exploded with color, the stone at her feet gained luster, and her very skin became more vibrant. In front of her, a white serving bowl began to shine; then it seemed to stretch, the white color splitting into the colors of the rainbow.

A serving woman tugged on Siri’s sleeve from where she knelt below. “Vessel,” the woman whispered, “behind you!”

