% \section{Annotation Chapter Fifty-Four}
\section*{Annotations}

\subsection*{Vivenna Meets with the Beggar, Then Goes to Get Nightblood}

She lays it on a little thick here. But hey, if you’re a beggar, sometimes you like to be brownnosed. Plus, she’s new to this kind of thing, and she \textit{did} give him a very pretty handkerchief. . . .

It was sad to kill off Old Chaps so fast, but at least he went out with style. Besides, he wasn’t a very good person, as you can guess. He quite literally sold out his own mother once. He wanted her apartment, so he pinned a theft on her when he was a teenager. That’s the room where he was still living. He didn’t realize, in his youth, that she didn’t even own the place, and all he ended up inheriting was a rent payment. Not exactly the brightest guy around. But at least he waited until after she had died in prison to tie rocks to her feet and toss her into the bay.

Nightblood was interesting to write in this book as he makes a very nice contrast to Vasher. Vasher doesn’t want to say anything about his past; he’s so tight-lipped about it that he rarely even spends any time thinking about it. Nightblood, however, dwells quite heavily on the past. Though in some ways his mind is very capable, he has the quirk of being an Awakened object. The first hours of his life—during which time he met Shashara, Denth, and Vasher—imprinted heavily on him. It’s like . . . a part of his mind is hard forged in that moment with read-only memory that cannot be changed. Much of him can learn and grow, despite what Vasher says, but he cannot overwrite those initial concepts, states, and understandings that were burned into him during his birth. Shashara was alive then, so he will always think of her as alive, even if thousands of years have passed. Denth will always be pleased with him. Vasher will always be friends with the other two. Those things were some of Nightblood’s first impressions.

\subsection*{Vasher Is Tortured More}

It’s very important to note that Vasher is hiding and saving his strength here. Writing his scenes here was tricky, since I knew that he would need to be able to pull off some feats of strength later in the book. I figured that one night of torture wouldn’t do very much to him, though I also didn’t want to spoil the tension by drawing too much attention to that fact.

Denth is frustrated, here, that he’s not enjoying the process of torturing his old friend—much as he’s frustrated with his life as it exists presently. He wants so badly to just be the carefree, work-for-whoever-pays thug. But he can’t. He can’t be like Tonk Fah, and it frustrates him. Hurting Vasher hurts Denth too, as it reminds him of so many things that have been lost.

\subsection*{Siri Is Taken to the God King, Then Discovers Who Is Really Behind the Attacks}

I’m hoping that by this point, readers will be very confused about the nature of this third force that is attacking. I hope it’s the good kind of confusion, though.

Let me explain. When I write, I sometimes want to inspire confusion. It helps keep the mysteries of the book shadowed and vague. It helps the reader connect with the characters, who—presumably—are also confused. But there’s a danger here in being too confusing. If the readers think that they’ve missed something, or if they can’t follow what is going on at all, then they will just put down the book.

The trick is to make certain to telegraph that the characters are confused as well, as I mentioned above. If the reader knows that they are supposed to be searching for answers, then it will be all right. (As long as it doesn’t get prolonged artificially.) If, instead, they get the impression that the author has simply made a mistake and isn’t explaining things clearly, they’ll react very differently.

Anyway, I hope that you have the first reaction and not the second. The twist of who is really behind everything should come as a shock, but I hope that it’s also well foreshadowed. The big clincher is the question that, perhaps, you’ve been asking this entire book. If the war is going to be so bad for everyone involved, then who could possibly be pushing for it to occur?

I’ve seeded quite a number of hints about the Pahn Kahl in the book. The first is Vahr and his rebellion, but there are a number of others. The first time that Siri assumes Bluefingers worships the Returned, he purses his lips in annoyance. We’ve got a lot of little hints like that that the Pahn Kahl are frustrated by their place in the empire. They controlled this land long ago; we discovered that from Hoid’s storytelling.

It’s well foreshadowed, but I still worry that it will be too surprising to people. This is primarily because I think that readers will just pass over the Pahn Kahl while reading. They’re forgettable by design. Easy to ignore, and most of the other characters have trouble remembering that they aren’t just Hallandren. They aren’t an angry and vocal minority, like the Idrians. They’re just there, or at least that’s how everyone sees them.

One of my big goals for this book, however, was to have a good reversal for who is the bad guy pulling the strings. It’s not the high priest. It’s not the crafty god. It’s not even the brutal mercenary. It’s the simple, quiet scribe. It’s one of the biggest conceptual reversals in the book. Hopefully it works for you.

\subsection*{Siri Saves Them from Bluefingers}

Some people, as I’ve said, have complained about Siri’s damsel in distress place in the book during the next couple of chapters. I want to draw their attention to this chapter, however, which is where she shines. She’s in control and careful. She’s become a leader out of necessity. She’s able to make demands of Treledees and get answers. And she’s gotten good enough at politics to make the connection that nobody else did, seeing through Bluefingers’s ploy.

If she hadn’t acted here in this chapter, this book would have ended very differently. She saved Susebron’s life here. Because of what she did, Bluefingers wasn’t able to implement his plan to sneak the two of them out onto the waiting boat in the Inner Sea. Her delay gave just enough time that Bluefingers had to go with his secondary plan of getting the God King to the dungeons for the next few chapters.

More than that, however, Siri became the person she needed to in these chapters. She was able to grow as much as Vivenna, but she didn’t have to be knocked down for it to happen first.

\subsection*{Lightsong Gives Up}

Oddly, Lightsong’s character arc here in this chapter was to give in. To give up, to abandon his mocking and his glibness. To finally accept what he’s been pretending all this time. That he’s useless.

The games are over.



