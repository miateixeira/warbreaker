\section{Annotation Chapter Twenty-Nine}

\subsection*{Siri and Susebron Discuss Mountains}

One of the things I like about having wildly different plots and viewpoints put into one book is that I can use the viewpoints for different things. In the case of this section of the book, we’ve got death and tension in Vivenna’s plot, and we have soul-searching and mystery in Lightsong’s plot. Amid this I was able to sprinkle Siri scenes that are more relaxed, with her and the God King talking and falling in love. The scenes add a nice balance to the book.

I made Susebron get better at spelling quite quickly—this is only our second scene with him writing on his board, but already the spelling errors are gone. There is some small justification of this—he’s able to use the artisan’s script, and he’s very clever; besides, the Hallandren alphabet is phonetic. But it still probably happens too quickly.

Having to slog through dialect is just too distracting for readers, however. I wanted to do it once to show his innocence, but I wanted to get past it quickly—as quickly as possible—so that it wouldn’t distract from the story. I don’t want Susebron to come off as too childlike; I think that would ruin the romance.

All in all, I think that these chapters are some of the most sensual ones I’ve ever written. I always think that hinting and reserving will always be better than over-the-top romance. The fact that the two of them are forbidden sex because of the danger of having a child, mixed with some of the conversations they have about beauty and their separate lives, makes a very nice tension that I’m pleased to have managed to work in.

\subsection*{Vivenna and Denth Visit the Corpses in the D’Denir Garden}

That these deaths happened in this place is a coincidence. Yes, Vasher killed these men because he knew they were connected with Denth. However, he didn’t do it in the garden because that was where Vivenna had been the day before. That just happened. (The garden is a popular meeting place after hours for clandestine operations. All Vasher had to do was throw in Nightblood and let him do what he does. To Vasher, that’s often all the justification he needs. If the sword can make them kill each other, then they were guilty.)

It was important to have this scene here, however, to reinforce the tension between Denth and Vasher. I also wanted a good chance for Vasher to watch Vivenna. She notices him, but doesn’t point him out to Denth—she’s too afraid of Denth making a scene, and she just wants to get away from Vasher.



<p></p>
<p>\subsection*{The Pahn Kahl Religion}</p>
<p>In the Siri section, she mentions the Pahn Kahl religion, but she doesn’t know what it is. This happens numerous times in the book, people getting confused about whether the Pahn Kahl are just Hallandren or being unable to describe their religion.</p>
<p>If you’re curious, the Pahn Kahl are nature worshippers who focus on the storms of the Inner Sea as a manifestation of their unity of five gods. They believe that all Returned are men who deny the power of the gods and are forbidden entrance into heaven, yet are otherwise just men and not sinners worthy of hell—so they’re given a chance to come back to have another try at life, to try to find belief this time.</p>
<p>Anyway, the purpose of having people so confused about the Pahn Kahl was to try to make readers vague about them in the same way. In this case, I want the reader to feel that the Pahn Kahl are unimportant, like the characters do, which is exactly the reason why the Pahn Kahl are so annoyed in the first place. If Hallandren didn’t take them for granted so much, there’s a good chance they wouldn’t be so inclined to rebel.</p>
<p></p>



