\section{Chapter Thirty-Nine}

A week living in the gutters served to drastically change Vivenna’s perspective on life.

She sold her hair on the second day for a depressingly small amount of money. The food that she’d bought hadn’t even filled her stomach, and she didn’t have the strength to regrow the locks. The haircut didn’t even have the dignity of being cleanly shorn—it was a ragged job of hackwork, and the remaining hair would have still been a pale white, save for the fact that it was matted and blackened with dirt and soot.

She’d thought about selling her Breath, but wouldn’t even know where to go or how to go about it. Besides, she had a strong feeling that Denth would be watching places where she might sell the Breath. Beyond that, she had no idea how to get the Breaths back from her shawl, now that she’d put them into it.

No. She had to remain secret, unseen. Couldn’t draw attention to herself.

She sat on the side of the street, holding out her hand to the passing crowds, keeping her eyes down. No offerings came. She wasn’t certain how the other beggars did it; their meager earnings seemed an amazing treasure. They knew so much she didn’t—where to sit, how to plead. Passers learned to avoid beggars, even with their eyes. The successful beggars, then, were those who managed to draw attention.

Vivenna wasn’t certain if she wanted the attention or not. Though the gnawing pain of hunger had eventually driven her out onto busy streets, she was still frightened that Denth or Vasher might find her.

The more hungry she got, the less other worries bothered her. Eating was a problem for\textit{now.}~Being killed by Denth or Vasher was a problem for~\textit{later.}

The flood of people in their colors continued to pass. Vivenna watched them without focusing on faces or bodies. Just colors. Like a spinning wheel, each spoke a different hue.\textit{Denth won’t find me here,}~she thought.~\textit{He won’t see the princess in the beggar on the side of the street.}

Her stomach growled. She was learning to ignore it. Just as the people ignored her. She didn’t feel like she was a true beggar or child of the street, not after just one week. But she~\textit{was}learning to imitate them, and her mind felt so fuzzy lately. Ever since she’d gotten rid of her Breath.

She pulled the shawl close. She kept it with her always.

She still hardly believed what Denth and the others had done. She had such fond memories of their joking. She couldn’t connect that to what she’d seen in the cellar. In fact, sometimes she found herself rising to seek them out. Surely the things she’d seen had been hallucinations. Surely they couldn’t be such terrible men.

\textit{That’s foolish,}~she thought.~\textit{I need to focus. Why isn’t my mind working right anymore?}

Focus on what? There wasn’t much to think about. She couldn’t go to Denth. Parlin was dead. The city authorities would be no help—she had now heard the rumors of the Idrian princess who had been causing such troubles. She’d be arrested in a heartbeat. If there were any more of her father’s agents in the city, she had no idea how to locate them without exposing herself to Denth. Besides, there was a good chance that Denth had found those agents and killed them. He’d been so clever at keeping her captive, quietly eliminating those who could have taken her to safety. What did her father think? Vivenna lost to him, every man he sent to retrieve her vanishing mysteriously, Hallandren inching closer and closer to declaring war.

Those were distant worries. Her stomach growled. There were soup kitchens in the city, but at the first one she’d gone to, she’d spotted Tonk Fah lounging in a doorway across the street. She’d turned and scurried away, hoping he hadn’t seen her. For the same reason, she didn’t dare leave the city. Denth was sure to have agents watching the gates. Besides, where would she go? She didn’t have the supplies for a trip back to Idris.

Perhaps she could leave if she managed to save up enough money. That was hard, almost impossible. Every time she got a coin, she spent it on food. She couldn’t help herself. Nothing else seemed to matter.

She’d already lost weight. Her stomach growled again.

So she sat, sweaty and filthy in the meager shade. She still wore only her shift and the shawl, though she was dirty enough that it was difficult to tell where clothing ended and skin began. Her former arrogant refusal to wear anything but the elegant dresses now seemed ridiculous.

She shook her head, trying to clear the fog from it. One week on the street felt like an eternity—yet she knew that she’d only just begun to experience the life of the poor. How did they survive, sleeping in alleyways, getting rained on every day, jumping at every sound, feeling so hungry they were tempted to pick at and eat the rotting garbage they found in gutters? She’d tried that. She’d even managed to keep some down.

It was the only thing she’d had to eat in two days.

Someone stopped beside her. She looked up, eager, hand stretching out further until she saw the colors he wore. Yellow and blue. City guard. She grabbed at her shawl, pulling it closer. It was foolish, she knew—nobody knew about the Breaths it contained. The move was reflexive. The shawl was the only thing she owned, and—meager though it was—several urchins had already tried to steal it from her while she slept.

The guard didn’t reach for her shawl. He just nudged her with his truncheon. “Hey,” he said. “Move. No begging on this corner.”

He didn’t explain. They never did. There were apparently rules about where beggars could sit and where they couldn’t, but nobody bothered to provide the specifics to the beggars. Laws were things of lords and gods, not the lowly.

\textit{I’m already starting to think about lords as if they were some other species.}

Vivenna rose and felt a moment of nausea and dizziness. She rested against the side of the building, and the guard nudged her again, prompting her to shuffle away.

She bowed her head and moved along with the crowd, though most of them kept their distance from her. Ironic that they would leave her space now that she didn’t care. She didn’t want to think about how she smelled—though more than the scent, it was the fear of being robbed that probably kept the others away. They needn’t have worried. She wasn’t skilled enough to cut purses or pick pockets, and she couldn’t afford to get caught trying.

She’d stopped worrying about the morality of stealing days ago. Even before leaving the slum alleys for the streets, she hadn’t been so naive as to believe that she wouldn’t steal if she were denied food, though she had assumed that it would take her far longer to reach that state.

She didn’t head to another corner, but instead shuffled out of the crowds, making her way back into the Idrian slums. Here she’d gained some small measure of ac ceptance. At least she was considered one of them. None knew that she was the princess—after that first man, nobody had recognized her. However, her accent had earned her a place.

She began to seek out a location to spend the night. That was one of the reasons she’d decided not to continue begging for the evening. It was a profitable time, true, but she was just so tired. She wanted a good place to sleep. She wouldn’t have thought that it would make much difference which alleyway one huddled in, but some were warmer than others and some had better cover from the rain. Some were safer. She was beginning to learn these things, as well as who to avoid angering.

In her case, that last group included pretty much everyone—including the urchins. They were all above her in the pecking order. She’d learned that the second day. She’d tried to bring back a coin from selling her hair, intending to save it for a chance at leaving the city. She wasn’t certain how the urchins had known that she had coin, but she’d gotten her first beating that day.

Her favorite alleyway turned out to be occupied by a group of men with dark expressions, doing something that was obviously illegal. She left quickly, going to her second-favorite. It was crowded with a gang of urchins. The ones who had beaten her before. She left that one quickly as well.

The third alley was empty. This one was beside a building with a bakery. The ovens hadn’t yet been stoked for the night’s baking, but they would provide some warmth through the walls in the early morning.

She lay down, curling up with her back against the bricks, clutching her shawl close. Despite the lack of pillow or blanket, she was asleep in moments.

