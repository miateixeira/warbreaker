\section*{Annotations}

\subsection*{Naming in This Book}

The names in this novel, particularly in Hallandren and Idris, follow the concept of repeated consonant sounds.

I wanted to try something a little more distinctive in this book than the names were in \textit{Mistborn}. In that book, I intentionally backed away from the insane craziness of the names in \textit{Elantris}. I've written \href{https://brandonsanderson.com/creating-the-languages-of-elantris/}{entire essays} on how I devised the languages in that book. The names were appropriate for the novel, since the language was so important to the story. However, I know that the number and oddity of many of the names in \textit{Elantris} was off-putting.

So, instead, in \textit{Mistborn} I chose names that were much easier to say, and gave everyone a simple nickname. When it came time for \textit{Warbreaker}, I wanted to try something else, to take a step back toward distinctiveness in the language, but not go as far as I had in \textit{Elantris}.

I’ve long toyed with using double consonants as a naming structure. I played with a lot of different ways of writing these. I could either use the letters doubled up, with no break (\textit{Ttelir}). I could slip a vowel in the middle and hope people pronounced it as a schwa sound (\textit{Tetelir}). Or I could use the fantasy standard of an apostrophe (\textit{T’telir}).

In the end, I decided to go with all three. I felt that writing all the names after one of the ways would look repetitive and annoying. By using all three, I could have variety, yet also have a theme. So, you have doubles in names like Llarimar. You have inserted vowels like in Vivenna. And you have apostrophes like in T’Telir.

I think it turned out well. Some members of my writing group complained about fantasy novels and their overuse of apostrophes in names. My answer: Tough. Just because English doesn’t like to do it doesn’t mean we have to eschew it in other languages. I like the way T’Telir looks with an apostrophe, and the way people will say it. So it stays.
% <img draggable="false" role="img" class="emoji" alt="😉" src="https://s.w.org/images/core/emoji/14.0.0/svg/1f609.svg">

\subsection*{Siri Approaches T’Telir}

And we finally get to see T’Telir. I’m still a tad bothered that it’s chapter four before we get to see the city. I worry that people will read the book and have trouble getting grounded in it, since we’ve now had five viewpoints across five chapters and have been in a lot of different locations.

However, I think that the groundwork in the first four chapters is needed to make the book work. I just couldn’t figure out a way to cut it all out and still have things work. Perhaps (just perhaps) I could have moved the Vasher prologue into the middle and made it a regular chapter, then moved the original Siri/Dedelin chapter to a prologue. Then, with the decision to send Siri into the city made, I could have jumped straight to this one. However, we’d have lost too much in that. Doing it this way isn’t perfect either, but I think it’s still the best way the book could have been done.

\subsection*{Hawaii}

Why, yes, I did visit Hawaii in the middle of writing this book. Did you notice?

Following \textit{Mistborn}, I wanted to do a book set in a place that looked \textit{very} different from the Final Empire. What’s different from a burned-out wasteland? Why, a tropical paradise of course! One of the great things about being an author is the ability to justify going to Hawaii just so I could do research on how to properly describe the plants, landscape, and atmosphere in a place like that. It’s really a tough job, but I’m willing to sacrifice for you all. No need to thank me.

\subsection*{Undead}

I’d been toying for a long time with doing a book with “technological” undead in a fantasy world. A place where a body could be recycled, restored to a semblance of life, then set to work. I’m always looking for ways to explore new ground in fantasy, and I’ve seen people sticking to the same old tropes with undead. (Mindless, rotting zombies or dynamic, goth-dressed vampires.)

I wanted to play with a middle ground. If you’ve got a magic that can make a stick figure come to life, what could it do with a dead body? How could a society make use of these walking corpses, treating them as a realistic resource?

The Lifeless grew out of this desire. I developed something like them for use earlier in a completely different novel, but I abandoned that plan years ago. They returned to the scrap pile of my mind, from which I draw forth and recombine ideas to create novels.

\subsection*{Other Notes}

Yes, there are Returned in Idris. There are Returned everywhere in this world that there are people. (The name of this world is Nalthis, by the way. \textit{Mistborn} takes place on a world called Scadrial, and \textit{Elantris} on a world known as Sel. See the fun things you learn by reading annotations?)

I’d like someday to do a sequel to \textit{Warbreaker}, in part because I want to show off all of the different ways people in Nalthis deal with the Returned. They’re treated in very strange ways some places. For instance, just across the mountains there’s a kingdom where when someone dies in a way that might be heroic, the corpse is immediately purchased by a nobleman hoping to hit the jackpot and get a Returned. You see, since Returned can heal people, keeping one around to act as an emergency insurance plan to restore your health is a great idea.

Also, just in case you’re wondering, the Bright Sea and the Inner Sea are both the same place. It’s another Idris/Hallandren thing. Most mountains, oceans, and lakes have two names—the Idrian one and the Hallandren one. Originally, this happened because there was bad blood between the two kingdoms, so they’d call things different names in order to differentiate themselves. Ironically, in a lot of cases both names have stuck, and both kingdoms have found themselves alternating between the two names.

Inner Sea was the Idrian name for the body of water, renamed because they wanted to downplay how important it was. (Idris is landlocked, after all.) Bright Sea was the original name.




