% \section{Chapter Forty-Six}
\chapter{}

Vivenna choked down her meal. The dried meat tasted strongly of fish, but she had learned that by breathing through her mouth, she could ignore much of the flavor. She ate every bite, then washed the taste away with a few mouthfuls of warm boiled water.

She was alone in the room. It was a small chamber built onto the side of a building near the slums. Vasher had paid a few coins for a day in it, though he wasn’t there at the moment. He’d rushed off to deal with something.

She leaned back, food consumed, closing her eyes. She’d reached the point where she was so exhausted that she actually found it difficult to sleep. The fact that the room was so small didn’t help. She couldn’t even stretch out all the way.

Vasher hadn’t been exaggerating when he’d said that their work would be rigorous. Stop after stop, she spoke with the Idrians, consoling them, begging them not to push Hallandren to war. There were no restaurants as there had been with Denth. No dinners with men in fine clothing and guards. Just group after group of tired, working-class men and women. Many of them weren’t rebellious and a large number of them didn’t even live in the slums. But they were part of the Idrian community in T’Telir, and they could influence how their friends and family felt.

She liked them. Empathized with them. She felt far better about her new efforts than she had about her work with Denth, and so far as she could tell, Vasher was being honest with her. She had decided to trust those instincts. That was her decision, and that decision meant helping Vasher, for now.

Vasher didn’t ask her if she wanted to continue. He simply led her from location to location, expecting her to keep up. And so she did, meeting with the people and begging their forgiveness, despite how emotionally draining it was. She wasn’t certain if she could repair what she had done, but she was willing to try. This determination seemed to gain her some respect from Vasher. It was much more reluctantly given than Denth’s respect had been.

\textit{Denth was fooling me the entire time.}~It was still hard to remember that fact. Part of her didn’t want to. She leaned forward, staring at the bland wall in front of her in the boxlike room. She shivered. It was a good thing that she’d been working herself so hard lately. It kept her from thinking about things.

Discomforting things.

Who was she? How did she define herself now that everything she’d been, and everything she’d tried, had collapsed around her? She couldn’t be Vivenna the confident princess anymore. That person was dead, left behind in that cellar with Parlin’s bloody corpse. Her confidence had come from na’veté.

Now she knew how easily she had been played. She knew the cost of ignorance, and she had glimpsed the grim truths of real poverty.

Yet, she also couldn’t be~\textit{that}~woman—the waif of the streets, the thief, the beaten-down wretch. That wasn’t her. She felt as if those weeks had been a dream, brought on by the stress of isolation and trauma of her betrayal, fueled by becoming a Drab and being suffocated by disease. To pretend that was the real her would be to parody those who truly lived on the streets. The people she’d hidden among and tried to imitate.

Where did that leave her? Was she the penitent, quiet princess who knelt with bowed head, pleading with the peasants? This, too, was partially an act. She really did feel sorry. However, she was using her stripped pride as a tool. That wasn’t her.

Who was she?

She stood up, feeling cramped in the tiny room, and pushed open the door. The neighborhood outside wasn’t quite a slum, but it wasn’t rich either. It was simply a place where people lived. There were enough colors along to street to be welcoming, but the buildings were small and held a number of families each.

She walked along the street, careful not to stray too far from the room Vasher had rented. She passed trees, admiring their blooms.

Who was she really? What was left, when one stripped away the princess and the hatred of Hallandren? She was determined. That part of her, she liked. She’d forced herself to become the woman she needed to be in order to marry the God King. She’d worked hard, sacrificing, to reach her goal.

She was also a hypocrite. Now she knew what it was to be truly humble. Compared to that, her former life seemed more brash and arrogant than any colorful skirt or shirt.

She did believe in Austre. She loved the teachings of the Five Visions. Humility. Sacrifice. Seeing another’s problems before your own. Yet she was beginning to think that she—along with many others—had taken this belief too far, letting her desire to seem humble become a form of pride itself. She now saw that when her faith had become about clothing instead of people, it had taken a wrong turn.

She wanted to learn to Awaken. Why? What did that say about her? That she was willing to accept a tool her religion rejected, just because it would make her powerful?

No, that wasn’t it. At least, she hoped it wasn’t.

Looking back on her recent life, she felt frustrated at her frequent helplessness. And~\textit{that}~felt like part of who she really was. The woman who would do anything to be sure she wasn’t helpless. That was why she’d studied so hard with the tutors in Idris. That was also why she wanted to learn how to Awaken. She wanted as much information as she could get, and wanted to be prepared for the problems that might come at her.

She wanted to be capable. That might be arrogant, but it was the truth. She wanted to learn everything she could about how to survive in the world. The most humiliating aspect of her time in T’Telir was her ignorance. She wouldn’t make that mistake again.

She nodded to herself.

\textit{Time to practice, then,}~she thought, returning to the room. Inside, she pulled out a piece of rope—the one that Vasher had used to tie her up, the first thing that she had Awakened. She’d since retrieved the Breath from it.

She went back outside, holding the rope between her fingers, twisting it, thinking.~\textit{The Commands that Denth taught me were simple phrases. Hold things. Protect me.}~He’d implied that the intent was important. When she’d Awakened her bonds, she’d made them move as if part of her body. It was more than just the Command. The Command brought the life, but the intent—the instructions from her mind—brought focus and action.

She stopped beside a large tree with thin, blossom-laden branches that drooped toward the ground. She stood beside a branch, touched the bark of the tree’s trunk itself to use its color. She held out the rope to the branch. “Hold things,” she Commanded, reflexively letting out some of her Breath. She felt an instant of panic as her sense of the world dimmed.

The rope twitched. However, instead of drawing color from the tree, the Awakening pulled color from her tunic. The garment bled grey, and the rope moved, wrapping like a snake around the branch. Wood cracked slightly as the rope pulled tight. However, the other end of the rope twisted in an odd pattern, writhing.

Vivenna watched, frowning, until she figured out what was happening. The rope was twisting around her hand, trying to hold it as well.

“Stop,” Vivenna said.

Nothing happened. It continued to pull tight.

“Your Breath to mine,” she Commanded.

The rope stopped twisting and her Breath returned. She shook the rope free.~\textit{All right,}~she thought.~\textit{“Hold things” works, but it’s not very specific. It will wrap around my fingers as well as the thing I want it to tie up. What if I tried something else?}

“Hold that branch,” she Commanded. Again, Breath left her. More of it this time. Her trousers drained of color, and the rope end twisted, wrapping around the branch. The rest of it remained still.

She smiled in satisfaction.~\textit{So the more complicated the Command, the more Breath it requires.}

She took back her Breath. As Vasher had explained, doing so didn’t shock her senses, for it was a mere restoration to a normal state for her. If she’d gone several days without that Breath, she’d have been overwhelmed by recovering its power. It was a little like taking a first bite of something very flavorful.

She eyed her clothes, which were now completely grey. Out of curiosity, she tried Awakening the rope again. Nothing happened. She picked up a stick, then Awakened the rope. It worked this time, the stick losing its color, though it took a~\textit{lot}~more breath. Perhaps this was because the stick wasn’t very colorful. The tree trunk didn’t work for color, though. Presumably, one couldn’t draw color from something that was itself alive.

She discarded the branch and fetched a few of Vasher’s colored handkerchiefs from the room. She walked back to the tree.~\textit{Now what?}~she thought. Could she put the Breath into the rope now, then command it to hold something later? How would she even phrase that?

“Hold things that I tell you to hold,” she Commanded.

Nothing happened.

“Hold that branch when I tell you.”

Again, nothing.

“Hold whatever I say.”

Nothing.

A voice came from behind. “Tell it to ‘Hold when thrown.’~”

Vivenna jumped, spinning. Vasher stood behind her, Nightblood held before him, point down. He had his pack over his shoulder.

Vivenna flushed, glancing back at the rope. “Hold when thrown,” she said, using a handkerchief for color. Her Breath left her, but the rope remained limp. So she tossed it to the side, hitting one of the hanging tree branches.

The rope immediately twisted about, locking the branches together and holding them tightly.

“That’s useful,” Vivenna said.

Vasher raised an eyebrow. “Perhaps. Dangerous though.”

“Why?”

“Get the rope back.”

Vivenna paused, realizing that the rope had twisted around branches that were too high for her to reach. She hopped up, trying to grab it.

“I prefer to use a longer rope,” Vasher said, raising Nightblood by the blade and using its hooked crossguard to pull the branches down. “If you always keep hold of one end, then you don’t have to worry about it getting taken from you. Plus, you can Awaken when you need to, rather than leaving a bunch of Breath locked into a rope that you may or may not need.”

Vivenna nodded, recovering her Breath from the rope.

“Come on,” he said, walking back toward the room. “You’ve made enough of a spectacle for one day.”

Vivenna followed, noticing that several people on the street had stopped to watch her. “How did they notice?” she asked. “I wasn’t~\textit{that}~obvious about what I was doing.”

Vasher snorted. “And how many people in T’Telir walk around in grey clothing?”

Vivenna blushed as she followed Vasher into the cramped room. He set down his pack and then leaned Nightblood against one wall. Vivenna eyed the sword. She still wasn’t certain what to make of the weapon. She felt a little nauseated every time she looked at it, and the memory of how violently sick she’d felt when touching it was still fresh.

Plus there had been that voice in her head. Had she really heard it? Vasher had been characteristically tight-lipped when she’d asked about it, rebuffing her questions.

“Aren’t you an Idrian?” Vasher asked, drawing her attention as he settled down.

“Last I checked,” she replied.

“You seem oddly fascinated with Awakening for a follower of Austre.” He spoke with eyes closed as he rested his head against the door.

“I’m not a very good Idrian,” she said, sitting down. “Not anymore. I might as well learn to use what I have.”

Vasher nodded. “Good enough. I’ve never really understood why Austrism suddenly turned its back on Awakening.”

“Suddenly?”

He nodded, eyes still closed. “Wasn’t like that before the Manywar.”

“Really?”

“Of course,” he said.

He often spoke that way, mentioning things that seemed farfetched to her, yet saying them as if he knew exactly what he was talking about. No conjecture. No wavering. As if he knew everything. She could see why it was sometimes hard for him to get along with people.

“Anyway,” Vasher said, opening his eyes. “Did you eat all of that squid?”

She nodded. “Is that what that was?”

“Yes,” he said, opening his pack, getting out another dried chunk of meat. He held it up. “Want more?”

She felt sick. “No, thank you.”

He paused, noticing the look in her eyes. “What? Did I give you a bad piece?”

She shook her head.

“What?” he asked.

“It’s nothing.”

He raised an eyebrow.

“I said it’s nothing.” She glanced away. “I just don’t care for fish very much.”

“You don’t?” he asked. “I’ve been feeding it to you for five days now.”

She nodded silently.

“You ate it every time.”

“I’m dependent upon you for food,” she said. “I don’t intend to complain about what you give me.”

He frowned, then took a bite of squid and began chewing. He still wore his torn, almost-ragged clothing, but Vivenna had now been around him enough to know that he kept it clean. He obviously had the resources to get new clothing, yet he chose to wear the worn and tattered things instead. He also wore the same half-scrub, half-beard on his face. It never seemed to get longer, yet she never saw him trim it or shave it. How did he manage to keep it just the right length? Was that intentional, or was she reading too much into it?

“You aren’t what I expected,” he said.

“I would have been,” she said. “A few weeks ago.”

“I doubt it,” he said, gnawing on his chunk of squid. “That tenacious spirit you’ve got doesn’t come from a few weeks on the streets. Neither does that sense of martyrdom.”

She met his eyes. “I want you to teach me more about Awakening.”

He shrugged. “What do you want to know?”

“I don’t even know how to answer that,” she said. “Denth taught me a few Commands, but that was the same day that you took me captive.”

Vasher nodded. They sat silent for a few minutes.

“Well?” she finally asked. “Are you going to say anything?”

“I’m thinking,” he said.

She raised an eyebrow.

He scowled. “Awakening is something I’ve done for a very, very long time. I always have trouble trying to explain it. Don’t rush me.”

“It’s okay,” she said. “Take your time.”

He shot her a glance. “Don’t patronize me either.”

“I’m not patronizing; I’m being polite.”

“Well next time, be polite with less condescension in your voice,” he said.

\textit{Condescension?}~she thought.~\textit{I wasn’t condescending!}~She eyed him as he sat, chewing on his dried squid. The more time she spent with him, the less frightening she found him, but the more frustrating.~\textit{He is a dangerous man,}~she reminded herself.~\textit{He has left corpses strewn all over the city, using that sword of his to make people slaughter each other.}

She’d considered running from him on several occasions, but had eventually decided that she’d be a fool to do so. She could find no fault in his efforts to stop the war, and his solemn promise in the basement that first day still stuck with her. She believed him. Hesitantly.

She just intended to keep her eyes open a little wider from now on.

“All right,” he said. “I guess this is for the best. I’m getting tired of you walking around with that bright aura of yours that you can’t even use.”

“Well?”

“Well, I think we should start with theory,” he said. “There are four kinds of BioChromatic entities. The first, and most spectacular, are the Returned. They’re called gods here in Hallandren, but I’d rather call them Spontaneous Sentient BioChromatic Manifestations in a Deceased Host. What is odd about them is that they’re the only naturally occurring BioChromatic entity, which is theoretically the explanation for why they can’t use or bestow their BioChromatic Investiture. Of course, the fact is that~\textit{every}~living being is born with a certain BioChromatic Investiture. This could also explain why Type Ones retain sentience.”

Vivenna blinked. That wasn’t what she had been expecting.

“You’re more interested in Type Two and Type Three entities,” Vasher continued. “Type Two being Mindless Manifestations in a Deceased Host. They are cheap to make, even with awkward Commands. This is per the Law of BioChromatic Parallelism: the closer a host is to a living shape and form, the easier it is to Awaken. BioChroma is the power of life, and so it seeks patterns of life. That, however, leads us to another law—the Law of Comparability. It states that the amount of Breath required to Awaken something isn’t necessarily indicative of its power once Awakened. A piece of cloth cut into a square and a piece of cloth cut into the shape of a person will take very different amounts of Breath to Awaken, but will be essentially the same once they have been Invested.

“The explanation for this is simple. Some people think of Awakening as pouring water into a cup. You pour until the cup is filled, and then the object comes to life. This is a false analogy. Instead, think of Awakening as beating down a door. You pound and pound, and some doors are easier to open than others, but once they’re open, they do about the same thing.”

He glanced at her. “Understand?”

“Uh~.~.~.” she said. She’d spent her youth training with the tutors, but this was beyond even their methods of teaching. “It’s a little dense.”

“Well, do you want to learn or not?”

\textit{You asked me if I understood,}~she thought.~\textit{And I answered.}~However, she didn’t voice her objections. Better for him to keep talking.

“Type Two BioChromatic entities,” he said, “are what people in Hallandren call Lifeless. They are different from Type One entities in several ways. Lifeless can be created at will, and require only a few Breaths to Awaken—anywhere between one and hundreds, depending on the Commands used—and they feed off of their own color when being Invested. They don’t present an aura when Awakened, but the Breath sustains them, keeping them from needing to eat. They can die, and need a special alcohol solution to remain functional past a few years of Awakened status. Because of their organic host, their Breath clings to the body, and cannot be withdrawn once Invested.”

“I know a little about them,” Vivenna said, “Denth and his team have a Lifeless.”

Vasher fell silent. “Yes,” he finally said. “I know.”

Vivenna frowned, noticing a strange look in his eyes. They sat silently for a few moments. “You were talking about Lifeless and their Commands?” she prompted.

Vasher nodded. “They need a Command to Awaken them, just like anything else. Even your religion teaches about Commands—it says that Austre is the one who Commands the Returned to come back.”

She nodded.

“Understanding the theory of Commands is tough. Look at Lifeless, for instance. It’s taken us centuries to discover the most efficient ways to bring a body to a Lifeless state. Even now, we’re not sure if we understand how it works. I guess~\textit{this}~is the first thing I’d like to get across to you—that BioChroma is complicated, and we really don’t understand most of it.”

“What do you mean?” she asked.

“Just what I said,” Vasher replied, shrugging. “We don’t really know what we’re doing.”

“But you sound so technical and precise in your descriptions.”

“We’ve figured out some things,” he said. “But Awakeners really haven’t been around that long. The more you learn about BioChroma, the more you’ll realize that there are more things that we~\textit{don’t}~know than there are things we do. Why are the specific Commands so important, and why do they have to be spoken in your native language? What brings Type One entities—Returned—back to life in the first place? Why are Lifeless so dull-minded, while Returned fully sentient?”

Vivenna nodded.

“Creating Type Three BioChromatic entities is what we traditionally call ‘Awakening,’~” Vasher continued. “That’s when you create a BioChromatic manifestation in an organic host that is far removed from having been alive. Cloth works the best, though sticks, reeds, and other plant matter can be used.”

“What about bones?” Vivenna asked.

“They’re strange,” Vasher said. “They take far more Breath to awaken than a body held together with flesh and aren’t as flexible as something like cloth. Still, Breath will stick to them fairly easily, since they were once alive and maintain the form of a living thing.”

“So Idrian stories that talk about skeletal armies~\textit{aren’t}~just fabrications?”

He chuckled. “Oh, they are. If you wanted to Awaken a skeleton, you’d have to arrange all the bones together in their correct places. That’s a lot of work for something that will take upwards of fifty or a hundred Breaths to Awaken. Intact corpses make far more sense economically, even if the Breath sticks to them so well that it becomes impossible to recover. Still, I’ve seen some very interesting things done with skeletons which have been Awakened.

“Anyway, Type Three entities—regular Awakened objects—are different. BioChroma doesn’t stick to them very well at all. The result is that they require quite a bit of Investiture—often well over a hundred Breaths—to Awaken them. The benefit of this, of course, is that the Breath can be drawn back out again. This has allowed for quite a bit more experimentation, and that has resulted in a more comprehensive understanding of Awakening techniques.”

“You mean the Commands?” Vivenna asked.

“Right,” Vasher said. “As you’ve seen, most basic Commands work easily. If the Command is something the object could do, and you state it in a simple way, the Command will usually work.”

“I tried some simple Commands,” she said. “On the rope. They didn’t work.”

“Those may have sounded simple, but they weren’t. Simple Commands are only two words long. Grab something. Hold something. Move up. Move down. Twist around. Even some two-word Commands can be more complicated, and it takes practice visualizing—or, well, imagining. Well, using your mind to—”

“I understand that part,” she said. “Like flexing a muscle.”

He nodded. “The Command ‘Protect me,’ though only two words, is extremely complicated. So are others, like Fetch something. You have to give the right impulse to the object. This area is where you really begin to understand how little we know. There are probably thousands of Commands we don’t know. The more words you add, the more complicated the mental component becomes, which is why discovering a new Command can take years of study.”

“Like the discovery of a new Command to make Lifeless,” she said thoughtfully. “Three hundred years ago, those who had the one-Breath Command could make their Lifeless much more cheaply than those who didn’t. That disparity started the Manywar.”

“Yes,” Vasher said. “Or, at least, that was part of what caused the war. It’s not really important. The thing to understand is that we’re still children when it comes to Awakening. It doesn’t help that a lot of people who learn new, valuable Commands never share them, and probably die with the knowledge.”

Vivenna nodded, noticing how his lesson grew more relaxed and conversational as he got into the topic. His expertise surprised her.

\textit{He sits on the floor,}~she thought,~\textit{eating a dry piece of squid, not having shaven in weeks and wearing clothing that looks like it’s about to fall off. Yet he talks like a scholar giving a lecture. He carries a sword that leaks black smoke and causes people to kill each other, yet he works so hard to stop a war. Who is this man?}

She glanced to the side, to where Nightblood sat leaning against the wall. Perhaps it was the discussion of the technical aspects of BioChroma, or perhaps it was simply her growing suspicion. She was beginning to understand what wasn’t right about the sword.

“What is a Type Four BioChromatic entity?” Vivenna asked, glancing back at Vasher.

He fell silent.

“Type One is a human body with sentience,” Vivenna said. “Type Two is a human body without sentience. Type three is an Awakened object like a rope—an object with no sentience. Is there a way to create an Awakened object~\textit{with}~sentience? Like a Returned, but inside of something other than a human body?”

Vasher stood. “We’ve covered enough for one day.”

“You didn’t answer my question.”

“And I’m not going to,” he said. “And I advise you never to ask it again. Understand?” He glanced at her, and she felt a chill at the harshness in his voice.

“All right,” she said, though she didn’t glance away.

He snorted to himself, then reached into his large pack, yanking something out. “Here,” he said. “I brought you something.”

He tossed a long, cloth-wrapped object to the floor. Vivenna stood, walking over to pull the cloth off. Inside was a thin, well-polished dueling blade.

“I don’t know how to use one of these,” she said.

“Then learn,” he replied. “If you know how to fight, you’ll be far less annoying to have around. I won’t have to keep pulling you out of trouble all the time.”

She flushed. “One time.”

“It’ll happen again,” he said.

She hesitantly picked up the sheathed sword, surprised at how light it was.

“Let’s go,” Vasher said. “I’ve got another group for us to visit.”

