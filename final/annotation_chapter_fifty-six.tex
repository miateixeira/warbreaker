% \section{Annotation Chapter Fifty-Six}
\section*{Annotations}

\subsection*{Vivenna Saves Vasher—Kind Of}

Vivenna has a few things going for her here. First off, Denth has gotten rid of his Breath. He doesn’t want to have it as he tortures Vasher. It made him too aware, too pained. Being a drab as he does it is easier for him. With Tonks dozing, that means that nobody in the room has enough life sense to notice Vivenna hanging outside.

Secondly, Denth doesn’t really like what he’s doing. He feels he took Vasher too easily, and the torture isn’t satisfying. He’d much rather kill Vasher in a fight, as he later realizes. So there’s some hesitance to him in this scene, as you might notice. He doesn’t just stab Vasher or Vivenna. He goes to free Tonk Fah, then hesitates before turning back and challenging Vasher. Denth was actually hoping that something like this would happen. (Plus, he \textit{does} care for his friend Tonk Fah. Again, Denth is far from purely evil, no matter what he would like people to assume.)

Denth \textit{is} the better duelist. Even if Vasher hadn’t been beaten and tortured, Denth would have won. Except for the trick Vasher was planning, which Vivenna interfered with. But we don’t know about that yet . . .

I don’t know if you remember that Vivenna put a whole bunch of Breath into Tonk Fah’s cloak accidentally, but it happened during the time when she found Parlin. It might be just a little bit of a stretch here, as I don’t know that people will remember it. As I consider it, I should have mentioned what she’d done one more time.

Also, I hope that you don’t mind the line that goes something like “Vasher is plunging to his doom from a three story window—of course he’ll live!” It’s a little bit self-aware, and I’m not trying to break the fourth wall. Denth has simply known Vasher for a very, very long time, and knows that something so simple isn’t likely to kill his old friend. That, mixed with Denth’s penchant for sarcasm, produced this line.

\subsection*{Vasher Fights the Soldiers and Finally Pulls Nightblood}

In my annotations, I’ve often talked about focus scenes. These are the scenes of a book that I imagine cinematically before I sit down and write the novel. They’re part of what drives me to want to work on that book in particular, and I need a few really good ones before I’ll write a book.

This was one of the primary focus scenes for this book. I had this in mind before I developed a lot of the rest of the story. I’m glad that I was able to write to a point where I was able to use it. Vasher, Awakening a rope to save himself, then fighting alongside Awakened sets of clothing. Then finally, at long last, drawing Nightblood. You probably knew that had to happen in this book. I certainly built up to it long enough.

I originally imagined the pulling of Nightblood from a body a little like a dark “sword in the stone” moment. I don’t think that quite made the transition to the final book, but hopefully the image of a black sword leaking smoke is visually potent for you. I ended the scene in my head with Vasher standing amid those puffs of black smoke that used to be bodies, Nightblood at his side, feeding off of him with pulsing black veins.

\subsection*{Denth Finds Vasher and Forces Him to Duel}

Note that Denth, way back many chapters ago, mentioned that he felt the only way to defeat Vasher was to get him to draw Nightblood. Denth knew that would leak away all of Vasher’s Breath and thereby leave him unable to use Nightblood any further. (This exchange with Denth and Tonk Fah happened in the D’Denir garden after meeting with the forgers.)

Denth has been planning to find a way to force Vasher to draw the sword and use it. He was hoping that the sword would consume him, which he felt would be a fitting end for Vasher, considering that Vasher killed Denth’s sister with Nightblood. When he didn’t die from pulling the blade, Denth decided that killing him with a dueling blade—as Arsteel should have—would be a fitting end instead.



