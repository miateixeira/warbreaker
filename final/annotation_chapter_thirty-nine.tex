% \section{Annotation Chapter Thirty-Nine}
\section*{Annotations}

\subsection*{Vivenna Begs}

This chapter and the next one were originally a single chapter. In the drafting process, I realized that my original chapter just wouldn’t do. I’d been in a hurry to get on with Vivenna’s viewpoint, and I had been worried about spending a lot of time on the streets with her, since I didn’t want to retread ground I’ve seen in a lot of other books.

In this case, I was letting my bias against doing the expected thing make the book worse. Now, my drive to find new twists on fantasy tropes and plots usually serves me well. I think it makes my books stand out. You know that when you pick up a Brandon Sanderson fantasy novel, you’re going to get a complex, epic story with an original take on magic and a different spin on the fantasy archetypes.

However, this same sense can be problematic if I let it drive me too far. It’s nearly impossible to write a book that doesn’t echo anything someone else has done. It’s tough enough to come up with one original idea, let alone make every single idea in a book original. I think that trying to do so would be a path to folly—a path to rarely, if ever, completing anything.

In this case, we \textit{needed} to have a longer time with Vivenna on the streets. We needed it to feel like she’d earned the sections of time she spent there. I knew I didn’t want to go overboard on it, but I also couldn’t skimp. So I sliced the chapter into two and added some material to each one, particularly the second chapter.

\subsection*{Vivenna Finds an Alley to Sleep In}

One of the big stories I’m worried about channeling here is \textit{Les Misérables}. It’s one of my favorite stories of all time, so sometimes it’s difficult not to find myself drawing upon Hugo’s story and characters. That constant fight to keep myself from leaning too much on what has come before went into overdrive in these chapters.

In the end, however, I think that Vivenna’s scenes belong here and accent the story. So yes, if you noticed them, there are some echoes of Fantine in these sections—Vivenna selling her hair and noticing the prostitutes most prominent among them. These two items, most of all, I considered cutting. But in the end, I decided that if there was anyone I was proud to have influencing my writing, it was Hugo, and I left the references. Partially as an homage, I guess—though that’s always the excuse of someone who ends up echoing a great story of the past.



