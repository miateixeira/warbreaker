\section{Annotation Chapter Seven}

\subsection*{Siri Enters the God King’s Chambers}

This is one of those chapter breaks that is there for stylistic drama more than anything else. Thematically, these two chapters are really the same chapter. However, I wanted to break before she steps in because it works so well as a dramatic turn in the story.

I’ve had e-mails asking me about how to decide when to break a chapter. Honestly, I’m not sure how to answer this one. Breaking chapters isn’t something I plan; it’s something I just do. A good chapter should have a nice arc of its own, with rising action, a climax, then perhaps some brief falling action. (And thinking of that, you can probably see why chapters five and six can be considered a single chapter in this regard.) But there’s not a real science to it—break where it feels right.

Anyway, Siri’s entrance here is probably the first big climactic moment of the book. It’s where I’ve been pushing the novel since the beginning, and is one of the focal scenes for this book. (The scenes that I imagine and develop before I being writing, which then propel their section of the novel.)

\subsection*{Blushweaver}

Blushweaver was the first of the gods who I named, and her title then set the standard for the others in the Court of Gods. Lightsong was second, and I toyed with several versions of his name before settling. Blushweaver’s name, however, came quickly and easily—and I never wanted to change it once I landed on it.

When developing the Court of Gods, I wanted to design something that felt a little like a Greek pantheon—or, rather, a constructed one. Everyone is given their portfolio by the priests after they Return. Blushweaver was given the portfolio of honesty and interpersonal relations, and over the fifteen years of her rule, she’s become one of the most dynamic figures in the court. Few remember it anymore, but she was successful at having her name changed during her first year. She used to be Blushweaver the Honest, and she became Blushweaver the Beautiful through a campaign and some clever politicking.

Many think of her as the goddess of love and romance, though that technically isn’t true. It’s just the name and persona she’s crafted for herself, as she saw that as a position of greater power. She actually toyed with going the opposite direction, becoming the chaste goddess of justice and honor. However, in the end, she decided to go the direction that felt more natural to her.

After these fifteen years, it’s hard to distinguish when she is being herself and when she’s playing a part. The two have become melded and interchangeable.

When designing this story, I knew I wanted to have a beautiful goddess to give Lightsong some verbal sparring. However, I realized early on that I didn’t want to go the route of having a disposable, sultry bimbo goddess of love. I needed someone more complicated and capable than that, someone who was a foil to Lightsong not just in verbal sparring, but someone who could prod him to be more proactive. And from that came Blushweaver.

In the original draft of the book, this chapter had a slightly different tone. Lightsong didn’t look forward to sparring with Blushweaver; he cringed and wished she wouldn’t bother him. That artifact remained until the later drafts, though it didn’t belong. I wrote the later chapters with them getting along quite well, so I wanted to revise this first chapter to imply that he looked forward to their conversations.



