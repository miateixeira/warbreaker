% \section{Annotation Chapter Forty-One}
\section*{Annotations}

\subsection*{Vivenna, Sick and Disoriented, Gets Turned Away by the Restaurant Keeper}

One of the ways I decided to make Vivenna’s sections here work better was by enhancing the fuzziness of her mind. By giving her this sense of numbness, I hope to indicate that something is not right with her.

It’s common for someone who suddenly becomes a Drab to get sick almost immediately. For a time, her immune system was magically enhanced and warded, in a way, to keep her from becoming ill. With that removed suddenly, sickness can strike. She hasn’t built up immunities to the sicknesses going around, and by becoming a Drab, her immune system suddenly works far worse than that of other people.

These things combined made her come down with something pretty nasty the very day she put away her Breath. This would have killed her, eventually, if she hadn’t done something about it. She would have grown so dizzy and confused that she wouldn’t have even been able to walk.

By sending men to find her, Denth saved her life.

Anyway, I feel that these scenes work much better now. We can look at Vivenna’s time on the streets in the same surreal sense that she does. They happened in the past, in a strange dream state. In that way, they can seem much longer than just two chapters and a couple of weeks.

\subsection*{Nightblood}

Nightblood’s name, by the way, is supposed to sound kind of like the names of the Returned. I played with various different ways for his powers to manifest. I liked the idea of him driving those who hold him to kill anyone nearby. It seemed to work with the concepts that have come before—a kind of unholy, sentient mix of Stormbringer and the One Ring.

The strangest thing about him is the idea that his form isn’t that important. The sheath is like a binding for him, keeping his power contained. So drawing him out isn’t like drawing a regular weapon, but rather an unleashing of a creature who has been kept chained.

Once that creature is unleashed, he becomes a weapon—even if he’s unleashed only a little bit. The sheath itself turns into a weapon, twisting those around it. You don’t need to stab someone with Nightblood to kill them; smashing them on the back with the sheath works just as well. It will crunch bones, but beyond that, merely touching them with the sheath when the smoke is leaking can be deadly.

My editor tried to take out the shot of the final man, slumping back but remaining kneeling, staring up into the sky with Nightblood rammed through his chest and propping him up from behind. But I think it’s one of the more powerful ones in the book, so I fought for it. (He didn’t think it was realistic that the body would just remain there kneeling.)



