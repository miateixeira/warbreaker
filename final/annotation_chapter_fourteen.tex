\section*{Annotations}

\subsection*{Lightsong and Blushweaver}

This is another of the scenes I revised heavily to make the conversation between Lightsong and Blushweaver more snappy. I work very hard in the beginning of the book to establish their personalities and their dialogue, and so the first few chapters were revised more heavily than the later ones. Also, my editor thought that the later ones were already amusing enough; it was the beginning ones that he wanted to have a little more zip.

Their conversation about the weather (playing off the one between Lightsong and Scoot) is one of my favorites from the book. I like how it’s able to show some worldbuilding through the theology of the religion, give a strong dose of character through the different ways that Lightsong and Blushweaver talk about the weather and their desires for how it should go, and all the while be snappy and amusing. The line about serving followers as food is a little cheap, though. Sorry.

\subsection*{Siri Enters and Sees Returned}

Just a little note here. Returned live for eight days without a Breath, though the week is seven days long in this world. Why? Well, I figured that they’d need an extra day as leeway. On day seven, they start to grow weak and sluggish. If they don’t consume a Breath, their body will consume their own on the eighth day of their life, and they’ll die again.

In some parts of this world, Returned aren’t worshipped, but instead seen as something akin to vampires. They draw in Breath to survive, and need a supply of people to feed off of. They tend to wear black, since it’s the most powerful color for draining to Awaken things.

Oh, and since we’re on to random notes, I want to mention that I’m not intending Siri to ever betray who she is through the reversals of this book. When I say that she and Vivenna are switching places, I don’t mean that they’ll start acting like each other. Siri will always be the type who likes to feel the rain on her face and walk barefoot in the grass. Vivenna will probably always be the type who restrains herself from those kinds of activities.

My intention was to have them remain who they are, but still progress and learn to fill one another’s roles.

\subsection*{Vivenna Enters the Court}

Color harmonics are one of the things in this book that, I think, have some very interesting philosophical implications. I’ve always been fascinated by the concept of perfect pitch. Pitches and tones are an absolute; music isn’t just something we humans devise and construct out of nothing. It’s not arbitrary. Like mathematics, music is based on principles greater than human intervention in the world. Someone with perfect pitch can recognize pure tones, and they exist outside of our perception and division of them. (Unlike something like our appreciation of other kinds of art, which is dealing with things that are far more subjective.)

However, I wondered if—perhaps—there are perfect steps of colors just like there are perfect tones, with color fifths, sevenths, and chords and the like. In our world, nobody has the ability to distinguish these things—but what if there were someone who could? Someone who could tell something innate about color that isn’t at all subjective?

I’m not sure if I explained that right, but it intrigued me enough to become part of this book.

\subsection*{Different Viewpoints in the Same Chapter}

In the \textit{Mistborn} books, most of the characters were either involved in the same plotline or separated from one another by distance. I missed being able to do what I did in \textit{Elantris}, where I would show an event from the perspectives of characters who were involved in very different storylines.

The characters in \textit{Warbreaker} are a little more focused on the same things, and are tied together by plots, but they’re also very separate. (At least at the beginning.) It’s fun for me as a writer to be able to show Lightsong, Vivenna, and Siri all attending the same event, but drawing very different experiences from it.




