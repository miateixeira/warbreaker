\section*{Annotations}

\subsection*{Similarities Between \textit{Warbreaker} and \textit{Elantris}}

And finally, we arrive at my personal favorite character in the book. Lightsong the Bold, the god who doesn’t believe in his own religion.

I had the idea for Lightsong a number of years ago. My first book, \textit{Elantris}, dealt with the concept of men who were made gods. However, in that book, we never actually get to see men living \textit{as} gods. The gods have lost their powers and have been locked away.

This time I wanted to tell a different story, a story about what it is like to live as a member of a pantheon of deities. Yet I didn’t want them to be too powerful. Or even powerful at all.

I realize that there is some resonance here with \textit{Elantris}. I hope that the concepts don’t seem too much alike. What I wanted to do with this story was look at some of the same ideas in \textit{Elantris}, but turn them about completely. Instead of dealing with gods who had fallen, I wanted to look at gods at the height of their political power. Instead of dealing with people who were ridiculously powerful, I wanted gods who were more about prophecy and wisdom.

I made it so that the Returned couldn’t remember their old lives as a way to distinguish them from the Elantrians. However, I can’t help the fact that the ideas had the same (yet opposite) seed. But I’m confident that there’s plenty of room in the idea to explore it in a different direction, and I think this book comes out feeling very much its own novel.

\subsection*{First Line and Lightsong’s Origins}

Lightsong’s character came from a one-line prompt I had pop into my head one day. “Everyone loses something when they die and Return. An emotion, usually. I lost fear.”

Of course, it changed a \textit{lot} from that one line. Still, I see that as the first seed of his character. The idea of telling a story about someone who has died, then come back to life, losing a piece of himself in the return intrigued me.

The other inspiration for him was my desire to do a character who could fit into an Oscar Wilde play. I’m a big fan of Wilde’s works, particularly the comedies, and have always admired how he can have someone be glib and verbally dexterous without coming across as a jerk. Of course, a character like this works differently in a play than in a book. For a story to be epic, you need depth and character arcs you don’t have time for in a play.

So, think of Lightsong as playing a part. When he opens his mouth, he’s usually looking for something flashy to say to distract himself from the problems he feels inside. I think the dichotomy came across very well in the book, as evidenced by how many readers seem to find him to be their favorite character in the novel.

\subsection*{Llarimar}

Llarimar is based on a friend of mine, Scott Franson. Back when I was working on \textit{Hero of Ages}, my local church group had a service auction for the local food bank. The idea was that church members would offer up services—like a car wash, or some baked cookies, or something like that—and then we’d all get together and bid cans of food for them.

Well, I offered up for auction naming rights in one of my books. The idea being that if you won the auction, you’d get a character named after you and based on you. It was a big hit, as you might imagine, and ended up going for several hundred cans of food. The guy who won was Aaron Yeoman. (And you can see him in \textit{The Hero of Ages} as Lord Yomen.)

Well, the other major bidder on that was Scott. He’s a fantasy buff, a big fan of classic works like Tolkien and Donaldson. (Though he reads pretty much everything that gets published.) He really wanted the naming rights, but I think he let Aaron have it, as Aaron was very excited and vocal about wanting to win.

About a year later, I discovered that Scott, being the kind soul he was, paid for Aaron’s cans himself and donated them on the younger man’s behalf. I was touched by this, so I decided to put Scott into \textit{Warbreaker}. It happened there was a very good spot for him, as I’d already planned Llarimar to have a very similar personality to Scott.

I decided that Franson wouldn’t work for the name. (Though you do see that one pop up in \textit{The Hero of Ages} as a nod to Scott as well.) Instead, I used Scott’s nickname, Scoot. I thought it worked pretty well, as it’s only one letter off from his first name, and his brother claims that they always used to call him that.

So, there you are, Scott. Thanks for being awesome.

\subsection*{Lightsong Feeds on the Child}

Why a child? It doesn’t much matter, truthfully. An adult, or even someone elderly, could provide a Breath that would keep a god alive.

But the Breaths of those who are aged aren’t as vigorous as those of those who are young. If Lightsong were given one of those to feed on, he’d survive for another week—but he wouldn’t feel as vibrant or alive as he does after feeding on the child’s Breath.

The people of Hallandren are faithful. Even if Lightsong himself doesn’t believe, they do, and they want to provide the best for him. Hence they use children. Old enough to know what they are doing, yet young enough to give a powerful, vibrant Breath to their god.




