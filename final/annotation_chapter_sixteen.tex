% \section{Annotation Chapter Sixteen}
\section*{Annotations}

\subsection*{Lightsong Listens to the Priests Discuss War}

Is this an antiwar novel? I’m not sure, honestly. I didn’t sit down to write one, certainly. I rarely try to interject messages into my books, though sometimes they worm their way in. (The Alcatraz books are particularly bad about this.)

A war here would be a bad thing. Idris and Hallandren shouldn’t be involved in trying to kill one another. But am I, myself, antiwar? Again, I don’t know how to answer that.

Is anyone prowar? War is a terrible, terrible thing. Sometimes it’s necessary, but that doesn’t make it any less terrible. I’m no great political thinker. In fact, being a novelist has made me very bad at talking about political topics. Because I spend so much time in the heads of so many different characters, I often find myself sympathizing with wildly different philosophies. I like to be able to see how a person thinks and why they believe as they do.

I didn’t mean this to be a book about the Iraq war—not at all. But war is what a lot of people are talking about, and I think it’s wise to be cautionary. War should never be entered into lightly. If you ask me if the Iraq war was a good idea, you’ll probably find me on both sides of the argument. (Though I certainly don’t like a lot of aspects about it, particularly how we entered it.)

Regardless, this isn’t a book about anything specific. It’s a story, a story told about characters. It’s about what they feel, what they think, and how their world changes who they are.

As a very, very wise man once said, “I don’t mind if my books raise questions. In fact, I like it. But I never want to give you the answers. Those are yours to decide.” —Robert Jordan. (FYI, that’s not quoted exactly. I can’t even quote myself exactly, let alone other people.)

\subsection*{The High Priest Tells Siri She Needs to Produce an Heir}

Note that in a previous section where I said that I couldn’t delve as deeply into Siri’s plot in this book as I could have in one where there was only one viewpoint character, I didn’t mean that I didn’t intend to give her a lot of political intrigue and plot twisting. I only meant that I decided it was best to keep things a little more focused for her, rather than adding a lot of subplots.

I’ve been wanting to do a story like this one, with a woman sent to marriage in a politically hostile country, since I wrote \textit{Elantris}—where Sarene arrived and found out her wedding couldn’t happen. Again, this is an attempt to turn in a new direction for me, but the inspiration is the same. Sarene arrived and found that her fiancé had died and the court didn’t care about her. Siri arrives and \textit{does} get married, then has far too many people paying attention to her.

\subsection*{Siri and Lightsong Interact}

This chapter has our first real melding of several viewpoints. In a way, it’s a focus chapter for that reason. All four viewpoint characters, who have been off doing other things, congregate here, meeting and mixing. Lightsong and Siri, whose plotlines influence one another a fair amount, sit and talk for the first time. Vivenna and Vasher, who are far more intertwined through the story, meet eyes for the first time.

Vasher shouldn’t have brought Nightblood. But he’s always a little afraid to leave the sword alone for too long. That can have . . . consequences.

Anyway, it was good to be able to show an interaction between two of the viewpoint characters in the form of Siri and Lightsong. This lets us see how Siri acts through the eyes of another, and I think this scene here is one of the first where we really get to see into Lightsong’s soul.



