\section{Chapter Twenty-Eight}

The meeting is set, my lady,” Thame said. “The men are eager. Your work in T’Telir is gaining more and more notoriety.”

Vivenna wasn’t sure what she thought of that. She sipped her juice. The lukewarm liquid was addictively flavorful, although she wished for some Idrian ice.

Thame looked at her eagerly. The short Idrian was, according to Denth’s investigations, trustworthy enough. His story of being “forced” into a life of crime was a tad overstated. He filled a niche in Hallandren society—he acted as a liaison between the Idrian workers and the various criminal elements.

He was also, apparently, a staunch patriot. Despite the fact that he tended to exploit his own people, particularly newcomers to the city.

“How many will be at the meeting?” Vivenna asked, watching traffic pass on the street out beyond the restaurant’s patio gate. “Over a hundred, my lady,” Thame said. “Loyal to our king, I promise. And they’re influential men, all of them—for Idrians in T’Telir, that is.”

Which, according to Denth, meant that they were men who wielded power in the city because they could provide cheap Idrian workers and could sway the opinion of the underprivileged Idrian masses. They were men who, like Thame, thrived because of the Idrian expatriates. A strange duality. These men had stature among an oppressed minority, and without the oppression, they would be powerless.

\textit{Like Lemex,}~she thought,~\textit{who served my father—even seemed to respect and love him—all the while stealing every bit of gold he could lay his hands on.}

She leaned back, wearing a white dress with a long pleated skirt that rippled and blew in the wind. She tapped the side of her cup, which caused a serving man to refill her juice. Thame smiled, taking more juice as well, though he looked out of place in the fine restaurant.

“How many are there, you suppose?” she asked. “Idrians in the city, I mean.”

“Perhaps as many as ten thousand.”

“That many?”

“Trouble on the lower farms,” Thame said, shrugging. “It’s hard, sometimes, living up in those mountains. Crops fail, and what do you have? The king owns your land, so you can’t sell. You need to pay your levies~.~.~.”

“Yes, but one can petition in the case of disaster,” Vivenna said.

“Ah, my lady, but most of these men are several weeks’ travel from the king. Should they leave their families to make a petition, when they fear their loved ones will starve during the weeks it will take to bring food from the king’s store house if they are successful? Or do they travel the much shorter distance down to T’Telir? Take work there, loading on the docks or harvesting flowers in the jungle plantations? It’s hard work, but steady.”

\textit{And, in doing so, they betray their people.}

But who was she to judge? The Fifth Vision would define it as haughtiness. Here she sat in the cool shade of a canopy, enjoying a nice breeze and expensive juice while other men slaved to provide for their families. She had no right to sneer at their motivations.

Idrians shouldn’t have to seek for work in Hallandren. She didn’t like to admit fault in her father, yet his was not a bureaucratically efficient kingdom. It consisted of dozens of scattered villages with poor highways that were often blocked by snows or rockslides. In addition, he was forced to expend a lot of resources keeping his army strong in case of a Hallandren assault.

He had a difficult job. Was that a good enough excuse for the poverty of her people who had been forced to flee their homeland? The more she listened and learned, the more she realized that many Idrians had never known anything like the idyllic life she’d lived in her lovely mountain valley.

“Meeting is three days hence, my lady,” Thame said. “Some of these men are hesitant after Vahr and his failure, but they will listen to you.”

“I will be there.”

“Thank you.” Thame rose—bowed, despite the fact that she’d asked him not to draw attention to her—and withdrew.

Vivenna sat and sipped her juice. She felt Denth before he arrived. “You know what interests me?” he said, taking the seat Thame had been using.

“What?”

“People,” he said, tapping an empty cup, drawing the serving man back over. “People interest me. Particularly people who don’t act like they’re supposed to. People who surprise me.”

“I hope you aren’t talking about Thame,” Vivenna asked, raising an eyebrow.

Denth shook his head. “I’m talking about you, Princess. Wasn’t too long ago that—no matter what or who you looked at—you had a look of quiet dis pleasure in your eyes. You’ve lost it. You’re starting to fit in.”

“Then that’s a problem, Denth,” Vivenna said. “I don’t want to fit in. I hate Hallandren.”

“You seem to like that juice all right.”

Vivenna set it aside. “You’re right, of course. I shouldn’t be drinking it.”

“If you say so,” Denth said, shrugging. “Now, if you were to ask the mercenary—which, of course, nobody ever does—he might mention that it’s~\textit{good}~for you to start acting like a Hallandren. The less you stand out, the less likely people are to connect you to that Idrian princess hiding in the city. Take your friend Parlin.”

“He looks like a fool in those bright colors,” she said, glancing across the street toward where he and Jewels were chatting as they watched the escape route.

“Does he?” Denth said. “Or does he just look like a Hallandren? Would you hesitate at all if you were in the jungle and saw him put on the fur of a beast, or perhaps shroud himself in a cloak colored like fallen leaves?”

She looked again. Parlin lounged against the side of a building much like street toughs his age she’d seen elsewhere in the city.

“You both fit better here than you once did,” Denth said. “You’re learning.”

Vivenna looked down. Some things in her new life were actually starting to feel natural. The raids, for instance, were becoming surprisingly easy. She was also growing used to moving with the crowds and being part of an underground element. Two months earlier, she would have been indignantly opposed to dealing with a man like Denth, simply because of his profession.

She found it very difficult to reconcile herself to some of these changes. It was growing harder and harder to understand herself, and to decide what she believed.

“Though,” Denth said, eyeing Vivenna’s dress, “you might want to think about switching to trousers.”

Vivenna frowned, looking up.

“Just a suggestion,” Denth said, then gulped down some juice. “You don’t like the short Hallandren skirts, but the only decent clothing we can buy you that are ‘modest’ are of foreign make—and that makes them expensive. That means we have to use expensive restaurants, lest we stand out. That means you have to deal with all of this terrible lavishness. Trousers, however, are modest~\textit{and}~cheap.”

“Trousers are~\textit{not}~modest.”

“Don’t show knees,” he said.

“Doesn’t matter.”

Denth shrugged. “Just giving my opinion.”

Vivenna looked away, then sighed quietly. “I appreciate the advice, Denth. Really. I just~.~.~. I’m confused by a lot of my life lately.”

“World’s a confusing place,” Denth said. “That’s what makes it fun.”

“The men we’re working with,” Vivenna said. “They lead the Idrians in the city but exploit them at the same time. Lemex stole from my father but still worked for the interests of my country. And here I am, wearing an overpriced dress and sipping expensive juice while my sister is being abused by an awful dictator and while this wonderful, terrible city prepares to launch a war on my homeland.”

Denth leaned back in his chair, looking out over the short railing toward the street, watching the crowds with their colors both beautiful and terrible. “The motivations of men. They never make sense. And they always make sense.”

“Right now,~\textit{you}~don’t make sense.”

Denth smiled. “What I’m trying to say is that you don’t understand a man until you understand what makes him do what he does. Every man is a hero in his own story, Princess. Murderers don’t believe that they’re to blame for what they do. Thieves, they think they deserve the money they take. Dictators, they believe they have the right—for the safety of their people and the good of the nation—to do whatever they wish.”

He stared off, shaking his head. “I think even Vasher sees himself as a hero. The truth is, most people who do what you’d call ‘wrong’ do it for what they call ‘right’ reasons. Only mercenaries make any sense. We do what we’re paid to do. That’s it. Perhaps that’s why people look down on us so. We’re the only ones who don’t pretend to have higher motives.”

He paused, then met her eyes. “In a way, we’re the most honest men you’ll ever meet.”

The two of them fell silent, the crowd passing by just a short distance away, a river of flashing colors. Another figure approached the table. “That’s right,” Tonk Fah said, “but, you forgot to mention that in addition to being honest, we’re also clever. And handsome.”

“Those both go without saying,” Denth said.

Vivenna turned. Tonk Fah had been watching from nearby, ready to provide backup. They were letting her start to take the lead in some of the meetings. “Honest, perhaps,” Vivenna said. “But I certainly~\textit{hope}~that you’re not the most handsome men I’ll ever meet. Are we ready to go?”

“Assuming you’re finished with your juice,” Denth said, smirking at her.

Vivenna glanced at her cup. It was~\textit{very}~good. Feeling guilty, she drained the juice.~\textit{It would be a sin to waste it,}~she thought. Then she rose and swished her way from the building, leaving Denth—who now handled most of the coins—to settle the bill. Outside on the street, they were joined by Clod, who’d been given orders to come if she screamed for help.

She turned, looking back at Tonk Fah and Denth. “Tonks,” she said. “Where’s your monkey?”

He sighed. “Monkeys are boring anyway.”

She rolled her eyes. “You lost~\textit{another}~one?”

Denth chuckled. “Get used to it, Princess. Of all the happy miracles in the universe, one of the greatest is that Tonks has never fathered a child. He’d probably lose it before the week was out.”

She shook her head. “You may be right,” she said. “Next appointment. D’Denir garden, right?”

Denth nodded.

“Let’s go,” she said, walking down the street. The others trailed behind, picking up Parlin and Jewels on the way. Vivenna didn’t wait for Clod to force a way through the crowd. The less she depended on that Lifeless, the better. Moving through the streets really wasn’t that difficult. There was an art to it—one moved with a crowd, rather than trying to swim against its flow. It wasn’t long before, Vivenna at the front, the group turned off into the wide grassy field that was the D’Denir garden. Like the crossroads square, this place was an open space of green life set among the buildings and colors. Yet, here no flowers or trees broke the landscape, nor did people bustle about. This was a more reverent place.

And it was filled with statues. Hundreds of them. They looked much like the other D’Denir in the city—with their oversized bodies and heroic poses, many tied with colorful cloths or garments. These were some of the oldest statues she had seen, their stone weathered from years spent enduring the frequent T’Telir rainfalls. This group was the final gift from Peacegiver the Blessed. The statues had been made as a memorial to those who had died in the Manywar. A monument and a warning. So the legends said. Vivenna couldn’t help thinking that if the people really did honor those that had fallen, they wouldn’t dress the statues up in such ridiculous costumes.

Still, the place was far more serene than most in T’Telir, and she could appreciate that. She walked down the steps onto the lawn, wandering between the silent stone figures.

Denth moved up beside her. “Remember who we’re meeting?”

She nodded. “Forgers.”

Denth eyed her. “You all right with this?”

“Denth, during our months together I’ve met with thief lords, murderers, and—most frighteningly—mercenaries. I think I can deal with a couple of spindly scribes.”

Denth shook his head. “These are the men who~\textit{sell}~the documents, not the scribes who do the work. You won’t meet more dangerous men than forgers. Within the Hallandren bureaucracy, they can make anything seem legal by putting the right documents in the right places.”

Vivenna nodded slowly.

“You remember what to have them make?” Denth asked.

“Of course I do,” she said. “This particular plan was~\textit{my}~idea, remember?”

“Just checking,” he said.

“You’re worried that I’ll mess things up, aren’t you?”

He shrugged. “You’re the leader in this little dance, Princess. I’m just the guy who mops the floor afterward.” He eyed her. “I hate mopping up blood.”

“Oh, please,” she said, rolling her eyes, walking faster and leaving him behind. As he fell back, she could hear him talking to Tonk Fah. “Bad meta phor?” Denth asked.

“Nah,” Tonk Fah said. “It had blood in it. That makes it a good meta phor.”

“I think it lacked poetic style.”

“Find something that rhymes with ‘blood’ then,” Tonk Fah suggested. He paused. “Mud? Thud? Uh~.~.~. tastebud?”

\textit{They sure are literate, for a bunch of thugs,}~she thought.

She didn’t have to go far before she spotted the men. They waited beside the agreed meeting place—a large D’Denir with a weathered axe. The group of people were having a picnic and chatting among themselves, a picture of harmless innocence.

Vivenna slowed.

“That’s them,” Denth whispered. “Let’s go sit beside the D’Denir across from them.”

Jewels, Clod, and Parlin hung back while Tonk Fah strolled away to watch the perimeter. Vivenna and Denth approached the statue near the forgers. Denth spread out a blanket for her, then stood to the side, as if he were a manservant.

One of the men beside the other statue looked across as Vivenna sat down; then he nodded. The others continued to eat. The Hallandren underground’s penchant for working in broad daylight still unnerved Vivenna, but she supposed it had advantages over skulking about at night.

“You want some work commissioned?” the forger closest to her asked, just loudly enough that Vivenna could hear. It almost seemed part of his conversation with his friends.

“Yes,” she said.

“It costs.”

“I can pay.”

“You’re the princess everyone is talking about?”

She paused, noticing Denth’s hand leisurely going to his sword hilt.

“Yes,” she said.

“Good,” the forger said. “Royalty always seems to know how to handle itself. What is it you desire?”

“Letters,” Vivenna said. “I want them to appear as if they were between certain members of the Hallandren priesthood and the king of Idris. They need to have official seals and convincing signatures.”

“Difficult,” the man said.

Vivenna pulled something from her dress pocket. “I have a letter written in King Dedelin’s hand. It has his seal on the wax, his signature at the bottom.”

The man seemed intrigued, though she could only see the side of his face. “That makes it possible. Still hard. What do you want these documents to prove?”

“That these particular priests are corrupt,” Vivenna said. “I have a list on this sheet. I want you to make it look like they’ve been extorting Idris for years, forcing our king to pay outrageous sums and make extreme promises in order to prevent war. I want you to show that Idris doesn’t want war and that the priests are hypocrites.”

The man nodded. “Is that everything?”

“Yes.”

“It can be done. We’ll be in touch. Instructions and explanations are on the back of the paper?”

“As requested,” Vivenna said.

The group of men stood, a servant moving forward to pack up their lunch. As he did so, he let a napkin blow in the wind, then rushed over and picked it up, grabbing Vivenna’s paper too. Soon, all of them were gone.

“Well?” Vivenna asked, looking up.

“Good,” Denth said, nodding to himself. “You’re becoming an expert.”

Vivenna smiled, settling back on her blanket to wait. The next appointment consisted of a group of thieves who had stolen—at Vivenna and Denth’s request—various goods from the war offices in the Hallandren bureaucratic building. The documents were of relatively little import themselves, but their absence would cause confusion and frustration.

That appointment wasn’t for a few hours, which meant she could enjoy some time relaxing on the lawn, away from the unnatural colors of the city. Denth seemed to sense her inclination, and he sat down, leaning back against the side of the statue’s bare pedestal. As Vivenna waited, she saw that Parlin was over talking to Jewels again. Denth was right; though his clothing looked ridiculous to her, that was because she~\textit{knew}~him as an Idrian. Looking at him more objectively, she saw that he fit in remarkably well with other young men in the city.

\textit{That’s well and good for him,}~Vivenna thought with annoyance, looking away.~\textit{He can dress as he wishes—he doesn’t have to worry about his neckline or hemline.}

Jewels laughed. It was almost a snort of derision, but there was~\textit{some}~mirth in it. Vivenna looked back immediately, watching Jewels roll her eyes at Parlin, a self-effacing smirk on his face. He knew he’d said something wrong. He didn’t know what. Vivenna knew him well enough to read the expression and to know that he’d just smile and go along with it.

Jewels saw his face, then laughed again.

Vivenna gritted her teeth. “I should send him back to Idris,” she said.

Denth turned, looking down at her. “Hum?”

“Parlin,” she said. “I sent my other guides back. I should have sent him too. He serves no function.”

“He’s quick at adapting to situations,” Denth said. “And he’s trustworthy. That’s good enough reason to keep him.”

“He’s a fool,” Vivenna said. “Has trouble understanding half of what goes on around him.”

“He’s not got the wit of a scholar, true, but he seems to instinctively know how to blend in. Besides, we can’t all be geniuses like you.”

She glanced at Denth. “What does that mean?”

“It means,” Denth said, “that you shouldn’t let your hair change colors in public, Princess.”

Vivenna started, noticing that her hair had shifted from a still, calm black to the red of frustration.~\textit{Lord of Colors!}~she thought.~\textit{I used to be so good at controlling that. What is happening to me?}

“Don’t worry,” Denth said, settling back. “Jewels has~\textit{no}~interest in your friend. I promise you.”

Vivenna snorted. “Parlin? Why should I care?”

“Oh, I don’t know,” Denth said. “Maybe because you and he have been practically engaged since you were children?”

“That’s completely untrue,” Vivenna said. “I’ve been engaged to the God King since before my birth!”

“And your father always wished you could marry the son of his best friend instead,” Denth said. “At least, that’s what Parlin says.” He eyed her with a smirk.

“That boy talks too much.”

“Actually, he’s usually rather quiet,” Denth said. “You have to pry to get him to talk about himself. Either way, Jewels has other ties. So stop your worrying.”

“I’m not worried,” Vivenna said. “And I’m~\textit{not}~interested in Parlin.”

“Of course not.”

Vivenna opened her mouth to object, but she noticed Tonk Fah wandering over, and didn’t want~\textit{him}~to join this discussion as well. She snapped her jaw shut as the hefty mercenary arrived.

“Flood,” Tonk Fah said.

“Hum?” Denth asked.

“Rhymes with blood,” Tonk Fah said. “Now you can be poetic. Flood of Blood. It is a nice visual image. Far better than tastebud.”

“Ah, I see,” Denth said flatly. “Tonk Fah?”

“Yes?”

“You’re an idiot.”

“Thanks.”

Vivenna stood up and began to walk through the statues, studying them—if only to escape having to watch Parlin and Jewels. Tonk Fah and Denth trailed along behind at a comfortable distance, keeping a watchful eye.

There was a beauty to the statues. They weren’t like the other kinds of art in T’Telir—flashy paintings, colorful buildings, exaggerated clothing. The D’Denir were solid blocks that had aged with dignity. The Hallandren, of course, did their best to destroy this with the scarves, hats, or other colorful bits they tied on the stone memorials. Fortunately, there were too many in this garden for all to be decorated.

They stood, as if on guard, somehow more solid than much of the city. Most stared up into the sky or looked straight ahead. Each one was different, each pose distinct, each face unique.~\textit{It must have taken decades to create all of these,}~she thought.~\textit{Perhaps that’s where the Hallandren got their penchant for art.}

Hallandren was such a place of contradictions. Warriors to represent peace. Idrians who exploited and protected each other at the same time. Mercenaries who seemed to be among the best men she had ever known. Bright colors that created a kind of uniformity.

And, over it all, BioChromatic Breath. It was exploitive, yet people like Jewels saw giving up their Breath as a privilege. Contradictions. The question was, could Vivenna afford to become another contradiction? A person who bent her beliefs in order to see that they were preserved?

The Breaths~\textit{were}~wonderful. It was more than just the beauty or the ability to hear changes in sound and sense intrinsically the distinct hues of color. It was more even than the ability to sense life around her. More than the sounds of the wind and the tones of people talking, or her ability to feel her way through a group of people and move easily with the motions of a crowd.

It was a connection. The world around her felt~\textit{close}. Even inanimate things like her clothing or fallen twigs felt near to her. They were dead, yet seemed to yearn for life again.

She could give it to them. They remembered life and she could Awaken those memories. But what good would it do to save her people if she lost herself?

\textit{Denth doesn’t seem lost,}~she thought.~\textit{He and the other mercenaries can separate what they believe from what they are forced to do.}

In her opinion, that was why people regarded mercenaries as they did. If you divorced belief from action, then you were on dangerous ground.

\textit{No,}~she thought.~\textit{No Awakening for me.}

The Breath would remain untapped. If it tempted her too much further, she would give the lot away to somebody who had none.

And become a Drab herself.

