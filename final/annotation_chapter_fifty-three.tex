\section{Annotation Chapter Fifty-Three}

\subsection*{Vivenna Wonders Where Vasher Is}

This is just a quick scene to update you on what Vivenna is doing. It’s approaching morning, so she’s been sitting and stewing all night. I felt we needed to at least get a glimpse of her here. If you can’t tell, the avalanche has begun.

\subsection*{Treledees Takes Siri}

At this point, you’re supposed to be confused at whose motivations are what. I’m not sure what you’ll be thinking of the priests at this point in the story. Suffice it to say that Denth’s men were in control of Siri’s chamber, but he left them once he got Vasher. While he’s been torturing Vasher, however, Treledees and his forces seized Siri’s room back and killed the guards out front. Now they’ve pulled her away.

Tonk Fah wasn’t there, as you’ll soon discover. He’s guarding the door to the room where Vasher and Denth are. He’s just outside, and he has orders not to let Denth get interrupted. When things get out of hand in the palace, however, he goes in to inform Denth of what’s going on. We’ll see him there in just a little bit.

\subsection*{Old Chaps}

I love having random little viewpoints like these in books. I don’t do them often, usually just once or twice a book. But I was excited to write this one, as Chaps has a very interesting way of thinking. Dance, dance, dance. I didn’t plan him into the book specifically; I simply wrote this scene as it arrived and I knew someone had to fetch Nightblood. I’m always pleased when a little glimpse like this gives us such a distinctive feel and flavor for a character, though.

Nightblood is better at communicating with people who are mentally unhinged. He can influence them more easily. Really, Denth, you should have known to toss Nightblood someplace far deeper than the shallow bay.

\subsection*{Vasher Tortured}

Moshe wanted Vasher naked here, but I felt that keeping him in the white shorts was good enough. We’ve had a lot of nudity in this book, both male and female, and I didn’t want to push it any farther and distract from the discussion here.

Vasher is wrong about Arsteel, by the way. Arsteel \textit{didn’t} need to be killed; Vasher misinterpreted the man’s motives in joining with Denth. It’s unfortunate that the two came to blows, but Arsteel never intended to kill Vasher in the duel, just subdue him and talk some sense into him. (“Sense” as Arsteel saw it. He wasn’t actually right in what he was doing—he didn’t understand Vasher’s reasoning either. All I’m saying is that Arsteel’s motives were, in fact, pure.)

Denth has done a relatively good job keeping Tonk Fah from murdering as often as he used to. Killing is necessary, sometimes, in Denth’s opinion—but there’s no need to go around killing people who don’t need it. He’s managed to rein Tonk Fah in. It’s a slight measure of the good that’s left in Denth.

\subsection*{Vivenna Suits Up and Leaves}

Vivenna is in a similar position to Siri here in these last chapters. Things are getting so dangerous that both women (well, and Lightsong too) are rather out of their elements. However, I knew that I had to have them both involved. It would be incredibly frustrating to read an entire book focused on two characters, then have them get pushed around for the entire climax.

So during my outlining, I made certain to build the story in such a way that they could be useful, even if they’re very much out of their elements. I feel this makes the story more tense in a lot of ways, since they’re forced to deal with things for which they’re completely unprepared.

Here, we have Vivenna sorting through her own emotions and finding enough determination left to go out and do something. This is an important moment for her, even though she doesn’t realize it. This is the moment where she takes her first real step toward becoming her new self.

\subsection*{Lightsong Is in Prison}

Lightsong here is not giving up, which I think is very appropriate for his character. He still has his sense of confidence. In a way, the priest who kills Blushweaver is right. Lightsong does still see it as a game. His life in the court has taught him that things aren’t ever dangerous for him. This is all just politics, and a big piece of him feels that he’s just on an adventure. He finds it exciting.

That’s why Llarimar blows up at him. It’s not Lightsong’s fault—he’s been trained by the last five years to look at life this way. But here, the games have ended, and it’s suddenly become very real and very dangerous. Llarimar is the type who is very calm headed until you just push him past his snapping point, and then he loses it. It’s hard to get him there, but the current situation is enough.

By the way, this is only the second time Lightsong has landed them both in prison. The first time happened a good twenty years earlier, even if Llarimar has never quite gotten over it. It involved a whole lot of drinking. (Llarimar, already then an acolyte priest of the Iridescent Tones, had never gotten “good drunk” as Lightsong called it at the time. So, he took him out on he day before his ordination as a full priest and got him solidly, rip-roaringly drunk. The embarrassment of what they did, landing themselves in prison for trying to bust into the Court of Gods while wearing only their underclothing, nearly got Llarimar tossed out of the priesthood. Needless to say, he didn’t make full priest the next day. It was three years before he was allowed to apply for ordination again.)

\subsection*{Blushweaver’s Death}

My editor was uncomfortable with the way this happened—he felt that the motivations for the killers weren’t solid enough. I tried to put a little more in, which placated Moshe, but I always felt that they \textit{were} solid.

Bad guys in books often do stupid things, and it annoys me. They’re often not allowed to do the smartest things they could because it would ruin the plot. I wanted them to do the smart thing here, and I felt that the smartest thing was to kill Blushweaver. Just threatening her wouldn’t have worked with Lightsong; he refused to take things seriously. A simple threat would have earned them mockery and frustration. So, not knowing that he loved her, they killed Blushweaver to show how serious they were. Then they grabbed Llarimar, not intending to actually kill him, as they knew he was the best bargaining chip against Lightsong they had. (If he hadn’t talked, they’d have started cutting off Llarimar’s fingers.)

The brutality of that moment of Blushweaver’s throat being slit is supposed to be a major reversal in tone for Lightsong’s sections. I hope that it worked for you; I think I laid the proper groundwork that this story could have things like that happen in it. I think I justified the motivations of the killers enough.

The games are over.



