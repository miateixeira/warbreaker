\chapter{Dedication}

This book is dedicated to my dear wife, Emily. I started writing this book when we were dating, and worked on it all through our engagement. I even took it on our honeymoon to Hawaii---though I didn't actually get any writing done on it then.

When I proposed to her, I wrote out a little poem in the form of a proposal that I said I'd use as the \textit{Alcatraz Versus the Evil Librarians} dedication. She didn't want it to appear in the book, however, because a live dedication in a novel would have embarassed her.

However, I asked if she'd mind having \textit{Warbreaker} dedicated to her, and she was excited about that. You may not know that when we were married, I commissioned a large batch of swords---inscribed with names from my books---and gave them to my closest friends. I named Emily's sword Lightsong, and she carried it around at the reception. (Mine is named Dragonsteel.)

So, anyway, this book is very dear to her. It's the first one of mine she had input on during the editing process. And now it's finally published, about three years since the date of our wedding. Ah, how time flies.

\orn

As a side note, when I was a teenager, I dreamed of someday proposing to my wife via a book dedication. Back then, being married and getting published were both very, very distant goals of mine. Lik twin holy grails, shining on the mountain, virtually unobtainable but hoped for nonetheless.

I can still remember thinking of how cool it would be to surprise my soon-to-be-fianc\'ee by walking with her into the bookstore to see if my new book was on the shelves yet. (In my daydream, it was the Cosmic Comics store back in Lincoln, which solf sf/fantasy books. It's the place where I first saw \textit{Eye of the World} on a store shelf, by the way.) I imagined myself walking over and finding it on the shelf during its release week, then calling her over. That would be the first time she'd see the dedication, which would be a proposal to her. Of course, I'd have worked it out with the bookstore owner so that there was a ring taped to the next page.

Ah, ignorance. It was a fond dream. What I didn't realize is that often, there are \textit{years} between the writing of a book and its publication. I didn't really think that Emily would want to wait three years for a proposal, just so that I could surprise her by having it in the front of a published book\ldots

Sometimes, though, it's still amazing to me to look back at that sixteen-year-old version of myself and realize that I've achieved both of those goals. I'm not only a published author, but I'm writing fantasy books as a  full-time job. And I'm not only married, but I'm married to just about the most wonderful woman who's ever lived.

So those things weren't so unobtainable after all. But they're still just as precious as I imagined.
