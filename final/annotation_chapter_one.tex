\section*{Annotations}

\subsection*{Tone}

You can probably guess why I was worried about the transition from the prologue with Vasher to this chapter with Siri. The tone shift is quite dramatic. Actually, one of the things my agent complained a lot about with this book \textit{was} the tone. Not just for this chapter shift, but for the entire book.

In his opinion, there were too many different tone shifts going on. We have Vasher’s plot, which is dark and sometimes violent. We have the Siri plotline, which is romantic and sometimes whimsical. We have Lightsong, whose chapters are glib and smell faintly of an old comedic murder mystery. Then we have Vivenna, whose tone bounces around across \textit{all} of these.

That’s one of the things I like about the book. My agent complained, but I know he likes things more streamlined than I sometimes do. He loved the \textit{Mistborn} books, and I do think they are excellent novels—but they are very focused. The characters are distinctive, but their plots are all centered on many of the same types of goals.

With \textit{Warbreaker}, one of the main things I’m trying to do is contrast it to \textit{Mistborn}. To do something different, something that harkens a little more back to \textit{Elantris}, with its three very different viewpoints.

I want there to be a lot of different tones and feels to this book. It’s part of the theme of the novel—that of vibrant Hallandren and its many wonders. I want it to feel like a lot is going on, and that in different parts of the city, very different stories can be told.

\subsection*{The Origins of Siri and Vivenna}

Back around the year 2000 or 2001 I started writing a book called \textit{Mythwalker}. It was an epic fantasy novel, an attempt to go back to basics in the genre. I’d tried several genre-busting epics (one of which was \textit{Elantris}) that focused on heroes who weren’t quite the standards of the genre. I avoided peasant boys, questing knights, or mysterious wizards. Instead I wrote books about a man thrown into a leper colony, or an evil missionary, or things like that.

I didn’t sell any of those books. (At least, not at first.) I was feeling discouraged, so I decided to write a book about a more standard fantasy character. A peasant boy who couldn’t do anything right, and who got caught up in something larger than himself and inherited an extremely powerful magic.

It was boring.

I just couldn’t write it. I ended up stopping about halfway through—it’s the only book of mine that I never finished writing. It sits on my hard drive, not even spellchecked, I think, half finished like a skyscraper whose builder ran out of funds.

One of the great things about \textit{Mythwalker}, however, was one of the subplots—about a pair of cousins named Siri and Vivenna. They switched places because of a mix-up, and the wrong one ended up marrying the emperor.

My alpha readers really connected with this storyline. After I abandoned the project, I thought about what was successful about that aspect of the novel. In the end, I decided it was just the characters. They \textit{worked}. This is odd because, in a way, they were archetypes themselves.

The story of the two princesses, along with the peasant/royalty swap, is an age-old fairy tale archetype. This is where I’d drawn the inspiration from for these two cousins. One wasn’t trained in the way of the nobility; she was a distant cousin and poor by comparison. The other was heir to her house and very important. I guess the idea of forcing them to switch places struck some very distinct chords in my readers.

Eventually, I decided that I wanted to tell their story, and they became the focus of a budding book in my mind. I made them sisters and got rid of the “accidental switch” plotline. (Originally, one had been sent by mistake, but they looked enough alike that nobody noticed. Siri kept quiet about it for reasons I can’t quite remember.) I took a few steps away from the fairy tale origins, but tried to preserve the aspects of their characters and identities that had worked so well with readers.

I’m not sure why using one archetype worked and the other didn’t. Maybe it was because the peasant boy story is \textit{so} overtold in fantasy, and I just didn’t feel I could bring anything new to it. (At least not in that novel.) The two princesses concept isn’t used nearly as often. Or maybe it was just that with Siri and Vivenna I did what you’re supposed to—no matter what your inspiration, if you make the characters live and breathe, they will come alive on the page for the reader. Harry Potter is a very basic fantasy archetype—even a cliché—but those books are wonderful.

You have to do new things. I think that fantasy \textit{needs} a lot more originality. However, not every aspect of the story needs to be completely new. Blend the familiar and the strange—the new and the archetypal. Sometimes it’s best to rely on the work that has come before. Sometimes you need to cast it aside.

I guess one of the big tricks to becoming a published author is learning when to do which.

\subsection*{Ramblemen}

Ramblemen are more than simple traveling jugglers or storytellers. They’re merchants who specialize in bringing news (for a price) and stories as well as goods and services.

Readers latched onto this word, and I’ve had a lot of people say, “I love that term! Why don’t we get to see a rambleman in the book?”

Because some things in books are just there to hint at the greater world. Sometimes a keen, cool word like that can evoke so much more when used in passing than it would if developed into a side plot or attached to a character.

\subsection*{Idris’s Drabness}

One thing to realize is that the Idrians’ attempts to make their city colorless are more superstition than they are effective. It’s much harder to get colors away from an Awakener than the Idrians think. For instance, black is one of the most powerful colors to use for fueling Awakening—but the Idrians don’t even consider it a color. Their browns and tans would also work for Awakening.

However, a lot of times, the traditions of a culture don’t have much to do with factual reality. The determination to avoid colors grew out of a desire to contrast with Hallandren and their devilish Awakeners. It got taken to the extreme, however, and as the centuries passed, the Idrians grew confused about just what Awakening is and what it can do. Of course, there are some who know—Hallandren isn’t \textit{that} far away. But there’s also a lot of rumor and misinformation.

\subsection*{Mab the Cook}

If it sounds to you like Mab knows a lot about Awakening and Hallandren, then you’ve picked up on something. Mab actually used to live in T’Telir. (She was born in Idris, but ran away during her teens.) During her twenties, she was a courtesan of some repute in the city. She had some fairly high-profile clients—so she was more than just a poor, street-corner prostitute. She fell in love with one of the men, however, and he convinced her to give him her Breath. Then he left her.

As a Drab, she had much more trouble finding work. She’d lost a bit of her sparkle, and whatever she’d used to capture the hearts of men, she’d lost that too. She ended up as a madam, running a much poorer whorehouse, using her old contacts and reputation to get clients.

As soon as she made enough, she bought another Breath and returned to Idris, where she got a job in the king’s kitchens. To this day, she bears a lot of ill will toward the Hallandren upper crust, and Awakeners in particular.

\subsection*{The King and Yarda Discuss Sending Vivenna}

I go back and forth on this scene. Sometimes I think it’s too long. Other times I worry that it’s not long enough.

Through the history of the book, this particular scene inched longer and longer as I tried very hard to explain why a good man like Dedelin would send Siri to die in Hallandren. (And also, I wanted to be sure to explain why he was sure she would die there.) There’s a whole lot of setup going on in this sequence between the king and his general.

And I worry that there should be more. While what they do makes intrinsic sense to me, a lot of readers have been confused about the tactics here. Why is the king doing what he’s doing? Is it really needed? Isn’t there another way? This section is the only answer we get to a lot of those questions, since it’s the one and only scene in the book from Dedelin’s viewpoint.

That said, I think this scene might also be too long. The more space I dedicate to Dedelin, the more readers are going to think that he might be a main character. Some are surprised to read on and find out that the king doesn’t make another appearance in the novel. (Well, okay, he makes one more—but he doesn’t have a viewpoint.) I don’t want to put too much here or have readers focus too much on the tactics of his decision, since really all that matters is that readers understand that Siri has been sent unexpectedly to marry the God King.

I’m still iffy on the scene. Some test readers wanted to see the scene where Dedelin says farewell to Siri. (We skip it; the next scene begins with Siri riding away.) They feel they missed a chapter. But I eventually decided that I needed to keep this beginning flowing quickly, because the longer we spend in Idris, the longer it will take us to get to the real plots in Hallandren. If it weren’t so important to set up Siri and Vivenna ahead of time (so that their reversal has impact), I would have just started the book with Siri arriving in Hallandren.



