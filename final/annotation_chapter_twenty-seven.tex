\section{Annotation Chapter Twenty-Seven}

\subsection*{Bluefingers Avoids Siri, So She Goes to Find Lightsong}

I considered having the men performing the athletics competitions in the court be naked. After all, there’s been so much female nudity in the book so far that it would only be fair to balance it out. . . .

I decided it would be gratuitous. Just because the Greeks competed in the nude doesn’t mean that it would naturally happen everywhere else. Still, thinking of how much it would embarrass Siri almost made me put it in.
% <img draggable="false" role="img" class="emoji" alt="😉" src="https://s.w.org/images/core/emoji/14.0.0/svg/1f609.svg">

The toughest thing to balance about Lightsong was how genuine to make his sense of indolence. His discussion with Siri here is probably the most candid he ever gets in the book in regard to the fact that, in part, he’s just putting on a show with all of his humor and remarks. They’re intended to distract, and are also a subtle commentary on what he thinks of the other gods and the way they’re all treated.

The problem is, unless he really \textit{is} somewhat like he pretends to be, it wouldn’t work at all. His advice here to Siri is based on his perception of the world.

When he first Returned, his initial inclination was to act like this. (I believe he brings that out later in the book.) However, after meeting Calmseer and having a relationship with her (it wasn’t love, not in the traditional sense; more of a sincere mutual respect that turned physical), he spent a lot of time trying to be the god who everyone expected him to be. He failed miserably, and his people were dissatisfied with him. He blames his failure mostly on the other gods, who mocked him for turning into a hypocrite.

So he returned to being Lightsong the indolent, and he sharpened his wit against the others and let loose with as much vengeance as he could muster. The others weren’t offended, however—they just took it as natural that he act that way. We find him several years after that in this book, where he’s just given up on being able to change things.



