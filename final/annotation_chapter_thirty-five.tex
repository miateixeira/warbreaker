\section{Annotation Chapter Thirty-Five}

\subsection*{Vivenna Awakes, Bound by Vasher}

This chapter—with what happens in the latter part of it—is the most dangerous in the book. Dangerous to me as an author, I mean. I love good plot twists, but I worry about leaving them without proper foreshadowing. I’ve never done something as drastic as I have in this book, having a group of sympathetic characters turn out to be working for the wrong side. I hope it succeeds, but I know that if it doesn’t, readers will be very mad. Nothing is sloppier than a book with unearned changes in character motivation.

But we’re not there quite yet. Before that we have the first real interaction between Vivenna and Vasher. He gives her what he likes to think of as the Nightblood test. One nice thing about having a sword that “cannot tempt the hearts of those who are pure” is that when someone like Vivenna touches it, she gets sick. I didn’t want Nightblood to come across as a “one ring” knockoff. He doesn’t turn people’s hearts or corrupt them. However, in order to be able to do his job and fulfill his Command, he needs the ability to determine who is good and who is evil.

This, of course, isn’t an easy thing to determine. In fact, I don’t think it’s a black or white issue for most people. When Nightblood was created, the Breaths infused in him did their best to interpret their Command. What they decided was evil was someone who would try to take the sword and use it for evil purposes, selling it, manipulating and extorting others, that sort of thing. Someone who wouldn’t want the sword for those reasons was determined to be good. If they touch the weapon, they feel sick. If others touch the weapon, their desire to kill and destroy with it is enhanced greatly.

Nightblood himself, unfortunately, doesn’t quite understand what good and evil are. (This is mentioned later in the text.) However, he knows that his master can determine who is good and who is evil—using the sword’s power to make people sick, or through other means. So, he pretty much just lets whoever is holding him decide what is evil. And if the one holding the sword determines—deep within their heart—that they are evil themselves, then they will end up killing themselves with the sword.

Vivenna passes the test, which surprises Vasher. He thought that she’d be the type who would use Nightblood to kill and destroy. (He doesn’t have a high opinion of her, obviously. Of course, that’s partially because he’s let his temper dictate what he thinks.)

\subsection*{Vivenna Escapes}

One of my big worries about the Vivenna sections is that she’ll come off as too weak as a character. That’s a particular danger once we reach these late middle sections, where it’s revealed how much she’s been manipulated. Remember that when you’re reading the Vivenna sections, if she comes off weak compared to Siri, consider their relative circumstances.

Vivenna is put through a \textit{lot} more in this book than Siri is. Why? Well, I felt that as a character, she had a lot more room to grow. In order to do that, however, she needed to have everything knocked out from underneath her. That happens primarily in this chapter and the next few.

But she is not helpless. Even while she’s numbed by the capture and betrayals, she manages to effect not one, but two escapes. She handles herself very well, finally overcoming her problems with Awakening and managing to get her Breath to work for her. (And remember that the more Breath one has, the easier it is to learn to get Commands to work right. That will be important later in the book. . . .)

\subsection*{Vivenna Wanders the Slums, Then Finds the Safe House}

I made one small revision here in this chapter. I added the statue as a reference point. Before, Vivenna just happened to run across the safe house while wandering.

Why the change? It’s just the same thing, right? She happens to wander by the statue, then manages to remember the directions. It’s still a big coincidence when you think about it.

However, it doesn’t \textit{read} like as big a coincidence. Adding in her seeing the statue, then having to work to find her way to the safe house was a way of making it seem, to readers, that it wasn’t just a coincidence. Because there was effort involved, I feel it will read more smoothly and less oddly to most readers. Part of this is because a statue in a city square is easier to notice than a given house on the side of the street, and partially because the discovery can be more gradual this way.

This is part of the smoke and mirrors that a writer uses. Sometimes I worry that explaining these things will ruin the book for readers—but I guess if you were the type it would ruin the magic for, you probably wouldn’t be reading behind-the-scenes annotations in the first place.

\subsection*{Vivenna Realizes That the Mercenaries Are Traitors}

And finally, here we are. The biggest gamble in the book. I went into the novel knowing I was going to do this, and I wrote all along with the intention that Denth and his crew were working against Vivenna’s interests.

As I mentioned in a spoiler section earlier, Tonk Fah is a sociopath, and much of the time when he makes his jokes about hurting people, he’s serious. (The vanishing pets are a subtle clue to this.) He finds the concept of hurting people funny. We laugh because of Denth, who’s running interference and making it seem like they’re just exaggerating to get a laugh.

The death of Lemex is another clue—he was, indeed, immune to disease. (Though not poison, if enough was used.) Anyone with that many Breaths is immune. Another clue is what the mercenaries are doing, riling up the Hallandren to war rather than working to prevent it. Not that Vivenna wanted them to, but through Denth’s manipulations, Siri has all but been forgotten in the face of the work against Hallandren. Of course, Vivenna herself was willing to forget Siri. Not by intent, but because she has always been more focused on Hallandren, and Siri was partially just an excuse.

The fact that Vivenna’s father’s agents are never seen looking for her, the fact that the mercenaries don’t seem to care about money, the way Jewels was frequently gone at the beginning (partially so she could tail Vivenna), and much of what they said and did were supposed to be reinforcement of this moment of betrayal.

All that said, however, I don’t think it’s at all obvious what they are really up to. And that’s why this is a gamble. This twist isn’t an “Ah, I should have seen it!” revelation like the one about the Lord Ruler at the end of \textit{Mistborn}. Instead, it’s a twist that—hopefully—has just enough groundwork underneath it not to seem out of nowhere. I fully expect it to blindside most readers.

\subsection*{Parlin Is Dead}

Parlin was always meant to die here. That’s one of the main reasons I left Vivenna with someone from Idris to be in her team, in fact. (The other reason is that I found it unrealistic that she wouldn’t have \textit{somebody} with her.)

Maybe this is why Parlin never worked as a character, to be honest. I wonder if he was always in my mind as the character who was going to get killed by Tonk Fah, which kept me from giving him enough depth. I’m not sure; I do know that in the book as it stands, he’s probably the biggest component I wish I had time to change. I’m not certain what I could put in his place that wouldn’t distract too much from the plot—and wouldn’t take away from the humor of Denth and the mercenaries—but would still be sympathetic enough that when he dies here, it would be more powerful. But I would have liked to have found something.

Tonk Fah tortured him to death. He wasn’t supposed to, but he got carried away. It was an accident, as Denth claims. (Denth shouldn’t have left Tonks alone with the prisoner to continue the torturing.) Denth came back and found Parlin dead, and was annoyed and frustrated. He left Tonks behind, storming out in anger, and eventually found Jewels and Clod, who were talking to slum contacts and trying to find Vivenna. They came back to regroup.

Meanwhile, Tonks heard Vivenna enter, and knew it wasn’t Denth. He put his Breath into his clothing, then ducked back under the stairs, his lantern extinguished, wondering who had come. He wasn’t terribly surprised to find Vivenna. That was when Denth and Jewels got back and the rest of the situation went down.

I added the corpses of Vivenna’s father’s agents in the last draft, by the way, since I figured I wanted it to be more obvious what had happened to them.



