\section{Annotation Chapter Eleven}

\subsection*{Siri Visits the God King’s Chamber Again}

To be honest, in a perfect world, I’d probably slow this down just a tad. I’d insert another chapter from Siri’s viewpoint with her going to the chambers, the God King watching her, and her being subservient. I wouldn’t do this chapter, where she explodes at him, until their third scene together.

But that would only happen in a book where I don’t have quite so much going on with other viewpoints. My books are already a tad on the long side, as far as the booksellers are concerned. They’d like it if epic fantasy novels shrank down to about 120,000 words (instead of my average of 240,000).

If I’d \textit{really} thought it mattered, I’d have put the extra scene in. The real problem is that since Siri is only one of four major viewpoints, I needed to be careful. If this book were only about her, I could have filled her chapters with more political intrigue and added a lot of subplots. That would have made a slower pacing with the God King work. However, I decided not to go that direction with the book, so I needed instead to make sure the pacing was quicker on the main plot she’s involved in.

\subsection*{Origin of Bluefingers as a Character}

Bluefingers originated, like most ideas for my books, as a character unconnected to any story or world. I wanted to tell a story about a scribe in a palace who was looked down on by the nobility for his simple birth, but who became the hero of the story. I felt that a scribe would make a nice, different kind of viewpoint character.

And maybe I someday will tell a story like that, but the character evolved to be the one who entered this story. He’s much changed from those origins, as you can see, but he’s largely the same person in my mind. And I love the name Bluefingers for a scribe character.



<p></p>
<p>Yes, Bluefingers was also planned as a traitor from the beginning. The whole reversals idea required me to build my shadowy villains quite carefully and deliberately.</p>
<p>Just above, I spoke of the original Bluefingers as a hero. Well, the thing is, that’s how he still sees himself. The heroic Pahn Kahl figure with his fingers in events, ignored by the nobility (or, in this case, the priests) because of his race and position, he was able to manipulate quite a bit of what was going on in the kingdom.</p>
<p>He was the hero trying to free his people. He just took it too far.</p>
<p>Anyway, in this chapter, he’s trying to give Siri a seed of worry and doubt. He’s hoping that if she feels she’s in danger, she’ll trust him more and that will let him do what he needs to. At this point, he’s not sure that he will kill her. It’s more that he’s hoping he’ll be able to manipulate her to in turn manipulate the Idrians in the city. So he wants to make sure Siri sees the Hallandren as her enemies. He can tell that she’s beginning to think her life in the city isn’t all that bad, and he’s worried about that. Idris and Hallandren won’t go to war, in his opinion, if Siri is too content.</p>
<p>However, Denth’s success with Vivenna out in the city (and yes, Bluefingers is the one employing Denth) will eventually convince Bluefingers that he doesn’t need Siri for that role. Unfortunately for her—and for him, in a way—he realizes that if she were seen as having been killed by the Hallandren priests, it would certainly spark a war.</p>
<p></p>



