\section{Annotation Chapter Thirty-Eight}

\subsection*{Lightsong Awakes from More Bad Dreams}

This is the scene in the book where I originally started to turn Lightsong’s dreams a tad darker. As you can see from the final version, I’ve now been doing that from the beginning. All to keep tension up.

Anyway, these dreams he saw—a prison, Scoot, Blushweaver—were there in the original draft. As I’ve said, I’m a planner, and so I had my ending well in mind by this point in the original version of the book. That ending changed in many ways during revision, but it’s kind of surprising how much stayed the same. Sometimes, things just work and you do get them right on the first try.

\subsection*{Lightsong Throws Pebbles to Count Priests}

One of the challenges in writing these sections was that Lightsong could never do anything the “normal” way. He could have simply sent his priests to count at the gates, then come back to him with some figures. But it wouldn’t have felt right.

Despite his protests, Lightsong likes to meddle. He likes to pick at things and be involved. He couldn’t just send someone to count; he had to go count himself. And he had to do so in a properly flamboyant way.

This scene with the pebbles is important for far more than the obvious reasons. Yes, we’re furthering the mystery plots (though this particular one isn’t as important to the overall plot as some others). However, the more important part of this scene is how it shows Lightsong’s progression and growth.

I know what it’s like to finally find something to latch onto, something to drive you and give added purpose to your life. For me, it was writing. For Lightsong, it’s the investigation of the murder.



<p></p>
<p>\subsection*{The Tunnels}</p>
<p>The tunnels become a focus for Lightsong, though the truth is that they’re not as important to the case as he thinks they are. Yes, there are things to be learned from them. Bluefingers has sequestered a large group of mercenaries down in a secure chamber under there. He’s also begun using Pahn Kahl Awakeners (yes, there are some) to Break some of the Lifeless. The tunnels are central to his plot of getting into the God King’s palace at the end of the book and securing it.</p>
<p>But Lightsong doesn’t know any of this, and doesn’t figure out most of it during the course of the book. (It’s left for the reader to infer.) Lightsong’s fixation with the tunnels is driven partially by the visions he’s been seeing at night, which include the tunnels and his discovery of Blushweaver being captured. He’s made a subconscious connection.</p>
<p></p>



