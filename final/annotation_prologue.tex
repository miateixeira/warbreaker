\section*{Annotations}

\subsection*{The Origins of the Prologue}

This began as a first chapter; I only later turned it into the prologue. My worry when I made the change (and it’s still a bit of a worry) was that it was kind of a sneaky way to begin the book. Let me explain.

This novel focuses primarily on Siri, Vivenna, and Lightsong. Vasher, as the fourth viewpoint, is only in there fairly sparsely. True, he drives a lot of what is happening from behind the scenes, but he’s a mysterious figure, and we don’t know a lot about him. This prologue is pretty much the most extensive, lengthy, and in-depth scene we get of him.

Therefore, it’s kind of sneaky to begin the book with him. I did it for a couple of reasons. First off—and this is the most important one—this scene is just a great hook. It shows off the magic system and the setting of the novel (most of the action takes place in T’Telir, even though the first few chapters are over in Idris). It’s full of conflict and tension, with a mysterious character doing interesting things. In short, it’s exactly how you want to begin a book.

My worries aren’t about this prologue so much as they are about the following three chapters, where things slow down a lot. I was tempted to cut this scene and put it in later, but I eventually decided that giving it the mantle of a prologue was enough. A lot of times, particularly in fantasy, we writers use a prologue to highlight a character or conflict that might not show up again for a while.

\subsection*{Naming Vasher}

Vasher’s name has interesting origins. I first began toying with the ideas that became \textit{Warbreaker} back in 2005. I was hanging out with my then girlfriend (not Emily, but Heather, the girl I dated before I met Emily). We were up at Heather’s family’s cabin in Island Park, Idaho, and I had just met her father for the first time. His name was Vance.

The name intrigued me. Yes, I’d heard it before, but for some reason at that moment it struck me. Later that day, sitting on the dock of the lake, I pulled out my notebook and began to play around with ideas for a story. I tweaked the name to Vancer, but that just didn’t sound right, though I used it for a while. The next incarnation was Vasher. [\textit{Editor’s note: Brandon had earlier used the name Vasher in 2003 for a different character in the draft of another novel, but he had completely forgotten that by the time he wrote this annotation.}]

I began doing some preliminary prose writing, plugging in a magic system I’d been working on. (I’ll talk more later about how I came up with Awakening.) It became a story about a guy who was thrown into prison, then used his Awakening magic to get out of it. (Along with the help of his longtime sidekick, whose name escapes me right now.)

It wasn’t very long. I’ll have to dig it out sometime—it’s only handwritten and wasn’t something I ever intended to publish. Just a quick character sketch. It did have the first line, however, of what eventually became this book: “Why does it always have to end up with me getting thrown into prison?”

\subsection*{First Line Origins}

Of course, this line got a tweak of its own in later drafts. I was fond of this first line, as I’d used it in the original short story with Vancer. However, in that story, he’d been thrown into prison for other reasons. In \textit{Warbreaker}, I began the book with Vasher getting himself purposefully tossed into prison.

So, in the end, my editor pointed out that the line no longer worked quite right. We had to change it—why would Vasher complain about getting thrown into prison if he had done it to himself on purpose? So, it became “It’s funny how many things begin with my getting thrown into prison.”

\orn

There are a couple of other relics from the original short story version of this chapter that made it into the final book. I toyed with cutting these, but figured that there were good reasons for them.

\subsection*{The Guard Approaches, and His Clothing Becomes Brighter}

This is an essential part of the magic system. When you get close to someone’s aura, their clothing—and everything else about them—brightens in color slightly. It’s important to show it in this prologue.

Unfortunately, it also shouldn’t be there. You see, Vasher should be smart enough to hide his Breath in his clothing, as the book later shows is quite easy to do. He shouldn’t have left himself holding any Breath. It’s suspicious. If those guards \textit{had} noticed his aura—or if someone working in the prison had been of the First Heightening—Vasher would have been spotted. It’s such an easy fix that he should have thought of it.

The problem is, I felt I needed to establish the way the magic works from the beginning. Having to explain why Vasher didn’t make the clothing glow would have been awkward and confusing at this point in the book. So I left this as it is.

However, being who I am, I developed a background for why Vasher did it this way. He left his Breath in, and thought that maybe it \textit{would} be noticed—but if it was, he knew that the guards would lock him in a cell much closer to Vahr. That would be convenient, as it would ensure that he was much closer to his quarry. Of course, in such a cell, he wouldn’t be able to Awaken anything and escape. However, he’d planned for that too. He set a little straw figure outside the prison the night before, with specific Commands instructing it to search through the cells and find him, delivering a set of lock picks.

It was risky—but either way he did it would be risky. He couldn’t know for certain that the guards would take him to the area he needed to be in, and even if he \textit{had} hidden his Breath in his clothing, some prisons have rules in place requiring each prisoner to be stripped, just in case they’ve done just that. Fortunately, these guards were particularly lazy. Anyway, Vasher’s contingency plan wasn’t needed, as the guards didn’t end up noticing his Breath.

\subsection*{Vasher Awakens the Straw Figure}

I love how intricate and delicate Vasher is in creating the straw figure. The little eyebrow is a nice touch, and forming the creature into the shape of a person has a nice resonance with our own world’s superstitions.

Voodoo dolls, for instance. This is very common in tribal magics and shamanistic rituals—something in the figure of a person, or the figure of the thing it’s supposed to affect, is often seen as being more powerful or more desirable. The same is said for having a drop of blood or a tiny piece of skin, even a piece of hair.

Those two things—making the doll in the shape of a man and using a bit of his own body as a focus—are supposed to create instant resonance in the magic for those reading it. I think it works, too. Unfortunately, there’s a problem with this, much like with the colors above. In later chapters, the characters are generally powerful enough with the magic that they don’t \textit{have} to make things in human shape or use pieces of their own body as a focus.

If I were to write a sequel to the book (and I just might—more on this later) I’d want to get back to these two aspects of the magic. Talk about them more, maybe have characters who have smaller quantities of Breath, and so need to use these tricks to make their Awakening more powerful.

Anyway, this little scene threw all kinds of problems into the book. Later on, I had to decide if I wanted to force the characters to always make things into the shape of a person before Awakening them. That proved impossible, it was \textit{too }limiting on the magic and interfered with action sequences. The same was true for using bits of their own flesh as focuses. It just didn’t work.

I toyed with cutting these things from the prologue. (Again, they are artifacts from the short story I wrote, back when Awakening wasn’t fully developed yet.) However, I like the resonance they give, and think they add a lot of depth to the magic system.

So I made them optional. They’re things that you \textit{can} do to make your Awakenings require fewer Breaths. That lets me have them for resonance, but not talk about them when I don’t need them. I still worry that they set up false expectations for the magic, however.

\orn

What else to say about the prologue? I’ll talk about Nightblood in a future annotation. Let’s see\ldots

\subsection*{Vasher Awakens the Cloak}

He doesn’t end up using it. A lot of people point this out. Him not needing it was intentional. I know it raises a question in the prologue, and seems kind of useless, but it’s there to give some added depth to the scene and the magic. Plus, it was just a smart thing to do. Awakening the cloak to protect him was a precaution—one that didn’t end up being needed, but one of the things that annoys me about books is when every single thing the heroes do ends up being important, useful, or even a hindrance. Sometimes you pack yourself a lunch, but then just don’t end up needing it.

\subsection*{The Straw Figure Returns with the Keys}

Vasher couldn’t have used a thread to unlock the door here, by the way. I know a certain person manages to pull it off later in the book, but that doesn’t happen in the God King’s dungeons.

One thing to remember about designing magic systems—particularly those as important to their societies as mine—is that the people in the world \textit{live} with this magic. They use it and see it being used regularly. They think of it and consider it.

It’s not hard to design a lock that an Awakened thread can’t unlock easily. It \textit{is} more expensive to buy a lock like that, and so not all locks have such precautions. These ones do, however.

If you’ve read the book through, then you know that Vasher’s simple-sounding Command of “Fetch Keys” given to the straw man is incredibly complex. In fact, it’s probably one of the most complicated Commands given to any Awakened object in the entire book. It’s kind of cool to me that Vasher uses it here, showing off incredible mastery of the magic, before anyone reading will even realize how much skill saying those two words correctly really takes.

\subsection*{Vasher Confronts Vahr}

Vahr’s original name was Pahn. You can find it used in earlier drafts of the book. I liked the sound and look of that so much, in fact, that I based the name of the people he came from on his own name.

That made for a problem, though. That’s like having a person named America. It happens, but it’s kind of confusing in a book. So, I eventually had to change his name to something that had a similar look and feel, but which wouldn’t lead to so much confusion.

Vahr dies here, and one of the major revisions I made to the book was to bring out more of his influence throughout the book. I didn’t want it to be \textit{too} in your face. However, he was a very important man. We see only the very tail end of his life here, but he worked for over a decade as a Pahn revolutionary, trying to inspire his people to rebel against Hallandren oppression. (Or at least what he saw as Hallandren oppression.) He eventually became such a popular figure that he raised an army, with monetary support from several of Hallandren’s trade competitors across the sea.

We see here the end of that—Vahr, captured and being tortured. He’s a lot more important than he seems, both to the world and to the novel itself.

Also, if you look, I’ve inputted in the last drafts a little hint here of Vasher being a Returned. He says he could have the Fifth Heightening if he wanted it, which is true. He has his Returned Breath suppressed, but if he let it out, he could instantly have the Fifth Heightening. However, he’d be instantly recognizable as Returned the moment he did that. Plus, he couldn’t use that Returned Breath for Awakening things.



