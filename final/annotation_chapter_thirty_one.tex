% \section{Annotation Chapter Thirty One}
\section*{Annotations}

\subsection*{Vivenna Visits the Idrian Slums}

Vivenna probably should have expected what she would find here. She knows that the slumlords, who are Idrians, run whorehouses and illegal fighting leagues. However, she deluded herself into assuming that they employ Hallandren whores or that the fights aren’t all that bad.

I think this would be a hard thing to come to grips with. It’s happened repeatedly throughout history—a poorer segment travels to a new country and becomes part of the lower, working class. In Korea, they were always complaining about people from Burma coming in and stealing their jobs. I remember hearing the Japanese saying the same thing about Koreans. I’ve heard Americans complain about all three. Things like this have far less to do with culture or race and far more to do with relative economic standing and fluency with the language/culture.

Knowing it happens, however, wouldn’t make it any easier to find your own people in such a state, I think.

Notice that the Idrians here often wear dark clothing. This is partially to hold to their old ways of avoiding colors, but they tend to wear clothes that are black and dark instead of light. (Though there are some who follow the more traditional way.) However, by wearing these dark colors, they completely defeat the original stated purpose of dull clothing—that of removing color to keep Awakeners from using their art.

\subsection*{Vivenna Meets with the Slumlords}

When I write a scene like this, I am never quite certain how much time I want to spend distinguishing the side characters who make an appearance. (Another scene like this is the one where Lightsong plays the game with the three other gods.) Here, we’re introduced to three different slumlords. They all have distinct personalities and different ways of looking at how Vivenna can help them. However, how much time do I spend explaining them and making them have an impact? It’s a tough line to walk. I don’t want to bog the scene down and spend a lot of time on characters you’ll never see again, but I also don’t want the scene to feel ambiguous or lacking precision because you can’t imagine the slumlords.

I suspect that most readers won’t care about telling the difference between the three, so I don’t dwell on it—but I try to give hints that will help those who want to visualize the scene exactly.

Anyway, that’s a tangent. Meeting with these men was a mistake, something that Vivenna realizes partway through the meeting. There is little she can gain from them—and that which she~\textit{could}~gain she’s not prepared to ask for. She should have come with more of a plan. Instead, she did what she’s done for most of the book—that is, pretend that she is in charge and in control, while in fact she’s just floating along with whatever comes at her.

I think this is the big thing Vivenna has to realize in the book. She has never had a good plan of how to deal with things in T’Telir. Unfortunately, I don’t think she can learn until she falls a bit.

\textcolor{red}{
\subsection*{The City Guard Attacks}
}
\textcolor{red}{
Some of you may be wondering whose plot led to this attack by the city guard on the meeting.
}
\textcolor{red}{
Well, it’s complicated. The city watch—worried about the upswing in crime and the political tension lately—has grown more aggressive. They know that someone snuck into the palace of Mercystar herself, threatening one of their goddesses. The watch captain is making a play for a promotion and favor, and is looking to score a major victory to look very good in front of the Returned. He got a tip that three of the most important slumlords—whom he’s been afraid to attack up until now, fearing to commit his guards to action—will be meeting together. He doesn’t even know about Vivenna.
}
\textcolor{red}{
But he did authorize his Lifeless (the city guard has a stock of about fifty that can be used at their discretion) to use deadly force. The Commands weren’t quite specific enough, unfortunately.
}
\textcolor{red}{
Beyond that, Bluefingers has managed—by sneaking through the tunnels that Vasher discovered—to get his forces to Command Break some of the Lifeless in the compound, then insert hidden Commands into them alongside their existing ones. In this case, he wanted the Idrians to see the Lifeless and the city watch cause a slaughter among their people. So he seeded some of the Lifeless with Commands to attack and kill if they were shown aggression by Idrians.
}
\textcolor{red}{
He didn’t know when the slaughter would happen; he doesn’t have enough control over events in order to do that. His little Lifeless bombs just happened to go off here, when the Idrians started to resist. Since the regular soldiers—and even the Lifeless not under Bluefingers’s Commands—overreacted once blood began to be shed, everything went crazy from there.
}
\textcolor{red}{
Denth wasn’t in on this plan, and Bluefingers never told him that he was behind it. In the end, the whole battle turned into a major embarrassment for the city guard, though they did capture one of the slumlords. He was held until after the events of the book, then eventually released.
}


