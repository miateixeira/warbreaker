% \section{Annotation Chapter Fifty-Eight}
\section*{Annotations}

\subsection*{Vasher Finds Vivenna}

I’m torn about this ending. It seems like this last chapter is a little anticlimactic, and yet at the same time, there is still the major conflict of the book to resolve.

Or \textit{is} it the major conflict of the book? Probably not, as I think about it. This book’s major conflicts were character conflicts. Yes, we want to save Idris, and it’s important—but what happens with the characters has overshadowed that. Perhaps that’s why this chapter feels just a bit tacked on. It’s not as bad as the \textit{Well of Ascension} second ending, however, and I think it’s nearly the best way to format this story. That doesn’t stop it from feeling a little extraneous, though.

Anyway, a lot of important things happen here. Note that Nightblood doesn’t remember being drawn. When he was created, the Breaths gave him sentience as planned. (That was a big part of the goal in making him—to prove the existence of Type Four BioChromatic entities.) However, once he is drawn, his Command takes force and he acts much more like a regular Awakened object—but one with very strange abilities and powers. During this time, his Breath is diverted to creating the powers, and his mind goes fuzzy.

\subsection*{Siri and Susebron Visit the Body of Lightsong}

I wanted to have this scene as a little epilogue to Lightsong’s storyline. He was a great character, one of the best I’ve ever written, and I think he fulfilled his place in this book wonderfully.

I often say that I don’t see my endings as sad, even though they do tend to involve the deaths of major characters. In this case, Lightsong’s ending is triumphant because of what he was able to achieve. At least that’s my perspective on it.

What did Blushweaver achieve? In fact, she Returned in the first place to be involved in this ending as well. One thing to note about the Returned coming back is that they \textit{do} see the future, but when they Return, they aren’t guaranteed to be able to change anything. Before her Return, Blushweaver was a powerful merchant in the city, and very well known. She was assassinated after denouncing a group of dye merchants she’d worked with for their deceptive and criminal practices. Her testimony ended with them in jail, but it got her killed. That’s how she earned the title of Blushweaver the Honest (which, if you’ll remember, she eventually got changed to Blushweaver the Beautiful).

She Returned because she didn’t want T’Telir to fall to the invaders she saw taking it after Bluefingers and the others caused their revolt. That was why she gathered the armies. While she didn’t succeed in her quest as well as Lightsong did, she did help out quite a bit. I think she’s pleased, on the other side, with how things turned out.

\subsection*{Vivenna and Vasher Talk about What to Do}

One of the biggest revisions to the ending was what to do with the D’Denir. When first drafting the book, I wasn’t 100% sure on what Awakening could and couldn’t do. I figured that Vasher could have Commands that would Awaken statues, and I wrote the ending that way.

Unfortunately, through revising and developing the story, this ended up not being viable. I was also disappointed in how poorly telegraphed the use of the statues ultimately ended up being. So in revisions, I switched it to make them Lifeless created from bones, something special that Vasher came up with during the Manywar. I then added the concept of Kalad’s Phantoms as a mystery in the book, so that readers would be expecting that army to show up by the end. I think this mitigates the surprise somewhat. (Though not completely; see below.)

\subsection*{Vivenna and Siri Reunite; Vasher Shows Off His Returned Breath}

I believe that this is the \textit{first} time in the book that Vivenna and Siri talk to each other. (Weird, eh?) I knew I couldn’t make their reunion very effusive, since they’re both Idrians, and Siri has learned to control herself. Plus, the situation is very tense. (And beyond that, despite Vivenna’s coming to rescue her sister, the two were never terribly close. They were sisters, but separated by five years or so.)

This chapter focuses on other things, primarily the changes in the God King’s personality and the revelations about Vasher. For the first, I hope they are plausible. Remember, the God King has grown a lot with Siri’s help. Beyond that, he’s been trained to look regal and act like a king, even if he’s not had any practice talking like one. I think he works well here, projecting more confidence and nobility than he really feels, speaking in ways that don’t make him sound too stupid, yet still betraying an innocence.

The bigger surprise is Vasher’s revelation about his nature. I \textit{almost} didn’t put this in the book, instead intending to hint at it and save it for the second book. The reason for this is that I knew it would be confusing.

The big question is, if Vasher is Returned, why can he give away his Breaths and Awaken things without killing himself?

The answer is simple, in many ways, but I’m not sure if I have the groundwork for it properly laid in the book. (Which is why I hesitated in explaining it.) Remember when Denth said that Awakening was all or nothing? Well, he lied. (I think you’ve figured this out now.) A very skilled Awakener can give away only part of their Breath. It depends on their Command visualizations. So Vasher needs to always give away everything except for that one Returned Breath that keeps him alive. As long as he has that one Breath (which he’s learned to suppress and hide), he can stay alive.



