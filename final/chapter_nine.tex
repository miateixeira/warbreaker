\section{Chapter Nine}

Vivenna—firstborn child of Dedelin, king of Idris—gazed upon the grand city of T’Telir. It was the ugliest place she had ever seen.

People jostled their way through the streets, draped flagrantly in colors, yelling, and talking, and moving, and stinking, and coughing, and bumping. Her hair lightened to grey, she pulled her shawl close as she maintained her imitation—such that it was—of an elderly woman. She had feared that she might stand out. She needn’t have worried. Who could ever stand out in this confusion?

Nevertheless, it was best to be safe. She had come—arriving in T’Telir just hours ago—to rescue her sister, not to get herself kidnapped.

It was a bold plan. Vivenna could hardly believe that she’d come up with it. Still, of the many things her tutors had taught her, one was foremost in her mind: A leader was someone who acted. Nobody else was going to help Siri, and so it was up to Vivenna.

She knew that she was inexperienced. She hoped that her awareness of that would keep her from being~\textit{too}~foolhardy, but she had the best education and political tutelage her kingdom could provide, and much of her training had focused on life in Hallandren. As a devout daughter of Austre, she’d practiced all of her life to avoid standing out. She could hide in a vast, disorganized city like T’Telir.

And vast it was. She’d memorized maps, but they hadn’t prepared her for the sight, sound, scent, and~\textit{colors}~of the city on market day. Even the livestock wore bright ribbons. Vivenna stood at the side of the road, stooped beside a building draped in flapping streamers. In front of her, a herdsman drove a small flock of sheep toward the market square. They had each been dyed a different color.~\textit{Won’t that ruin the wool?}~Vivenna thought sourly. The different colors on the animals clashed so terribly that she had to look away.

\textit{Poor Siri,}~she thought.~\textit{Caught up in all of this, locked in the Court of Gods, probably so overwhelmed that she can barely think.}~Vivenna had been trained to deal with the terrors of Hallandren. Though the colors sickened her, she had the fortitude to withstand them. How would little Siri manage?

Vivenna tapped her foot as she stood beside the building in the shadow of a large stone statue.~\textit{Where is that man?}~she thought. Parlin had yet to return from his scouting.

There was nothing to do but wait. She glanced up at the statue beside her; it was one of the famous D’Denir Celabrin. Most of the statues depicted warriors. They stood in every imaginable pose all across the entire city, armed with weapons and often dressed in colorful clothing. According to her lessons, the people of T’Telir found dressing the statues to be an amusing pastime. Lore had it that the first ones had been commissioned by Peacegiver the Blessed, the Returned who had taken command of Hallandren at the end of the Manywar. The number of statues had increased each year as new ones were paid for by the Returned—whose money, of course, came from the people themselves.

\textit{Excess and waste,}~Vivenna thought, shaking her head.

Finally, she noticed Parlin coming back down the street. She frowned as she saw that he was wearing some ridiculous frippery on his head—it looked a little like a sock, though much larger. The bright green hat flopped down one side of his square face, and looked very out of place against his dull brown Idrian travel clothing. Tall but not lanky, Parlin was only a few years Vivenna’s se nior. She’d known him for most of her life; General Yarda’s son had practically grown up in the palace. More recently, he’d been out in the forests, watching the Hallandren border or guarding one of the northern passes.

“Parlin?” she said as he approached, carefully keeping the annoyance out of her voice and her hair. “What is that on your head?”

“A hat,” he said, characteristically terse. It wasn’t that Parlin was rude; it just seemed he rarely felt he had much to say.

“I can see that it’s a hat, Parlin. Where did you get it?”

“The man in the market said they’re very popular.”

Vivenna sighed. She’d hesitated to bring Parlin into the city. He was a good man—as solid and reliable as she’d ever known—but the life he knew was one of living in the wilderness and guarding isolated outposts. The city was probably overwhelming to him.

“The hat is ridiculous, Parlin,” Vivenna said, hair controlled to keep the red out of it. “And makes you stand out.”

Parlin removed the hat, tucking it in his pocket. He said nothing further, but did turn, watching the crowds of people pass. They seemed to make him as nervous as they did Vivenna. Perhaps more so. However, she was glad to have him. He was one of the few people she trusted not to go to her father; she knew that Parlin fancied her. During their youth, he’d often brought her gifts from the forest. Usually, those had taken the form of some animal he’d killed.

To Parlin’s mind, nothing showed affection like a hunk of something dead and bleeding on the table.

“This place is strange,” Parlin said. “People here move like herds.” His eyes followed a pretty Hallandren girl as she walked by. The hussy was—like most of the women in T’Telir—wearing practically nothing. Blouses that were open well below the neck, skirts well above the knees—some women even wore trousers, just like men.

“What did you discover in the market?” she asked, drawing his attention back.

“There are a lot of Idrians here,” he said.

“\textit{What?}” Vivenna said, forgetting herself and showing her shock.

“Idrians,” Parlin said. “In the market. Some were trading goods; many looked like common laborers. I watched them.”

Vivenna frowned, folding her arms. “And the restaurant?” Vivenna asked. “Did you scout it as I asked?”

He nodded. “Looks clean. Feels strange to me that people eat food made by strangers.”

“Did you see anyone suspicious there?”

“What would be ‘suspicious’ in this city?”

“I don’t know. You’re the one who insisted on scouting ahead.”

“It’s always a good idea when hunting. Less likely to scare away the animals.”

“Unfortunately, Parlin,” Vivenna said, “people aren’t like animals.”

“I am aware of that,” Parlin said. “Animals make sense.”

Vivenna sighed. However, she did notice just then that Parlin had been right on at least one count. She caught sight of a group of Idrians walking along the street nearby, one pulling a cart that had probably once held produce. They were easy to distinguish by their muted dress and the slight accent to their voices. It surprised her that they would come so far to trade. But, admittedly, commerce hadn’t been particularly robust in Idris lately.

Reluctantly, she closed her eyes and—using the shawl to hide the transformation—changed her hair from grey to brown. If there were other Idrians in town, it was unlikely that she would stand out. Trying to act like an old woman would be more suspicious.

It still felt wrong to be exposed. In Bevalis, she’d have been recognized instantly. Of course, Bevalis had only a few thousand people in it. The vastly greater scale of T’Telir would require a conscious adjustment.

She gestured to Parlin and—gritting her teeth—joined the crowd and began making her way toward the marketplace.

The inland sea made all the difference. T’Telir was a prime port, and the dyes it sold—made from the Tears of Edgli, a local flower—made it a center of trade. She could see the evidence all around her. Exotic silks and clothing. Brown-skinned traders from Tedradel with their long black beards bound with tight leather cords into cylindrical shapes. Fresh foodstuffs from cities along the coast. In Idris, the population was spread out thinly across the farms and rangelands. In Hallandren—a country that controlled a good third of the inland sea’s coast—things were different. They could burgeon. Grow.

Get flamboyant.

In the distance, she could see the plateau that held the Court of Gods, the most profane place beneath Austre’s colorful eyes. Inside its walls, within the God King’s terrible palace, Siri was being held captive, prisoner of Susebron himself. Logically, Vivenna understood her father’s decision. In raw political terms, Vivenna was more valuable to Idris. If war was certain, it made sense to send the less useful daughter as a stalling tactic.

But it was hard for Vivenna to think of Siri as “less useful.” She was gregarious, but she’d also been the one who smiled when others were down. She was the one who brought gifts when nobody was expecting them. She was infuriating, but also innocent. She was Vivenna’s baby sister, and someone had to look out for her.

The God King would demand an heir. That was to have been Vivenna’s duty—her sacrifice for her people. She had been prepared and willing. It felt~\textit{wrong}~for Siri to have to do something so terrible.

Her father had made his decision; the best one for Idris. Vivenna had made her own. If there was going to be war, then Vivenna wanted to be ready to get her sister out of the city the moment it got dangerous. In fact, Vivenna felt there~\textit{had}~to be a way to rescue Siri before the war came—a way of fooling the Hallandren, making them think that Siri had died. Something that would save Vivenna’s sister, yet not further provoke hostilities.

This wasn’t something her father could condone. So she hadn’t told him. Better for him to be able to deny involvement if things went wrong.

Vivenna moved down the street, eyes downcast, careful to not draw attention to herself. Getting away from Idris had been surprisingly easy. Who would suspect such a brash move from Vivenna—she who had always been perfect? Nobody wondered when she’d asked for food and supplies, explaining that she wanted to make emergency kits. Nobody questioned when she’d proposed an expedition to the higher reaches to gather important roots, an excuse to disguise the first few weeks of her disappearance.

Parlin had been easy enough to persuade. He trusted her, perhaps too much, and he had intimate knowledge of the paths and trails leading down to Hallandren. He’d been as far as the city walls on one scouting trip a year back. With his help, she’d been able to recruit a few of his friends—also woodsmen—to protect her and be part of her “expedition.” She’d sent the rest of them back earlier that morning. They would be of little use in the city, where she had already arranged for other allies to be her protection. Parlin’s friends would carry word to her father, who would already have heard of what she’d done. Before leaving, she’d arranged for her maid to deliver a letter to him. Counting off the days, she realized that her letter would be delivered that very evening.

She didn’t know what her father’s reaction would be. Perhaps he would covertly send soldiers to retrieve her. Perhaps he’d leave her be. She’d warned him that if she saw Idrian soldiers searching for her, she would simply go to the Court of Gods, explain that there had been a mistake, and trade herself for her sister.

She sincerely hoped she wouldn’t have to do that. The God King was not to be trusted; he might take Vivenna captive and keep Siri, thereby gaining two princesses to provide pleasure instead of one.

\textit{Don’t think about that,}~Vivenna told herself, pulling her shawl closer despite the heat.

Better to find another way. The first step was to find Lemex, her father’s chief spy in Hallandren. Vivenna had corresponded with him on several occasions. Her father had wanted her to be familiar with his best intelligence agent in T’Telir, and his foresight would work against him. Lemex knew Vivenna, and had been told to take orders from her. She’d sent the spy a letter—delivered via a messenger with multiple mounts to allow quick delivery—the day she’d left Idris. Assuming the message had arrived safely, the spy would meet her in the appointed restaurant.

Her plan seemed good. She was prepared. Why, then, did she feel so utterly daunted when she entered the market?

She stood quietly, a rock in the stream of human traffic flooding down the street. It was such an enormous expanse, covered in tents, pens, buildings, and people. There were no cobblestones here, only sand and dirt with the occasional patch of grass, and there didn’t appear to be much reason or direction to the arrangement of buildings. The arbitrary streets had simply been made where people felt like going. Merchants yelled out what they sold, banners waved in the wind, and entertainers vied for attention. It was an orgy of color and motion.

“Wow,” Parlin said quietly.

Vivenna turned, shaking off her stupor. “Weren’t you just here?”

“Yeah,” Parlin said, eyes a little glazed over. “Wow again.”

Vivenna shook her head. “Let’s go to the restaurant.”

Parlin nodded. “This way.”

Vivenna followed him, annoyed. This was Hallandren—she shouldn’t be awed by it. She should be disgusted. Yet she was so overwhelmed that it was hard to feel anything beyond a slight sense of sickness. She’d never realized how much she took Idris’s beautiful simplicity for granted.

Parlin’s familiar presence was welcome as the powerful wave of scents, sounds, and sights tried to drown her. In some places the crowds grew so thick that they had to shove their way through. On occasion, Vivenna found herself on the edge of panic, pressed in by dirty, repulsively colored bodies. Blessedly, the restaurant wasn’t too far in, and they arrived just when she thought the sheer excess of the place would make her scream. On its signboard out front, the restaurant had a picture of a boat sailing merrily. If the scents coming from inside were any indication, then the ship represented the restaurant’s cuisine: fish. Vivenna barely kept herself from gagging. She’d eaten fish several times in preparation for her life in Hallandren. She’d never grown to like it.

Parlin walked in, immediately stepping to the side and crouching, almost like a wolf, as he let his eyes adjust to the dimness. Vivenna gave the restaurant keeper the fake name Lemex knew to call her by. The restaurant keeper eyed Parlin, then shrugged and led the two of them to one of the tables on the far side of the room. Vivenna sat down; despite her training, she was a little uncertain what one did at a restaurant. It seemed significant to her that places like restaurants could exist in Hallandren—places meant to feed not travelers, but the locals who couldn’t be bothered to prepare their own food and dine at their own homes.

Parlin didn’t sit, but remained standing beside her chair, watching the room. He looked as tense as she felt. “Vivenna,” he said softly, leaning down. “Your hair.”

She started, realizing that her hair had lightened from the trauma of pushing her way through the crowd. It hadn’t bleached completely white—she was far too well trained for that—but it had grown whiter, as if it had been powdered.

Feeling a jolt of paranoia, Vivenna replaced the shawl on her head, looking away as the restaurant owner approached to take their order. A short list of meals was scratched into the table, and Parlin finally sat down, drawing the restaurant owner’s attention away from Vivenna.

\textit{You’re better than this,}~she told herself sternly.~\textit{You’ve studied Hallandren for most of your life.}~Her hair darkened, returning to its brown. The change was subtle enough that if someone had been watching, they would have probably thought it to be a trick of the light. She kept the shawl up, feeling ashamed. One walk through the market, and she lost control?

\textit{Think of Siri,}~she told herself. That gave her strength. Her mission was impromptu, even reckless, but it was important. Calm once again, she put the shawl back down and waited while Parlin chose a dish—a seafood stew—and the innkeeper walked away.

“Now what?” Parlin asked.

“We wait,” Vivenna said. “In my letter, I told Lemex to check the restaurant each day at noon. We will sit here until he arrives.”

Parlin nodded, fidgeting.

“What is it?” Vivenna asked calmly.

He glanced toward the door. “I don’t trust this place, Vivenna. I can’t smell anything but bodies and spices, can’t hear anything but the chatter of people. There’s no wind, no trees, no rivers, just~.~.~.~\textit{people.}”

“I know.”

“I want to go back outside,” he said.

“What?” she said. “Why?”

“If you aren’t familiar with a place,” he said awkwardly, “you need to become familiar with it.” He gave no other explanation.

Vivenna felt a stab of fear at the thought of being left alone. However, it wasn’t proper to demand Parlin stay and attend her. “Do you promise to stay close?”

He nodded.

“Then go.”

He did, walking from the room. He didn’t move like one of the Hallandren—his motions were too fluid, too much like a prowling beast.~\textit{Perhaps I should have sent him back with the others.}~But the thought of being completely alone had been too much. She needed~\textit{someone}~to help her find Lemex. As it was, she felt that she was probably taking too great a risk at entering the city with only one guard, even one as skilled as Parlin.

But it was done. No use worrying now. She sat, arms folded on the table, thinking. Back in Idris, her plan to save Siri had seemed simpler. Now the true nature of it lay before her. Somehow, she had to get into the Court of Gods and sneak her sister out. How would one accomplish something so audacious? Surely the Court of Gods would be well guarded.

\textit{Lemex will have ideas,}~she told herself.~\textit{We don’t have to do anything yet. I’m—}

A man sat down at her table. Less colorfully dressed than most Hallandren, he wore an outfit made mostly of brown leather, though he did have a token red cloth vest thrown over the top. This was not Lemex. The spy was an older man in his fifties. This stranger had a long face and styled hair, and couldn’t have been older than thirty-five.

“I hate being a mercenary,” the man said. “You know why?”

Shocked, Vivenna sat frozen, mouth opened slightly.

“The prejudice,” the man said. “Everyone else, they work, they ask for recompense, and they are respected for it. Not mercenaries. We get a bad name just for doing our job. How many minstrels get spat on for accepting payment from the highest bidder? How many bakers feel guilty for selling more of pastries to one man, then selling those same pastries to the man’s enemies?” He eyed her. “No. Only the mercenary. Unfair, wouldn’t you say?”

“Wh-who are you?” Vivenna finally managed to ask. She jumped as another man sat down on her other side. Large of girth, this man had a cudgel strapped to his back. A colorful bird was sitting on the end of it.

“I’m Denth,” the first man said, taking her hand and shaking it. “That’s Tonk Fah.”

“Pleased,” Tonk Fah said, taking her hand once Denth was through with it.

“Unfortunately, Princess,” Denth said, “we’re here to kill you.”

