% \section{Annotation Chapter Eighteen}
\section*{Annotations}

\subsection*{Siri Decides to Spite the Priests, Then Reverses That Decision}

This chapter involves a bit of a backslide for both Siri and Lightsong. It was important to establish that they, as characters, are still the same people that you started reading the book about—even if both of them are being forced to change the way they react to things. (Well, at least Siri is being forced to change. Lightsong is more just mulling over what he wants to do. Or not do, as the case may be.)

Siri’s decision here is intended to show just how far she has come during her short time in Hallandren. Siri had all the potential to blossom like this before; she just never had a good reason. With Vivenna there dominating and drawing everyone’s attention, Siri was like a plant growing beneath the shade of an enormous tree—she couldn’t get enough sunlight to grow herself. Freed from that shadow, she’s ready to go.

Her first impulse is very characteristic—it’s the sort of behavior that she’s ingrained in herself for many years. But she decides against it, which should be a big tip-off that she’s capable of much greater things.

\subsection*{Siri Does Her Show for the First Time}

This little sequence is far more discomforting to me than the actual nudity, to be honest. Being somewhat of a prude as I am, I hesitated to put this into the book. I realize that to most readers, it’s not even very risqué. But I’m the one writing the book, and I’m the one who decides what I include. I have to be willing to take responsibility for what’s in my stories.

Why did I put this in if it discomforts me? Well, to be honest, there was no other way. It was what the story demanded. I couldn’t see the priests not at least listening. (And, as I think will be mentioned later in the story, they did have some people watching the first nights—no matter what Bluefingers says in this chapter. He’s not lying, he’s just wrong. The priests would never let a potential assassin near their God King without taking precautions. There was even a soldier hiding under the bed that first night, and another watching from a secret chamber beside the hearth. It was still a risk to let Siri into the room, of course, but they were fairly certain—after taking her clothing and instructing the serving girls to watch carefully during the bathing—that Siri had no weapons on her.)

Regardless, it was ridiculous to think the priests wouldn’t listen in, knowing what they do of the God King. That meant Siri had to either sleep with him for real, or find a way to distract them. This was a clever move on her part, and I like it when my characters can be appropriately clever. And so the scene stays. If I hadn’t allowed her to do this, then I would have—as an author—been holding her back artificially.

\subsection*{Lightsong Refuses to Get Out of Bed}

As I’ve already mentioned, this is a chapter where—after a climactic focal point in the book—the characters backslide a tad in order to enforce that there are still struggles going on. Did I consciously decide this? No, honestly. I sat down to write this chapter, and I felt that I needed to spend a little more time focusing on the conflicts of the characters. So that was intentional. But the placement in the book? That was just by gut instinct.

Llarimar has been holding this little tidbit—the knowledge that he knew Lightsong before the Return—back for just the right moment. He knows his god well, and understands that information like this can be very powerful as a motivator. He’s been waiting for years to use this hint at a time when Lightsong was morose. (And yes, that happens to Lightsong fairly often.) This seemed an important moment to keep the god motivated, so Llarimar doled out the tidbit. Talking about the past of one’s god is unorthodox, and maybe even a little sacrilegious. Fortunately, one of the nice things about being high priest is that, on occasion, you get to subtly redefine what is orthodox and what isn’t.

He did make sure to send the servants away first, though.



