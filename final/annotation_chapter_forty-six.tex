\section{Annotation Chapter Forty-Six}

\subsection*{Vivenna Sits Alone and Thinks about Her Place in the World Now}

Vivenna needed a “Who am I?” moment. I struggled with this chapter because I worried it was simply about a character who sits around and thinks. I tend to put a lot of those into my books, and I don’t want to overdo it. I realize that many readers don’t enjoy those kinds of scenes as much as I do.

The thing is, Vivenna has had so much pulled out from underneath her, she needs time to establish for herself—and for the reader—who she really is. What about her has made the transition? Now we’re getting the real and pure Vivenna, the true woman that she is inside. That determination and, more importantly, that desire to be skilled and competent form the core of her identity.

Now that she’s cast off the trappings, the things she was pretending to be and the excuses she was making, she can take these elements of herself and do something with them.

\subsection*{Vasher Shows Her Some Commands with the Rope}

I’m sorry it took so long in the book to get to a point where we could start exploring the magic system. I wanted to do it differently from the previous two books I’d written. In \textit{Elantris}, we didn’t get to learn about the magic system until the end. In \textit{Mistborn}, we got it straight out. Here, I wanted to try putting it into the middle—to have us experience it and see it work before we got a lot of the rules. Plus, there just wasn’t a good character to show learning about it until now.

Vasher casually mentions that the Idrians used to be Awakeners. That’s true. Before they left, they were as big into Awakening as anyone else—of course, what he doesn’t mention is that Awakening back then was much more new than it is now. It was fresh then, and the Idrians had some very bad experiences with it turning against them. (And what we call Idrians were just one noble house, the Idrian line, those related to the king and his servants.)

\subsection*{Vasher Explains the Different Kinds of BioChromatic Entities}

This is a scene I’d been waiting to write for almost the entire book. Not just because I wanted to get into the scientific rules for Awakening, but because I wanted to pull a good reversal for Vasher. When he begins talking like this, I hope that the reader responds like Vivenna: Who \textit{is} this guy?

A lot of readers, my editor included, resisted the term BioChroma. They wanted me to simply use Breath, as they thought BioChroma was just too scientific sounding. I like this concept, however. I \textit{want} people to read the book and think it sounds scientific. My novels, my magic systems, have a kind of “hard magic” sense to them. I want there to be an edge of science to them, a feeling that people are studying them and trying to learn about them using the scientific method.

Vasher’s explanations here are dead on. He’s got a lot of good information, and he has a handle on what he doesn’t understand. That alone should be a big clue about who he is. The fact that he never has to trim his beard is another one.

\subsection*{Origin of Awakening as a Magic System}

I never did write out in annotation form an explanation of where Awakening came from. I believe I talked about the origin of the term Awakening, but never the actual powers of the magic.

As I’ve said, I wanted to do something that had a very “vulgar magic” feel to it. Something gritty, dealing with the forms of people, like voodoo or hedge magics. I wanted to have something that reached back into our cultural unconscious, and something that dealt with necromancy in a new way.

Those are all pieces of the puzzle. Another piece, however, was the desire to do an animation magic—a magic focused around bringing inanimate objects to life on order to serve you. As I’ve said, it’s very tough to come up with completely new powers nobody has written about or used (though I think I’ve got a few in store for \textit{The Way of Kings}). However, a good magic system can be crafted from the interpretation of old powers used in new ways with interesting limitations and cultural connections.

I’ve seen people bring objects to life in books or movies, but I’ve never seen a formal magic system designed completely around it.

One of the other things I’m always looking for is new ways for people to gain their magical powers. As much as I like \textit{Mistborn}, the “It’s genetic and you’re born with it” method of gaining magical abilities is just about the oldest and most commonly used way. It’s used so much because it makes sense, and because it’s easy to explain. Breath, and its transference, came from my desire to come up with something different—something that had an economic component, something that allowed anyone to become a magic user, but which still had limited resources so that not everyone could be one.

I’m still trying to innovate in this area, but I think my favorite part about Awakening is the concept of Breath and how it’s transferred. It turns people into resources for the magic, but in a way I hadn’t seen done before.



