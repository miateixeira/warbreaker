\section{Annotation Chapter Thirteen}

\subsection*{Timing of This Chapter}

My editor threw me a little curveball in the last edit for this book by asking me if I could move the first Vivenna chapter (the previous one) up a few spots so that she was introduced earlier in the book.

This presented a problem, since I had her arriving, meeting the mercenaries, going to Lemex, then going to see Siri all in the same day. (Though across three chapters.) That meant that I had to move two chapters forward, then, since I didn’t want to break with the mercenaries telling her that they were there to kill her. I wanted to go directly to the next scene with her.

It took a lot of juggling. One of the revisions I had to make was to move this third chapter a day later in the process. She had to arrive, fall asleep, then get up the next morning and have a conversation about giving the Breaths away. Then she had to go see Siri that same day.

I still worry that this jumble caused timing issues. I think I caught them all, but I worry that at one point Lightsong says, “The presentation of the queen is two days away,” then we have Vivenna arrive that same day, then fall asleep and go see Siri the next day. If that’s the case, then the explanation is—unfortunately—that the chapters aren’t happening quite in chronological order.

Usually, I try to make my chapters all chronological, even across different viewpoints. But once in a while, the story is better if they aren’t. The distinction is very hard to pick up. But I think it may happen here. (Note that a lot of authors, like Robert Jordan, don’t strive for chronology—they like it better if the chapters are out of order a little. In a Robert Jordan book, for instance, we’ll often have characters doing things in one chapter, then jump to other characters doing things a few weeks earlier. The chapters are always chronological by viewpoint, but the viewpoints can be off from one another. In fact, he plays with this concept a lot, setting book ten mostly back during the same time as book nine.)

\subsection*{Denth Chats with Her about Breath}

Vivenna and Siri are beginning their role reversals here. Siri is learning to be more reserved—though it’s more that she’s learning to act like a queen. Taking responsibility, being active rather than inactive.

Vivenna is being forced, just a little bit, into inactivity. She thinks she’s doing things, but she’s mostly just reacting. Beyond that, she’s experiencing what it’s like to lose control of her emotions repeatedly.

A few notes here. When Tonk Fah says, “Unless you count pets,” in reference to their team only having three members, he’s not talking about the bird. He’s actually talking about Clod the Lifeless.



<p></p>
<p>\subsection*{Denth’s Motivations Here}</p>
<p>If you’re reading through for the second time, pay close attention to the things Denth says here about Lemex. They’re having a conversation about how Lemex could be a patriot but still steal from the king. Well, Denth is kind of talking about himself here, and not Lemex. He’s hinting that he thinks (or would like to think) that he can both do his job and be a good man at the same time.</p>
<p>These are things he’s struggling with. He tries to tell himself that he doesn’t care, but he does. He has kidnapped Vivenna here without her knowing it, and is very deftly manipulating her. (By the way, Jewels tails her to the assembly meeting, if you were wondering.) He does feel bad about this, just like he feels bad about killing Lemex. That doesn’t stop him from doing things like this, though.</p>
<p>He does plan to get Vivenna’s Breath. He knows, however, that in the end he can probably just torture her into giving it to him. In this scene, if you could see into his head, he’s trying to figure out how exactly he can get her to give it to him without having to hurt her.</p>
<p>He doesn’t really believe he can do it, though. Life has proven to Denth lately that he just has to do bad things. He almost sees it as inevitable.</p>
<p></p>



