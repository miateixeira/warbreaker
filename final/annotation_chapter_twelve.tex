\section{Annotation Chapter Twelve}

\subsection*{Lightsong Hears Petitions}

The concept of petitions—and the gods being able to heal someone one time—grew out of my desire to have something about them that \textit{was} miraculous. Something obvious, something more than just an ability to make vague prophecies. Their Breath auras are amazing, true, but an Awakener with a lot of Breath can replicate that.



I’ve always thought it was interesting conceptually, however, so I developed it into this book as an aspect of Returned that makes them different. They can create one miracle—and in this world, that one miracle \textit{has} to be a healing. They can expend their divine Breath to heal someone.

This created another problem for readers, however. It became very difficult in the book to explain to them that a Returned could still Awaken things—but not by using the Breath granted to them by their Return. In other words, if a Returned gained a hundred extra Breaths, they could use them just like anyone else’s. But if they give away the Breath they start with, it kills them.

Every person starts with a Breath. Well, Returned start with one too—a divine Breath that can be given away to heal someone else’s Breath that is weakening and dying. That’s what these petitioners are asking for.

But regular Breaths, they can give those away. They just have to be tricky about it.

\subsection*{Siri Realizes That She Needs to Be Proactive}

As I said in the other section, I think that Siri’s plot here is just a tad accelerated from what I’d like—but that’s necessary. Nothing is worse in a book than a character who never does anything. She needed to get through her fear and her worry and decide to become proactive. It was only then that interesting things could start to happen in her storyline.

So, I pushed through the moments of indecisiveness and inaction as fast as I could, getting to this moment where she decides to change. I feel that her character being what it is (impulsive and determined) justifies her quickly deciding to take responsibility for herself, now that she’s been placed into a situation of great stress.



